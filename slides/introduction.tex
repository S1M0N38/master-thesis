\begin{frame}{Introduction | Framework}
  \begin{center}
    \huge{Models for images classification\\}
    \only<1>{\huge{ignore\\}}
    \only<2>{\huge{\alert{exploit}\\}}
    \huge{relashionships between classes}
  \end{center}
  \note[item]{I modelli per la classificatione di immagini generamente ignorano
  le relazioni tra le classi.}
  \note[item]{NEXT}
  \note[item]{Noi abbiamo studiato e sviluppato modelli che
  invece le sfruttano.}
\end{frame}

\begin{frame}{Introduction | Framework}
  \tikzfig{figures/framework}
  \vspace{1.5em}
  \pause
  \begin{center}
    \large{We fix $\psi$ and play with $\phi$ and $\mathcal{L}$}
  \end{center}

  \note[item]{Un modello per la classificatione di immagini è un funzione
  $\psi_\theta$ che prende come input un'immagine e restituisce un vettore di
  output. La classe associata all'immagine è convertita da testo a numeri dalla
  funzione $\phi$ producendo il vettore di encoding. La funzione di Loss
  $\mathcal{L}$ quantifica con un numero la dissimilarità tra output del
  modello e corrispondente encoding, minimizzarla significa avviciare l'output
  del modello alla codifica della classe corretta e quindi migliorare la
  classificazione del modello. L'argoritmo di backpropagation minimizza in modo
  iterativo $\mathcal{L}$ agendo sui i parametri del modello $\theta$.}
  \note[item]{Generalmente si sperimenta modificando l'architettura del modello
  ovvero cambiando la forma funzionale di $\psi$.}
  \note[item]{NEXT}
  \note[item]{Noi invece abbiamo fissato $\psi$ e variato $\phi$ e
  $\mathcal{L}$.}
\end{frame}

\begin{frame}{Introduction | One-hot Enconding}
  \begin{align*}
    \phi &:
    \mathcal{C} \rightarrow {\left\{0, 1\right\}}^{|\mathcal{C}|} :
    c_i \mapsto \phi \, (c_i)
    \quad &\text{where}& \quad
    {\phi \, (c_i)}_j := \delta_{i,j} \\
    \psi_\theta &:
    \mathcal{X} \rightarrow {\left[0, 1\right]}^{|\mathcal{C}|} :
    x \mapsto \psi_\theta \, (x)
    \quad &\text{where}& \quad 
    \sum_{j=1}^{|\mathcal{C}|} {\psi_\theta \, (x)}_j = 1
  \end{align*}

  \metroset{block=fill}
  \begin{block}{Example}
    \begin{align*}
      \mathcal{C} &= [
        &\texttt{lemon} &,  &\texttt{pear} &,  &\texttt{apple} &,
        &\texttt{dog} &, &\texttt{cat} &, &\texttt{car} &\
      ] \\
      \phi\,\left(\texttt{apple}\right) &= [
        & 0 &, & 0 &, & 1 &,
        & 0 &, & 0 &, & 0 &\
      ] \\
      \psi_\theta \,\left(x\right) &= [
        & 0.2 &, & 0.3 &, & 0.1 &,
        & 0.15 &, & 0.05 &, & 0.2 &\
      ]
    \end{align*}

    \note[item]{La più semplice e diffusa funzione di encoding è One-hot
    encoding, una funzione $\phi$ che mappa dalla lista delle classi a un
    vettore binario le cui componenti sono nulle ad eccezione di quella il cui
    indice coincide con l'indice della classe mappata.}
    \note[item]{Introducendo un esempio che ci accompagnerà nel proseguio, se
    le classi del dataset sono "limone", "pera", "mela", "cane", "gatto" e
    "auto", $\phi(\texttt{apple})$, l'encoding della classe "mela", sarà un
    vettore di zeri con un "uno" nella terza componente.}
    \note[item]{Il modello mappa invece dallo spazio delle immagini a una
    "probability mass function" sull'iniseme delle classi.}
    \note[item]{L'indice della componente maggiore del output è la classe
    predetta dal modello.}
    \note[item]{Inizialmente non c'è correlazione tra input e classe predetta,
    in quanto in parametri sono inizializzati in modo casuale. Solo in seguito
    ad un processo di training è possibile produrre una corretta
    classificazione.}
  \end{block}


\end{frame}

\begin{frame}{Introduction | Cross Entropy}

  \begin{equation*}
    \mathcal{L} \, \left(p, q\right) := - q \cdot \log p
    \quad \Longrightarrow \quad
    \mathcal{L} \, \left(\psi_\theta(x), \phi(c_i)\right) =
    - \log \left({\psi_\theta(x)}_i\right)
  \end{equation*}

  \pause
  \only<2>{
    \vspace{-0.0cm}
    %% Creator: Matplotlib, PGF backend
%%
%% To include the figure in your LaTeX document, write
%%   \input{<filename>.pgf}
%%
%% Make sure the required packages are loaded in your preamble
%%   \usepackage{pgf}
%%
%% Also ensure that all the required font packages are loaded; for instance,
%% the lmodern package is sometimes necessary when using math font.
%%   \usepackage{lmodern}
%%
%% Figures using additional raster images can only be included by \input if
%% they are in the same directory as the main LaTeX file. For loading figures
%% from other directories you can use the `import` package
%%   \usepackage{import}
%%
%% and then include the figures with
%%   \import{<path to file>}{<filename>.pgf}
%%
%% Matplotlib used the following preamble
%%   
%%   \usepackage{fontspec}
%%   \setmainfont{DejaVuSerif.ttf}[Path=\detokenize{/Users/simo/.local/share/virtualenvs/master-thesis-code/lib/python3.10/site-packages/matplotlib/mpl-data/fonts/ttf/}]
%%   \setsansfont{DejaVuSans.ttf}[Path=\detokenize{/Users/simo/.local/share/virtualenvs/master-thesis-code/lib/python3.10/site-packages/matplotlib/mpl-data/fonts/ttf/}]
%%   \setmonofont{DejaVuSansMono.ttf}[Path=\detokenize{/Users/simo/.local/share/virtualenvs/master-thesis-code/lib/python3.10/site-packages/matplotlib/mpl-data/fonts/ttf/}]
%%   \makeatletter\@ifpackageloaded{underscore}{}{\usepackage[strings]{underscore}}\makeatother
%%
\begingroup%
\makeatletter%
\begin{pgfpicture}%
\pgfpathrectangle{\pgfpointorigin}{\pgfqpoint{4.251970in}{2.100000in}}%
\pgfusepath{use as bounding box, clip}%
\begin{pgfscope}%
\pgfsetbuttcap%
\pgfsetmiterjoin%
\definecolor{currentfill}{rgb}{0.980392,0.980392,0.980392}%
\pgfsetfillcolor{currentfill}%
\pgfsetlinewidth{0.000000pt}%
\definecolor{currentstroke}{rgb}{1.000000,1.000000,1.000000}%
\pgfsetstrokecolor{currentstroke}%
\pgfsetdash{}{0pt}%
\pgfpathmoveto{\pgfqpoint{0.000000in}{0.000000in}}%
\pgfpathlineto{\pgfqpoint{4.251970in}{0.000000in}}%
\pgfpathlineto{\pgfqpoint{4.251970in}{2.100000in}}%
\pgfpathlineto{\pgfqpoint{0.000000in}{2.100000in}}%
\pgfpathlineto{\pgfqpoint{0.000000in}{0.000000in}}%
\pgfpathclose%
\pgfusepath{fill}%
\end{pgfscope}%
\begin{pgfscope}%
\pgfsetbuttcap%
\pgfsetmiterjoin%
\definecolor{currentfill}{rgb}{0.917647,0.917647,0.949020}%
\pgfsetfillcolor{currentfill}%
\pgfsetlinewidth{0.000000pt}%
\definecolor{currentstroke}{rgb}{0.000000,0.000000,0.000000}%
\pgfsetstrokecolor{currentstroke}%
\pgfsetstrokeopacity{0.000000}%
\pgfsetdash{}{0pt}%
\pgfpathmoveto{\pgfqpoint{0.439176in}{0.313488in}}%
\pgfpathlineto{\pgfqpoint{4.101970in}{0.313488in}}%
\pgfpathlineto{\pgfqpoint{4.101970in}{1.950000in}}%
\pgfpathlineto{\pgfqpoint{0.439176in}{1.950000in}}%
\pgfpathlineto{\pgfqpoint{0.439176in}{0.313488in}}%
\pgfpathclose%
\pgfusepath{fill}%
\end{pgfscope}%
\begin{pgfscope}%
\definecolor{textcolor}{rgb}{0.137255,0.215686,0.231373}%
\pgfsetstrokecolor{textcolor}%
\pgfsetfillcolor{textcolor}%
\pgftext[x=0.792771in,y=0.174599in,,base]{\color{textcolor}\sffamily\fontsize{8.000000}{9.600000}\selectfont \texttt{lemon}}%
\end{pgfscope}%
\begin{pgfscope}%
\definecolor{textcolor}{rgb}{0.137255,0.215686,0.231373}%
\pgfsetstrokecolor{textcolor}%
\pgfsetfillcolor{textcolor}%
\pgftext[x=1.427021in,y=0.174599in,,base]{\color{textcolor}\sffamily\fontsize{8.000000}{9.600000}\selectfont \texttt{pear}}%
\end{pgfscope}%
\begin{pgfscope}%
\definecolor{textcolor}{rgb}{0.137255,0.215686,0.231373}%
\pgfsetstrokecolor{textcolor}%
\pgfsetfillcolor{textcolor}%
\pgftext[x=2.061271in,y=0.174599in,,base]{\color{textcolor}\sffamily\fontsize{8.000000}{9.600000}\selectfont \texttt{apple}}%
\end{pgfscope}%
\begin{pgfscope}%
\definecolor{textcolor}{rgb}{0.137255,0.215686,0.231373}%
\pgfsetstrokecolor{textcolor}%
\pgfsetfillcolor{textcolor}%
\pgftext[x=2.695521in,y=0.174599in,,base]{\color{textcolor}\sffamily\fontsize{8.000000}{9.600000}\selectfont \texttt{dog}}%
\end{pgfscope}%
\begin{pgfscope}%
\definecolor{textcolor}{rgb}{0.137255,0.215686,0.231373}%
\pgfsetstrokecolor{textcolor}%
\pgfsetfillcolor{textcolor}%
\pgftext[x=3.329771in,y=0.174599in,,base]{\color{textcolor}\sffamily\fontsize{8.000000}{9.600000}\selectfont \texttt{cat}}%
\end{pgfscope}%
\begin{pgfscope}%
\definecolor{textcolor}{rgb}{0.137255,0.215686,0.231373}%
\pgfsetstrokecolor{textcolor}%
\pgfsetfillcolor{textcolor}%
\pgftext[x=3.964021in,y=0.174599in,,base]{\color{textcolor}\sffamily\fontsize{8.000000}{9.600000}\selectfont \texttt{car}}%
\end{pgfscope}%
\begin{pgfscope}%
\definecolor{textcolor}{rgb}{0.137255,0.215686,0.231373}%
\pgfsetstrokecolor{textcolor}%
\pgfsetfillcolor{textcolor}%
\pgftext[x=0.149436in, y=0.271279in, left, base]{\color{textcolor}\sffamily\fontsize{8.000000}{9.600000}\selectfont \(\displaystyle {0.0}\)}%
\end{pgfscope}%
\begin{pgfscope}%
\definecolor{textcolor}{rgb}{0.137255,0.215686,0.231373}%
\pgfsetstrokecolor{textcolor}%
\pgfsetfillcolor{textcolor}%
\pgftext[x=0.149436in, y=0.568826in, left, base]{\color{textcolor}\sffamily\fontsize{8.000000}{9.600000}\selectfont \(\displaystyle {0.2}\)}%
\end{pgfscope}%
\begin{pgfscope}%
\definecolor{textcolor}{rgb}{0.137255,0.215686,0.231373}%
\pgfsetstrokecolor{textcolor}%
\pgfsetfillcolor{textcolor}%
\pgftext[x=0.149436in, y=0.866374in, left, base]{\color{textcolor}\sffamily\fontsize{8.000000}{9.600000}\selectfont \(\displaystyle {0.4}\)}%
\end{pgfscope}%
\begin{pgfscope}%
\definecolor{textcolor}{rgb}{0.137255,0.215686,0.231373}%
\pgfsetstrokecolor{textcolor}%
\pgfsetfillcolor{textcolor}%
\pgftext[x=0.149436in, y=1.163922in, left, base]{\color{textcolor}\sffamily\fontsize{8.000000}{9.600000}\selectfont \(\displaystyle {0.6}\)}%
\end{pgfscope}%
\begin{pgfscope}%
\definecolor{textcolor}{rgb}{0.137255,0.215686,0.231373}%
\pgfsetstrokecolor{textcolor}%
\pgfsetfillcolor{textcolor}%
\pgftext[x=0.149436in, y=1.461469in, left, base]{\color{textcolor}\sffamily\fontsize{8.000000}{9.600000}\selectfont \(\displaystyle {0.8}\)}%
\end{pgfscope}%
\begin{pgfscope}%
\definecolor{textcolor}{rgb}{0.137255,0.215686,0.231373}%
\pgfsetstrokecolor{textcolor}%
\pgfsetfillcolor{textcolor}%
\pgftext[x=0.149436in, y=1.759017in, left, base]{\color{textcolor}\sffamily\fontsize{8.000000}{9.600000}\selectfont \(\displaystyle {1.0}\)}%
\end{pgfscope}%
\begin{pgfscope}%
\pgfpathrectangle{\pgfqpoint{0.439176in}{0.313488in}}{\pgfqpoint{3.662794in}{1.636512in}}%
\pgfusepath{clip}%
\pgfsetbuttcap%
\pgfsetmiterjoin%
\definecolor{currentfill}{rgb}{0.623529,0.780392,1.000000}%
\pgfsetfillcolor{currentfill}%
\pgfsetlinewidth{1.505625pt}%
\definecolor{currentstroke}{rgb}{0.298039,0.447059,0.690196}%
\pgfsetstrokecolor{currentstroke}%
\pgfsetdash{}{0pt}%
\pgfpathmoveto{\pgfqpoint{2.089812in}{0.313488in}}%
\pgfpathlineto{\pgfqpoint{2.248374in}{0.313488in}}%
\pgfpathlineto{\pgfqpoint{2.248374in}{1.801226in}}%
\pgfpathlineto{\pgfqpoint{2.089812in}{1.801226in}}%
\pgfpathlineto{\pgfqpoint{2.089812in}{0.313488in}}%
\pgfpathclose%
\pgfusepath{stroke,fill}%
\end{pgfscope}%
\begin{pgfscope}%
\pgfpathrectangle{\pgfqpoint{0.439176in}{0.313488in}}{\pgfqpoint{3.662794in}{1.636512in}}%
\pgfusepath{clip}%
\pgfsetbuttcap%
\pgfsetmiterjoin%
\definecolor{currentfill}{rgb}{1.000000,0.756863,0.529412}%
\pgfsetfillcolor{currentfill}%
\pgfsetlinewidth{1.505625pt}%
\definecolor{currentstroke}{rgb}{0.921569,0.505882,0.105882}%
\pgfsetstrokecolor{currentstroke}%
\pgfsetdash{}{0pt}%
\pgfpathmoveto{\pgfqpoint{0.605667in}{0.313488in}}%
\pgfpathlineto{\pgfqpoint{0.764229in}{0.313488in}}%
\pgfpathlineto{\pgfqpoint{0.764229in}{0.611036in}}%
\pgfpathlineto{\pgfqpoint{0.605667in}{0.611036in}}%
\pgfpathlineto{\pgfqpoint{0.605667in}{0.313488in}}%
\pgfpathclose%
\pgfusepath{stroke,fill}%
\end{pgfscope}%
\begin{pgfscope}%
\pgfpathrectangle{\pgfqpoint{0.439176in}{0.313488in}}{\pgfqpoint{3.662794in}{1.636512in}}%
\pgfusepath{clip}%
\pgfsetbuttcap%
\pgfsetmiterjoin%
\definecolor{currentfill}{rgb}{1.000000,0.756863,0.529412}%
\pgfsetfillcolor{currentfill}%
\pgfsetlinewidth{1.505625pt}%
\definecolor{currentstroke}{rgb}{0.921569,0.505882,0.105882}%
\pgfsetstrokecolor{currentstroke}%
\pgfsetdash{}{0pt}%
\pgfpathmoveto{\pgfqpoint{1.239917in}{0.313488in}}%
\pgfpathlineto{\pgfqpoint{1.398479in}{0.313488in}}%
\pgfpathlineto{\pgfqpoint{1.398479in}{0.759809in}}%
\pgfpathlineto{\pgfqpoint{1.239917in}{0.759809in}}%
\pgfpathlineto{\pgfqpoint{1.239917in}{0.313488in}}%
\pgfpathclose%
\pgfusepath{stroke,fill}%
\end{pgfscope}%
\begin{pgfscope}%
\pgfpathrectangle{\pgfqpoint{0.439176in}{0.313488in}}{\pgfqpoint{3.662794in}{1.636512in}}%
\pgfusepath{clip}%
\pgfsetbuttcap%
\pgfsetmiterjoin%
\definecolor{currentfill}{rgb}{1.000000,0.756863,0.529412}%
\pgfsetfillcolor{currentfill}%
\pgfsetlinewidth{1.505625pt}%
\definecolor{currentstroke}{rgb}{0.921569,0.505882,0.105882}%
\pgfsetstrokecolor{currentstroke}%
\pgfsetdash{}{0pt}%
\pgfpathmoveto{\pgfqpoint{1.874167in}{0.313488in}}%
\pgfpathlineto{\pgfqpoint{2.032729in}{0.313488in}}%
\pgfpathlineto{\pgfqpoint{2.032729in}{0.462262in}}%
\pgfpathlineto{\pgfqpoint{1.874167in}{0.462262in}}%
\pgfpathlineto{\pgfqpoint{1.874167in}{0.313488in}}%
\pgfpathclose%
\pgfusepath{stroke,fill}%
\end{pgfscope}%
\begin{pgfscope}%
\pgfpathrectangle{\pgfqpoint{0.439176in}{0.313488in}}{\pgfqpoint{3.662794in}{1.636512in}}%
\pgfusepath{clip}%
\pgfsetbuttcap%
\pgfsetmiterjoin%
\definecolor{currentfill}{rgb}{1.000000,0.756863,0.529412}%
\pgfsetfillcolor{currentfill}%
\pgfsetlinewidth{1.505625pt}%
\definecolor{currentstroke}{rgb}{0.921569,0.505882,0.105882}%
\pgfsetstrokecolor{currentstroke}%
\pgfsetdash{}{0pt}%
\pgfpathmoveto{\pgfqpoint{2.508417in}{0.313488in}}%
\pgfpathlineto{\pgfqpoint{2.666979in}{0.313488in}}%
\pgfpathlineto{\pgfqpoint{2.666979in}{0.536649in}}%
\pgfpathlineto{\pgfqpoint{2.508417in}{0.536649in}}%
\pgfpathlineto{\pgfqpoint{2.508417in}{0.313488in}}%
\pgfpathclose%
\pgfusepath{stroke,fill}%
\end{pgfscope}%
\begin{pgfscope}%
\pgfpathrectangle{\pgfqpoint{0.439176in}{0.313488in}}{\pgfqpoint{3.662794in}{1.636512in}}%
\pgfusepath{clip}%
\pgfsetbuttcap%
\pgfsetmiterjoin%
\definecolor{currentfill}{rgb}{1.000000,0.756863,0.529412}%
\pgfsetfillcolor{currentfill}%
\pgfsetlinewidth{1.505625pt}%
\definecolor{currentstroke}{rgb}{0.921569,0.505882,0.105882}%
\pgfsetstrokecolor{currentstroke}%
\pgfsetdash{}{0pt}%
\pgfpathmoveto{\pgfqpoint{3.142667in}{0.313488in}}%
\pgfpathlineto{\pgfqpoint{3.301229in}{0.313488in}}%
\pgfpathlineto{\pgfqpoint{3.301229in}{0.387875in}}%
\pgfpathlineto{\pgfqpoint{3.142667in}{0.387875in}}%
\pgfpathlineto{\pgfqpoint{3.142667in}{0.313488in}}%
\pgfpathclose%
\pgfusepath{stroke,fill}%
\end{pgfscope}%
\begin{pgfscope}%
\pgfpathrectangle{\pgfqpoint{0.439176in}{0.313488in}}{\pgfqpoint{3.662794in}{1.636512in}}%
\pgfusepath{clip}%
\pgfsetbuttcap%
\pgfsetmiterjoin%
\definecolor{currentfill}{rgb}{1.000000,0.756863,0.529412}%
\pgfsetfillcolor{currentfill}%
\pgfsetlinewidth{1.505625pt}%
\definecolor{currentstroke}{rgb}{0.921569,0.505882,0.105882}%
\pgfsetstrokecolor{currentstroke}%
\pgfsetdash{}{0pt}%
\pgfpathmoveto{\pgfqpoint{3.776917in}{0.313488in}}%
\pgfpathlineto{\pgfqpoint{3.935479in}{0.313488in}}%
\pgfpathlineto{\pgfqpoint{3.935479in}{0.611036in}}%
\pgfpathlineto{\pgfqpoint{3.776917in}{0.611036in}}%
\pgfpathlineto{\pgfqpoint{3.776917in}{0.313488in}}%
\pgfpathclose%
\pgfusepath{stroke,fill}%
\end{pgfscope}%
\begin{pgfscope}%
\pgfsetrectcap%
\pgfsetmiterjoin%
\pgfsetlinewidth{1.003750pt}%
\definecolor{currentstroke}{rgb}{0.917647,0.917647,0.949020}%
\pgfsetstrokecolor{currentstroke}%
\pgfsetdash{}{0pt}%
\pgfpathmoveto{\pgfqpoint{0.439176in}{0.313488in}}%
\pgfpathlineto{\pgfqpoint{0.439176in}{1.950000in}}%
\pgfusepath{stroke}%
\end{pgfscope}%
\begin{pgfscope}%
\pgfsetrectcap%
\pgfsetmiterjoin%
\pgfsetlinewidth{1.003750pt}%
\definecolor{currentstroke}{rgb}{0.917647,0.917647,0.949020}%
\pgfsetstrokecolor{currentstroke}%
\pgfsetdash{}{0pt}%
\pgfpathmoveto{\pgfqpoint{4.101970in}{0.313488in}}%
\pgfpathlineto{\pgfqpoint{4.101970in}{1.950000in}}%
\pgfusepath{stroke}%
\end{pgfscope}%
\begin{pgfscope}%
\pgfsetrectcap%
\pgfsetmiterjoin%
\pgfsetlinewidth{1.003750pt}%
\definecolor{currentstroke}{rgb}{0.917647,0.917647,0.949020}%
\pgfsetstrokecolor{currentstroke}%
\pgfsetdash{}{0pt}%
\pgfpathmoveto{\pgfqpoint{0.439176in}{0.313488in}}%
\pgfpathlineto{\pgfqpoint{4.101970in}{0.313488in}}%
\pgfusepath{stroke}%
\end{pgfscope}%
\begin{pgfscope}%
\pgfsetrectcap%
\pgfsetmiterjoin%
\pgfsetlinewidth{1.003750pt}%
\definecolor{currentstroke}{rgb}{0.917647,0.917647,0.949020}%
\pgfsetstrokecolor{currentstroke}%
\pgfsetdash{}{0pt}%
\pgfpathmoveto{\pgfqpoint{0.439176in}{1.950000in}}%
\pgfpathlineto{\pgfqpoint{4.101970in}{1.950000in}}%
\pgfusepath{stroke}%
\end{pgfscope}%
\begin{pgfscope}%
\pgfsetbuttcap%
\pgfsetmiterjoin%
\definecolor{currentfill}{rgb}{1.000000,0.756863,0.529412}%
\pgfsetfillcolor{currentfill}%
\pgfsetlinewidth{1.505625pt}%
\definecolor{currentstroke}{rgb}{0.921569,0.505882,0.105882}%
\pgfsetstrokecolor{currentstroke}%
\pgfsetdash{}{0pt}%
\pgfpathmoveto{\pgfqpoint{2.932915in}{1.719477in}}%
\pgfpathlineto{\pgfqpoint{3.210693in}{1.719477in}}%
\pgfpathlineto{\pgfqpoint{3.210693in}{1.816699in}}%
\pgfpathlineto{\pgfqpoint{2.932915in}{1.816699in}}%
\pgfpathlineto{\pgfqpoint{2.932915in}{1.719477in}}%
\pgfpathclose%
\pgfusepath{stroke,fill}%
\end{pgfscope}%
\begin{pgfscope}%
\definecolor{textcolor}{rgb}{0.137255,0.215686,0.231373}%
\pgfsetstrokecolor{textcolor}%
\pgfsetfillcolor{textcolor}%
\pgftext[x=3.321804in,y=1.719477in,left,base]{\color{textcolor}\sffamily\fontsize{10.000000}{12.000000}\selectfont \(\displaystyle \psi_\theta\,\left(x\right)\)}%
\end{pgfscope}%
\begin{pgfscope}%
\pgfsetbuttcap%
\pgfsetmiterjoin%
\definecolor{currentfill}{rgb}{0.623529,0.780392,1.000000}%
\pgfsetfillcolor{currentfill}%
\pgfsetlinewidth{1.505625pt}%
\definecolor{currentstroke}{rgb}{0.298039,0.447059,0.690196}%
\pgfsetstrokecolor{currentstroke}%
\pgfsetdash{}{0pt}%
\pgfpathmoveto{\pgfqpoint{2.932915in}{1.509041in}}%
\pgfpathlineto{\pgfqpoint{3.210693in}{1.509041in}}%
\pgfpathlineto{\pgfqpoint{3.210693in}{1.606263in}}%
\pgfpathlineto{\pgfqpoint{2.932915in}{1.606263in}}%
\pgfpathlineto{\pgfqpoint{2.932915in}{1.509041in}}%
\pgfpathclose%
\pgfusepath{stroke,fill}%
\end{pgfscope}%
\begin{pgfscope}%
\definecolor{textcolor}{rgb}{0.137255,0.215686,0.231373}%
\pgfsetstrokecolor{textcolor}%
\pgfsetfillcolor{textcolor}%
\pgftext[x=3.321804in,y=1.509041in,left,base]{\color{textcolor}\sffamily\fontsize{10.000000}{12.000000}\selectfont \(\displaystyle \phi\,\left(\texttt{apple}\right)\)}%
\end{pgfscope}%
\end{pgfpicture}%
\makeatother%
\endgroup%

  }
  \only<3>{
    \vspace{-0.0cm}
    %% Creator: Matplotlib, PGF backend
%%
%% To include the figure in your LaTeX document, write
%%   \input{<filename>.pgf}
%%
%% Make sure the required packages are loaded in your preamble
%%   \usepackage{pgf}
%%
%% Also ensure that all the required font packages are loaded; for instance,
%% the lmodern package is sometimes necessary when using math font.
%%   \usepackage{lmodern}
%%
%% Figures using additional raster images can only be included by \input if
%% they are in the same directory as the main LaTeX file. For loading figures
%% from other directories you can use the `import` package
%%   \usepackage{import}
%%
%% and then include the figures with
%%   \import{<path to file>}{<filename>.pgf}
%%
%% Matplotlib used the following preamble
%%   
%%   \usepackage{fontspec}
%%   \setmainfont{DejaVuSerif.ttf}[Path=\detokenize{/Users/simo/.local/share/virtualenvs/master-thesis-code/lib/python3.10/site-packages/matplotlib/mpl-data/fonts/ttf/}]
%%   \setsansfont{DejaVuSans.ttf}[Path=\detokenize{/Users/simo/.local/share/virtualenvs/master-thesis-code/lib/python3.10/site-packages/matplotlib/mpl-data/fonts/ttf/}]
%%   \setmonofont{DejaVuSansMono.ttf}[Path=\detokenize{/Users/simo/.local/share/virtualenvs/master-thesis-code/lib/python3.10/site-packages/matplotlib/mpl-data/fonts/ttf/}]
%%   \makeatletter\@ifpackageloaded{underscore}{}{\usepackage[strings]{underscore}}\makeatother
%%
\begingroup%
\makeatletter%
\begin{pgfpicture}%
\pgfpathrectangle{\pgfpointorigin}{\pgfqpoint{4.251970in}{2.100000in}}%
\pgfusepath{use as bounding box, clip}%
\begin{pgfscope}%
\pgfsetbuttcap%
\pgfsetmiterjoin%
\definecolor{currentfill}{rgb}{0.980392,0.980392,0.980392}%
\pgfsetfillcolor{currentfill}%
\pgfsetlinewidth{0.000000pt}%
\definecolor{currentstroke}{rgb}{1.000000,1.000000,1.000000}%
\pgfsetstrokecolor{currentstroke}%
\pgfsetdash{}{0pt}%
\pgfpathmoveto{\pgfqpoint{0.000000in}{0.000000in}}%
\pgfpathlineto{\pgfqpoint{4.251970in}{0.000000in}}%
\pgfpathlineto{\pgfqpoint{4.251970in}{2.100000in}}%
\pgfpathlineto{\pgfqpoint{0.000000in}{2.100000in}}%
\pgfpathlineto{\pgfqpoint{0.000000in}{0.000000in}}%
\pgfpathclose%
\pgfusepath{fill}%
\end{pgfscope}%
\begin{pgfscope}%
\pgfsetbuttcap%
\pgfsetmiterjoin%
\definecolor{currentfill}{rgb}{0.917647,0.917647,0.949020}%
\pgfsetfillcolor{currentfill}%
\pgfsetlinewidth{0.000000pt}%
\definecolor{currentstroke}{rgb}{0.000000,0.000000,0.000000}%
\pgfsetstrokecolor{currentstroke}%
\pgfsetstrokeopacity{0.000000}%
\pgfsetdash{}{0pt}%
\pgfpathmoveto{\pgfqpoint{0.439176in}{0.313488in}}%
\pgfpathlineto{\pgfqpoint{4.101970in}{0.313488in}}%
\pgfpathlineto{\pgfqpoint{4.101970in}{1.950000in}}%
\pgfpathlineto{\pgfqpoint{0.439176in}{1.950000in}}%
\pgfpathlineto{\pgfqpoint{0.439176in}{0.313488in}}%
\pgfpathclose%
\pgfusepath{fill}%
\end{pgfscope}%
\begin{pgfscope}%
\definecolor{textcolor}{rgb}{0.137255,0.215686,0.231373}%
\pgfsetstrokecolor{textcolor}%
\pgfsetfillcolor{textcolor}%
\pgftext[x=0.792771in,y=0.174599in,,base]{\color{textcolor}\sffamily\fontsize{8.000000}{9.600000}\selectfont \texttt{lemon}}%
\end{pgfscope}%
\begin{pgfscope}%
\definecolor{textcolor}{rgb}{0.137255,0.215686,0.231373}%
\pgfsetstrokecolor{textcolor}%
\pgfsetfillcolor{textcolor}%
\pgftext[x=1.427021in,y=0.174599in,,base]{\color{textcolor}\sffamily\fontsize{8.000000}{9.600000}\selectfont \texttt{pear}}%
\end{pgfscope}%
\begin{pgfscope}%
\definecolor{textcolor}{rgb}{0.137255,0.215686,0.231373}%
\pgfsetstrokecolor{textcolor}%
\pgfsetfillcolor{textcolor}%
\pgftext[x=2.061271in,y=0.174599in,,base]{\color{textcolor}\sffamily\fontsize{8.000000}{9.600000}\selectfont \texttt{apple}}%
\end{pgfscope}%
\begin{pgfscope}%
\definecolor{textcolor}{rgb}{0.137255,0.215686,0.231373}%
\pgfsetstrokecolor{textcolor}%
\pgfsetfillcolor{textcolor}%
\pgftext[x=2.695521in,y=0.174599in,,base]{\color{textcolor}\sffamily\fontsize{8.000000}{9.600000}\selectfont \texttt{dog}}%
\end{pgfscope}%
\begin{pgfscope}%
\definecolor{textcolor}{rgb}{0.137255,0.215686,0.231373}%
\pgfsetstrokecolor{textcolor}%
\pgfsetfillcolor{textcolor}%
\pgftext[x=3.329771in,y=0.174599in,,base]{\color{textcolor}\sffamily\fontsize{8.000000}{9.600000}\selectfont \texttt{cat}}%
\end{pgfscope}%
\begin{pgfscope}%
\definecolor{textcolor}{rgb}{0.137255,0.215686,0.231373}%
\pgfsetstrokecolor{textcolor}%
\pgfsetfillcolor{textcolor}%
\pgftext[x=3.964021in,y=0.174599in,,base]{\color{textcolor}\sffamily\fontsize{8.000000}{9.600000}\selectfont \texttt{car}}%
\end{pgfscope}%
\begin{pgfscope}%
\definecolor{textcolor}{rgb}{0.137255,0.215686,0.231373}%
\pgfsetstrokecolor{textcolor}%
\pgfsetfillcolor{textcolor}%
\pgftext[x=0.149436in, y=0.271279in, left, base]{\color{textcolor}\sffamily\fontsize{8.000000}{9.600000}\selectfont \(\displaystyle {0.0}\)}%
\end{pgfscope}%
\begin{pgfscope}%
\definecolor{textcolor}{rgb}{0.137255,0.215686,0.231373}%
\pgfsetstrokecolor{textcolor}%
\pgfsetfillcolor{textcolor}%
\pgftext[x=0.149436in, y=0.568826in, left, base]{\color{textcolor}\sffamily\fontsize{8.000000}{9.600000}\selectfont \(\displaystyle {0.2}\)}%
\end{pgfscope}%
\begin{pgfscope}%
\definecolor{textcolor}{rgb}{0.137255,0.215686,0.231373}%
\pgfsetstrokecolor{textcolor}%
\pgfsetfillcolor{textcolor}%
\pgftext[x=0.149436in, y=0.866374in, left, base]{\color{textcolor}\sffamily\fontsize{8.000000}{9.600000}\selectfont \(\displaystyle {0.4}\)}%
\end{pgfscope}%
\begin{pgfscope}%
\definecolor{textcolor}{rgb}{0.137255,0.215686,0.231373}%
\pgfsetstrokecolor{textcolor}%
\pgfsetfillcolor{textcolor}%
\pgftext[x=0.149436in, y=1.163922in, left, base]{\color{textcolor}\sffamily\fontsize{8.000000}{9.600000}\selectfont \(\displaystyle {0.6}\)}%
\end{pgfscope}%
\begin{pgfscope}%
\definecolor{textcolor}{rgb}{0.137255,0.215686,0.231373}%
\pgfsetstrokecolor{textcolor}%
\pgfsetfillcolor{textcolor}%
\pgftext[x=0.149436in, y=1.461469in, left, base]{\color{textcolor}\sffamily\fontsize{8.000000}{9.600000}\selectfont \(\displaystyle {0.8}\)}%
\end{pgfscope}%
\begin{pgfscope}%
\definecolor{textcolor}{rgb}{0.137255,0.215686,0.231373}%
\pgfsetstrokecolor{textcolor}%
\pgfsetfillcolor{textcolor}%
\pgftext[x=0.149436in, y=1.759017in, left, base]{\color{textcolor}\sffamily\fontsize{8.000000}{9.600000}\selectfont \(\displaystyle {1.0}\)}%
\end{pgfscope}%
\begin{pgfscope}%
\pgfpathrectangle{\pgfqpoint{0.439176in}{0.313488in}}{\pgfqpoint{3.662794in}{1.636512in}}%
\pgfusepath{clip}%
\pgfsetbuttcap%
\pgfsetmiterjoin%
\definecolor{currentfill}{rgb}{0.623529,0.780392,1.000000}%
\pgfsetfillcolor{currentfill}%
\pgfsetlinewidth{1.505625pt}%
\definecolor{currentstroke}{rgb}{0.298039,0.447059,0.690196}%
\pgfsetstrokecolor{currentstroke}%
\pgfsetdash{}{0pt}%
\pgfpathmoveto{\pgfqpoint{2.089812in}{0.313488in}}%
\pgfpathlineto{\pgfqpoint{2.248374in}{0.313488in}}%
\pgfpathlineto{\pgfqpoint{2.248374in}{1.801226in}}%
\pgfpathlineto{\pgfqpoint{2.089812in}{1.801226in}}%
\pgfpathlineto{\pgfqpoint{2.089812in}{0.313488in}}%
\pgfpathclose%
\pgfusepath{stroke,fill}%
\end{pgfscope}%
\begin{pgfscope}%
\pgfpathrectangle{\pgfqpoint{0.439176in}{0.313488in}}{\pgfqpoint{3.662794in}{1.636512in}}%
\pgfusepath{clip}%
\pgfsetbuttcap%
\pgfsetmiterjoin%
\definecolor{currentfill}{rgb}{1.000000,0.756863,0.529412}%
\pgfsetfillcolor{currentfill}%
\pgfsetlinewidth{1.505625pt}%
\definecolor{currentstroke}{rgb}{0.921569,0.505882,0.105882}%
\pgfsetstrokecolor{currentstroke}%
\pgfsetdash{}{0pt}%
\pgfpathmoveto{\pgfqpoint{0.605667in}{0.313488in}}%
\pgfpathlineto{\pgfqpoint{0.764229in}{0.313488in}}%
\pgfpathlineto{\pgfqpoint{0.764229in}{0.462262in}}%
\pgfpathlineto{\pgfqpoint{0.605667in}{0.462262in}}%
\pgfpathlineto{\pgfqpoint{0.605667in}{0.313488in}}%
\pgfpathclose%
\pgfusepath{stroke,fill}%
\end{pgfscope}%
\begin{pgfscope}%
\pgfpathrectangle{\pgfqpoint{0.439176in}{0.313488in}}{\pgfqpoint{3.662794in}{1.636512in}}%
\pgfusepath{clip}%
\pgfsetbuttcap%
\pgfsetmiterjoin%
\definecolor{currentfill}{rgb}{1.000000,0.756863,0.529412}%
\pgfsetfillcolor{currentfill}%
\pgfsetlinewidth{1.505625pt}%
\definecolor{currentstroke}{rgb}{0.921569,0.505882,0.105882}%
\pgfsetstrokecolor{currentstroke}%
\pgfsetdash{}{0pt}%
\pgfpathmoveto{\pgfqpoint{1.239917in}{0.313488in}}%
\pgfpathlineto{\pgfqpoint{1.398479in}{0.313488in}}%
\pgfpathlineto{\pgfqpoint{1.398479in}{0.536649in}}%
\pgfpathlineto{\pgfqpoint{1.239917in}{0.536649in}}%
\pgfpathlineto{\pgfqpoint{1.239917in}{0.313488in}}%
\pgfpathclose%
\pgfusepath{stroke,fill}%
\end{pgfscope}%
\begin{pgfscope}%
\pgfpathrectangle{\pgfqpoint{0.439176in}{0.313488in}}{\pgfqpoint{3.662794in}{1.636512in}}%
\pgfusepath{clip}%
\pgfsetbuttcap%
\pgfsetmiterjoin%
\definecolor{currentfill}{rgb}{1.000000,0.756863,0.529412}%
\pgfsetfillcolor{currentfill}%
\pgfsetlinewidth{1.505625pt}%
\definecolor{currentstroke}{rgb}{0.921569,0.505882,0.105882}%
\pgfsetstrokecolor{currentstroke}%
\pgfsetdash{}{0pt}%
\pgfpathmoveto{\pgfqpoint{1.874167in}{0.313488in}}%
\pgfpathlineto{\pgfqpoint{2.032729in}{0.313488in}}%
\pgfpathlineto{\pgfqpoint{2.032729in}{0.908583in}}%
\pgfpathlineto{\pgfqpoint{1.874167in}{0.908583in}}%
\pgfpathlineto{\pgfqpoint{1.874167in}{0.313488in}}%
\pgfpathclose%
\pgfusepath{stroke,fill}%
\end{pgfscope}%
\begin{pgfscope}%
\pgfpathrectangle{\pgfqpoint{0.439176in}{0.313488in}}{\pgfqpoint{3.662794in}{1.636512in}}%
\pgfusepath{clip}%
\pgfsetbuttcap%
\pgfsetmiterjoin%
\definecolor{currentfill}{rgb}{1.000000,0.756863,0.529412}%
\pgfsetfillcolor{currentfill}%
\pgfsetlinewidth{1.505625pt}%
\definecolor{currentstroke}{rgb}{0.921569,0.505882,0.105882}%
\pgfsetstrokecolor{currentstroke}%
\pgfsetdash{}{0pt}%
\pgfpathmoveto{\pgfqpoint{2.508417in}{0.313488in}}%
\pgfpathlineto{\pgfqpoint{2.666979in}{0.313488in}}%
\pgfpathlineto{\pgfqpoint{2.666979in}{0.536649in}}%
\pgfpathlineto{\pgfqpoint{2.508417in}{0.536649in}}%
\pgfpathlineto{\pgfqpoint{2.508417in}{0.313488in}}%
\pgfpathclose%
\pgfusepath{stroke,fill}%
\end{pgfscope}%
\begin{pgfscope}%
\pgfpathrectangle{\pgfqpoint{0.439176in}{0.313488in}}{\pgfqpoint{3.662794in}{1.636512in}}%
\pgfusepath{clip}%
\pgfsetbuttcap%
\pgfsetmiterjoin%
\definecolor{currentfill}{rgb}{1.000000,0.756863,0.529412}%
\pgfsetfillcolor{currentfill}%
\pgfsetlinewidth{1.505625pt}%
\definecolor{currentstroke}{rgb}{0.921569,0.505882,0.105882}%
\pgfsetstrokecolor{currentstroke}%
\pgfsetdash{}{0pt}%
\pgfpathmoveto{\pgfqpoint{3.142667in}{0.313488in}}%
\pgfpathlineto{\pgfqpoint{3.301229in}{0.313488in}}%
\pgfpathlineto{\pgfqpoint{3.301229in}{0.387875in}}%
\pgfpathlineto{\pgfqpoint{3.142667in}{0.387875in}}%
\pgfpathlineto{\pgfqpoint{3.142667in}{0.313488in}}%
\pgfpathclose%
\pgfusepath{stroke,fill}%
\end{pgfscope}%
\begin{pgfscope}%
\pgfpathrectangle{\pgfqpoint{0.439176in}{0.313488in}}{\pgfqpoint{3.662794in}{1.636512in}}%
\pgfusepath{clip}%
\pgfsetbuttcap%
\pgfsetmiterjoin%
\definecolor{currentfill}{rgb}{1.000000,0.756863,0.529412}%
\pgfsetfillcolor{currentfill}%
\pgfsetlinewidth{1.505625pt}%
\definecolor{currentstroke}{rgb}{0.921569,0.505882,0.105882}%
\pgfsetstrokecolor{currentstroke}%
\pgfsetdash{}{0pt}%
\pgfpathmoveto{\pgfqpoint{3.776917in}{0.313488in}}%
\pgfpathlineto{\pgfqpoint{3.935479in}{0.313488in}}%
\pgfpathlineto{\pgfqpoint{3.935479in}{0.536649in}}%
\pgfpathlineto{\pgfqpoint{3.776917in}{0.536649in}}%
\pgfpathlineto{\pgfqpoint{3.776917in}{0.313488in}}%
\pgfpathclose%
\pgfusepath{stroke,fill}%
\end{pgfscope}%
\begin{pgfscope}%
\pgfsetrectcap%
\pgfsetmiterjoin%
\pgfsetlinewidth{1.003750pt}%
\definecolor{currentstroke}{rgb}{0.917647,0.917647,0.949020}%
\pgfsetstrokecolor{currentstroke}%
\pgfsetdash{}{0pt}%
\pgfpathmoveto{\pgfqpoint{0.439176in}{0.313488in}}%
\pgfpathlineto{\pgfqpoint{0.439176in}{1.950000in}}%
\pgfusepath{stroke}%
\end{pgfscope}%
\begin{pgfscope}%
\pgfsetrectcap%
\pgfsetmiterjoin%
\pgfsetlinewidth{1.003750pt}%
\definecolor{currentstroke}{rgb}{0.917647,0.917647,0.949020}%
\pgfsetstrokecolor{currentstroke}%
\pgfsetdash{}{0pt}%
\pgfpathmoveto{\pgfqpoint{4.101970in}{0.313488in}}%
\pgfpathlineto{\pgfqpoint{4.101970in}{1.950000in}}%
\pgfusepath{stroke}%
\end{pgfscope}%
\begin{pgfscope}%
\pgfsetrectcap%
\pgfsetmiterjoin%
\pgfsetlinewidth{1.003750pt}%
\definecolor{currentstroke}{rgb}{0.917647,0.917647,0.949020}%
\pgfsetstrokecolor{currentstroke}%
\pgfsetdash{}{0pt}%
\pgfpathmoveto{\pgfqpoint{0.439176in}{0.313488in}}%
\pgfpathlineto{\pgfqpoint{4.101970in}{0.313488in}}%
\pgfusepath{stroke}%
\end{pgfscope}%
\begin{pgfscope}%
\pgfsetrectcap%
\pgfsetmiterjoin%
\pgfsetlinewidth{1.003750pt}%
\definecolor{currentstroke}{rgb}{0.917647,0.917647,0.949020}%
\pgfsetstrokecolor{currentstroke}%
\pgfsetdash{}{0pt}%
\pgfpathmoveto{\pgfqpoint{0.439176in}{1.950000in}}%
\pgfpathlineto{\pgfqpoint{4.101970in}{1.950000in}}%
\pgfusepath{stroke}%
\end{pgfscope}%
\begin{pgfscope}%
\pgfsetbuttcap%
\pgfsetmiterjoin%
\definecolor{currentfill}{rgb}{1.000000,0.756863,0.529412}%
\pgfsetfillcolor{currentfill}%
\pgfsetlinewidth{1.505625pt}%
\definecolor{currentstroke}{rgb}{0.921569,0.505882,0.105882}%
\pgfsetstrokecolor{currentstroke}%
\pgfsetdash{}{0pt}%
\pgfpathmoveto{\pgfqpoint{2.932915in}{1.719477in}}%
\pgfpathlineto{\pgfqpoint{3.210693in}{1.719477in}}%
\pgfpathlineto{\pgfqpoint{3.210693in}{1.816699in}}%
\pgfpathlineto{\pgfqpoint{2.932915in}{1.816699in}}%
\pgfpathlineto{\pgfqpoint{2.932915in}{1.719477in}}%
\pgfpathclose%
\pgfusepath{stroke,fill}%
\end{pgfscope}%
\begin{pgfscope}%
\definecolor{textcolor}{rgb}{0.137255,0.215686,0.231373}%
\pgfsetstrokecolor{textcolor}%
\pgfsetfillcolor{textcolor}%
\pgftext[x=3.321804in,y=1.719477in,left,base]{\color{textcolor}\sffamily\fontsize{10.000000}{12.000000}\selectfont \(\displaystyle \psi_\theta\,\left(x\right)\)}%
\end{pgfscope}%
\begin{pgfscope}%
\pgfsetbuttcap%
\pgfsetmiterjoin%
\definecolor{currentfill}{rgb}{0.623529,0.780392,1.000000}%
\pgfsetfillcolor{currentfill}%
\pgfsetlinewidth{1.505625pt}%
\definecolor{currentstroke}{rgb}{0.298039,0.447059,0.690196}%
\pgfsetstrokecolor{currentstroke}%
\pgfsetdash{}{0pt}%
\pgfpathmoveto{\pgfqpoint{2.932915in}{1.509041in}}%
\pgfpathlineto{\pgfqpoint{3.210693in}{1.509041in}}%
\pgfpathlineto{\pgfqpoint{3.210693in}{1.606263in}}%
\pgfpathlineto{\pgfqpoint{2.932915in}{1.606263in}}%
\pgfpathlineto{\pgfqpoint{2.932915in}{1.509041in}}%
\pgfpathclose%
\pgfusepath{stroke,fill}%
\end{pgfscope}%
\begin{pgfscope}%
\definecolor{textcolor}{rgb}{0.137255,0.215686,0.231373}%
\pgfsetstrokecolor{textcolor}%
\pgfsetfillcolor{textcolor}%
\pgftext[x=3.321804in,y=1.509041in,left,base]{\color{textcolor}\sffamily\fontsize{10.000000}{12.000000}\selectfont \(\displaystyle \phi\,\left(\texttt{apple}\right)\)}%
\end{pgfscope}%
\end{pgfpicture}%
\makeatother%
\endgroup%

  }
  \only<4>{
    \vspace{-0.0cm}
    %% Creator: Matplotlib, PGF backend
%%
%% To include the figure in your LaTeX document, write
%%   \input{<filename>.pgf}
%%
%% Make sure the required packages are loaded in your preamble
%%   \usepackage{pgf}
%%
%% Also ensure that all the required font packages are loaded; for instance,
%% the lmodern package is sometimes necessary when using math font.
%%   \usepackage{lmodern}
%%
%% Figures using additional raster images can only be included by \input if
%% they are in the same directory as the main LaTeX file. For loading figures
%% from other directories you can use the `import` package
%%   \usepackage{import}
%%
%% and then include the figures with
%%   \import{<path to file>}{<filename>.pgf}
%%
%% Matplotlib used the following preamble
%%   
%%   \usepackage{fontspec}
%%   \setmainfont{DejaVuSerif.ttf}[Path=\detokenize{/Users/simo/.local/share/virtualenvs/master-thesis-code/lib/python3.10/site-packages/matplotlib/mpl-data/fonts/ttf/}]
%%   \setsansfont{DejaVuSans.ttf}[Path=\detokenize{/Users/simo/.local/share/virtualenvs/master-thesis-code/lib/python3.10/site-packages/matplotlib/mpl-data/fonts/ttf/}]
%%   \setmonofont{DejaVuSansMono.ttf}[Path=\detokenize{/Users/simo/.local/share/virtualenvs/master-thesis-code/lib/python3.10/site-packages/matplotlib/mpl-data/fonts/ttf/}]
%%   \makeatletter\@ifpackageloaded{underscore}{}{\usepackage[strings]{underscore}}\makeatother
%%
\begingroup%
\makeatletter%
\begin{pgfpicture}%
\pgfpathrectangle{\pgfpointorigin}{\pgfqpoint{4.251970in}{2.100000in}}%
\pgfusepath{use as bounding box, clip}%
\begin{pgfscope}%
\pgfsetbuttcap%
\pgfsetmiterjoin%
\definecolor{currentfill}{rgb}{0.980392,0.980392,0.980392}%
\pgfsetfillcolor{currentfill}%
\pgfsetlinewidth{0.000000pt}%
\definecolor{currentstroke}{rgb}{1.000000,1.000000,1.000000}%
\pgfsetstrokecolor{currentstroke}%
\pgfsetdash{}{0pt}%
\pgfpathmoveto{\pgfqpoint{0.000000in}{0.000000in}}%
\pgfpathlineto{\pgfqpoint{4.251970in}{0.000000in}}%
\pgfpathlineto{\pgfqpoint{4.251970in}{2.100000in}}%
\pgfpathlineto{\pgfqpoint{0.000000in}{2.100000in}}%
\pgfpathlineto{\pgfqpoint{0.000000in}{0.000000in}}%
\pgfpathclose%
\pgfusepath{fill}%
\end{pgfscope}%
\begin{pgfscope}%
\pgfsetbuttcap%
\pgfsetmiterjoin%
\definecolor{currentfill}{rgb}{0.917647,0.917647,0.949020}%
\pgfsetfillcolor{currentfill}%
\pgfsetlinewidth{0.000000pt}%
\definecolor{currentstroke}{rgb}{0.000000,0.000000,0.000000}%
\pgfsetstrokecolor{currentstroke}%
\pgfsetstrokeopacity{0.000000}%
\pgfsetdash{}{0pt}%
\pgfpathmoveto{\pgfqpoint{0.439176in}{0.313488in}}%
\pgfpathlineto{\pgfqpoint{4.101970in}{0.313488in}}%
\pgfpathlineto{\pgfqpoint{4.101970in}{1.950000in}}%
\pgfpathlineto{\pgfqpoint{0.439176in}{1.950000in}}%
\pgfpathlineto{\pgfqpoint{0.439176in}{0.313488in}}%
\pgfpathclose%
\pgfusepath{fill}%
\end{pgfscope}%
\begin{pgfscope}%
\definecolor{textcolor}{rgb}{0.137255,0.215686,0.231373}%
\pgfsetstrokecolor{textcolor}%
\pgfsetfillcolor{textcolor}%
\pgftext[x=0.792771in,y=0.174599in,,base]{\color{textcolor}\sffamily\fontsize{8.000000}{9.600000}\selectfont \texttt{lemon}}%
\end{pgfscope}%
\begin{pgfscope}%
\definecolor{textcolor}{rgb}{0.137255,0.215686,0.231373}%
\pgfsetstrokecolor{textcolor}%
\pgfsetfillcolor{textcolor}%
\pgftext[x=1.427021in,y=0.174599in,,base]{\color{textcolor}\sffamily\fontsize{8.000000}{9.600000}\selectfont \texttt{pear}}%
\end{pgfscope}%
\begin{pgfscope}%
\definecolor{textcolor}{rgb}{0.137255,0.215686,0.231373}%
\pgfsetstrokecolor{textcolor}%
\pgfsetfillcolor{textcolor}%
\pgftext[x=2.061271in,y=0.174599in,,base]{\color{textcolor}\sffamily\fontsize{8.000000}{9.600000}\selectfont \texttt{apple}}%
\end{pgfscope}%
\begin{pgfscope}%
\definecolor{textcolor}{rgb}{0.137255,0.215686,0.231373}%
\pgfsetstrokecolor{textcolor}%
\pgfsetfillcolor{textcolor}%
\pgftext[x=2.695521in,y=0.174599in,,base]{\color{textcolor}\sffamily\fontsize{8.000000}{9.600000}\selectfont \texttt{dog}}%
\end{pgfscope}%
\begin{pgfscope}%
\definecolor{textcolor}{rgb}{0.137255,0.215686,0.231373}%
\pgfsetstrokecolor{textcolor}%
\pgfsetfillcolor{textcolor}%
\pgftext[x=3.329771in,y=0.174599in,,base]{\color{textcolor}\sffamily\fontsize{8.000000}{9.600000}\selectfont \texttt{cat}}%
\end{pgfscope}%
\begin{pgfscope}%
\definecolor{textcolor}{rgb}{0.137255,0.215686,0.231373}%
\pgfsetstrokecolor{textcolor}%
\pgfsetfillcolor{textcolor}%
\pgftext[x=3.964021in,y=0.174599in,,base]{\color{textcolor}\sffamily\fontsize{8.000000}{9.600000}\selectfont \texttt{car}}%
\end{pgfscope}%
\begin{pgfscope}%
\definecolor{textcolor}{rgb}{0.137255,0.215686,0.231373}%
\pgfsetstrokecolor{textcolor}%
\pgfsetfillcolor{textcolor}%
\pgftext[x=0.149436in, y=0.271279in, left, base]{\color{textcolor}\sffamily\fontsize{8.000000}{9.600000}\selectfont \(\displaystyle {0.0}\)}%
\end{pgfscope}%
\begin{pgfscope}%
\definecolor{textcolor}{rgb}{0.137255,0.215686,0.231373}%
\pgfsetstrokecolor{textcolor}%
\pgfsetfillcolor{textcolor}%
\pgftext[x=0.149436in, y=0.568826in, left, base]{\color{textcolor}\sffamily\fontsize{8.000000}{9.600000}\selectfont \(\displaystyle {0.2}\)}%
\end{pgfscope}%
\begin{pgfscope}%
\definecolor{textcolor}{rgb}{0.137255,0.215686,0.231373}%
\pgfsetstrokecolor{textcolor}%
\pgfsetfillcolor{textcolor}%
\pgftext[x=0.149436in, y=0.866374in, left, base]{\color{textcolor}\sffamily\fontsize{8.000000}{9.600000}\selectfont \(\displaystyle {0.4}\)}%
\end{pgfscope}%
\begin{pgfscope}%
\definecolor{textcolor}{rgb}{0.137255,0.215686,0.231373}%
\pgfsetstrokecolor{textcolor}%
\pgfsetfillcolor{textcolor}%
\pgftext[x=0.149436in, y=1.163922in, left, base]{\color{textcolor}\sffamily\fontsize{8.000000}{9.600000}\selectfont \(\displaystyle {0.6}\)}%
\end{pgfscope}%
\begin{pgfscope}%
\definecolor{textcolor}{rgb}{0.137255,0.215686,0.231373}%
\pgfsetstrokecolor{textcolor}%
\pgfsetfillcolor{textcolor}%
\pgftext[x=0.149436in, y=1.461469in, left, base]{\color{textcolor}\sffamily\fontsize{8.000000}{9.600000}\selectfont \(\displaystyle {0.8}\)}%
\end{pgfscope}%
\begin{pgfscope}%
\definecolor{textcolor}{rgb}{0.137255,0.215686,0.231373}%
\pgfsetstrokecolor{textcolor}%
\pgfsetfillcolor{textcolor}%
\pgftext[x=0.149436in, y=1.759017in, left, base]{\color{textcolor}\sffamily\fontsize{8.000000}{9.600000}\selectfont \(\displaystyle {1.0}\)}%
\end{pgfscope}%
\begin{pgfscope}%
\pgfpathrectangle{\pgfqpoint{0.439176in}{0.313488in}}{\pgfqpoint{3.662794in}{1.636512in}}%
\pgfusepath{clip}%
\pgfsetbuttcap%
\pgfsetmiterjoin%
\definecolor{currentfill}{rgb}{0.623529,0.780392,1.000000}%
\pgfsetfillcolor{currentfill}%
\pgfsetlinewidth{1.505625pt}%
\definecolor{currentstroke}{rgb}{0.298039,0.447059,0.690196}%
\pgfsetstrokecolor{currentstroke}%
\pgfsetdash{}{0pt}%
\pgfpathmoveto{\pgfqpoint{2.089812in}{0.313488in}}%
\pgfpathlineto{\pgfqpoint{2.248374in}{0.313488in}}%
\pgfpathlineto{\pgfqpoint{2.248374in}{1.801226in}}%
\pgfpathlineto{\pgfqpoint{2.089812in}{1.801226in}}%
\pgfpathlineto{\pgfqpoint{2.089812in}{0.313488in}}%
\pgfpathclose%
\pgfusepath{stroke,fill}%
\end{pgfscope}%
\begin{pgfscope}%
\pgfpathrectangle{\pgfqpoint{0.439176in}{0.313488in}}{\pgfqpoint{3.662794in}{1.636512in}}%
\pgfusepath{clip}%
\pgfsetbuttcap%
\pgfsetmiterjoin%
\definecolor{currentfill}{rgb}{1.000000,0.756863,0.529412}%
\pgfsetfillcolor{currentfill}%
\pgfsetlinewidth{1.505625pt}%
\definecolor{currentstroke}{rgb}{0.921569,0.505882,0.105882}%
\pgfsetstrokecolor{currentstroke}%
\pgfsetdash{}{0pt}%
\pgfpathmoveto{\pgfqpoint{0.605667in}{0.313488in}}%
\pgfpathlineto{\pgfqpoint{0.764229in}{0.313488in}}%
\pgfpathlineto{\pgfqpoint{0.764229in}{0.328365in}}%
\pgfpathlineto{\pgfqpoint{0.605667in}{0.328365in}}%
\pgfpathlineto{\pgfqpoint{0.605667in}{0.313488in}}%
\pgfpathclose%
\pgfusepath{stroke,fill}%
\end{pgfscope}%
\begin{pgfscope}%
\pgfpathrectangle{\pgfqpoint{0.439176in}{0.313488in}}{\pgfqpoint{3.662794in}{1.636512in}}%
\pgfusepath{clip}%
\pgfsetbuttcap%
\pgfsetmiterjoin%
\definecolor{currentfill}{rgb}{1.000000,0.756863,0.529412}%
\pgfsetfillcolor{currentfill}%
\pgfsetlinewidth{1.505625pt}%
\definecolor{currentstroke}{rgb}{0.921569,0.505882,0.105882}%
\pgfsetstrokecolor{currentstroke}%
\pgfsetdash{}{0pt}%
\pgfpathmoveto{\pgfqpoint{1.239917in}{0.313488in}}%
\pgfpathlineto{\pgfqpoint{1.398479in}{0.313488in}}%
\pgfpathlineto{\pgfqpoint{1.398479in}{0.343243in}}%
\pgfpathlineto{\pgfqpoint{1.239917in}{0.343243in}}%
\pgfpathlineto{\pgfqpoint{1.239917in}{0.313488in}}%
\pgfpathclose%
\pgfusepath{stroke,fill}%
\end{pgfscope}%
\begin{pgfscope}%
\pgfpathrectangle{\pgfqpoint{0.439176in}{0.313488in}}{\pgfqpoint{3.662794in}{1.636512in}}%
\pgfusepath{clip}%
\pgfsetbuttcap%
\pgfsetmiterjoin%
\definecolor{currentfill}{rgb}{1.000000,0.756863,0.529412}%
\pgfsetfillcolor{currentfill}%
\pgfsetlinewidth{1.505625pt}%
\definecolor{currentstroke}{rgb}{0.921569,0.505882,0.105882}%
\pgfsetstrokecolor{currentstroke}%
\pgfsetdash{}{0pt}%
\pgfpathmoveto{\pgfqpoint{1.874167in}{0.313488in}}%
\pgfpathlineto{\pgfqpoint{2.032729in}{0.313488in}}%
\pgfpathlineto{\pgfqpoint{2.032729in}{1.578065in}}%
\pgfpathlineto{\pgfqpoint{1.874167in}{1.578065in}}%
\pgfpathlineto{\pgfqpoint{1.874167in}{0.313488in}}%
\pgfpathclose%
\pgfusepath{stroke,fill}%
\end{pgfscope}%
\begin{pgfscope}%
\pgfpathrectangle{\pgfqpoint{0.439176in}{0.313488in}}{\pgfqpoint{3.662794in}{1.636512in}}%
\pgfusepath{clip}%
\pgfsetbuttcap%
\pgfsetmiterjoin%
\definecolor{currentfill}{rgb}{1.000000,0.756863,0.529412}%
\pgfsetfillcolor{currentfill}%
\pgfsetlinewidth{1.505625pt}%
\definecolor{currentstroke}{rgb}{0.921569,0.505882,0.105882}%
\pgfsetstrokecolor{currentstroke}%
\pgfsetdash{}{0pt}%
\pgfpathmoveto{\pgfqpoint{2.508417in}{0.313488in}}%
\pgfpathlineto{\pgfqpoint{2.666979in}{0.313488in}}%
\pgfpathlineto{\pgfqpoint{2.666979in}{0.402752in}}%
\pgfpathlineto{\pgfqpoint{2.508417in}{0.402752in}}%
\pgfpathlineto{\pgfqpoint{2.508417in}{0.313488in}}%
\pgfpathclose%
\pgfusepath{stroke,fill}%
\end{pgfscope}%
\begin{pgfscope}%
\pgfpathrectangle{\pgfqpoint{0.439176in}{0.313488in}}{\pgfqpoint{3.662794in}{1.636512in}}%
\pgfusepath{clip}%
\pgfsetbuttcap%
\pgfsetmiterjoin%
\definecolor{currentfill}{rgb}{1.000000,0.756863,0.529412}%
\pgfsetfillcolor{currentfill}%
\pgfsetlinewidth{1.505625pt}%
\definecolor{currentstroke}{rgb}{0.921569,0.505882,0.105882}%
\pgfsetstrokecolor{currentstroke}%
\pgfsetdash{}{0pt}%
\pgfpathmoveto{\pgfqpoint{3.142667in}{0.313488in}}%
\pgfpathlineto{\pgfqpoint{3.301229in}{0.313488in}}%
\pgfpathlineto{\pgfqpoint{3.301229in}{0.328365in}}%
\pgfpathlineto{\pgfqpoint{3.142667in}{0.328365in}}%
\pgfpathlineto{\pgfqpoint{3.142667in}{0.313488in}}%
\pgfpathclose%
\pgfusepath{stroke,fill}%
\end{pgfscope}%
\begin{pgfscope}%
\pgfpathrectangle{\pgfqpoint{0.439176in}{0.313488in}}{\pgfqpoint{3.662794in}{1.636512in}}%
\pgfusepath{clip}%
\pgfsetbuttcap%
\pgfsetmiterjoin%
\definecolor{currentfill}{rgb}{1.000000,0.756863,0.529412}%
\pgfsetfillcolor{currentfill}%
\pgfsetlinewidth{1.505625pt}%
\definecolor{currentstroke}{rgb}{0.921569,0.505882,0.105882}%
\pgfsetstrokecolor{currentstroke}%
\pgfsetdash{}{0pt}%
\pgfpathmoveto{\pgfqpoint{3.776917in}{0.313488in}}%
\pgfpathlineto{\pgfqpoint{3.935479in}{0.313488in}}%
\pgfpathlineto{\pgfqpoint{3.935479in}{0.387875in}}%
\pgfpathlineto{\pgfqpoint{3.776917in}{0.387875in}}%
\pgfpathlineto{\pgfqpoint{3.776917in}{0.313488in}}%
\pgfpathclose%
\pgfusepath{stroke,fill}%
\end{pgfscope}%
\begin{pgfscope}%
\pgfsetrectcap%
\pgfsetmiterjoin%
\pgfsetlinewidth{1.003750pt}%
\definecolor{currentstroke}{rgb}{0.917647,0.917647,0.949020}%
\pgfsetstrokecolor{currentstroke}%
\pgfsetdash{}{0pt}%
\pgfpathmoveto{\pgfqpoint{0.439176in}{0.313488in}}%
\pgfpathlineto{\pgfqpoint{0.439176in}{1.950000in}}%
\pgfusepath{stroke}%
\end{pgfscope}%
\begin{pgfscope}%
\pgfsetrectcap%
\pgfsetmiterjoin%
\pgfsetlinewidth{1.003750pt}%
\definecolor{currentstroke}{rgb}{0.917647,0.917647,0.949020}%
\pgfsetstrokecolor{currentstroke}%
\pgfsetdash{}{0pt}%
\pgfpathmoveto{\pgfqpoint{4.101970in}{0.313488in}}%
\pgfpathlineto{\pgfqpoint{4.101970in}{1.950000in}}%
\pgfusepath{stroke}%
\end{pgfscope}%
\begin{pgfscope}%
\pgfsetrectcap%
\pgfsetmiterjoin%
\pgfsetlinewidth{1.003750pt}%
\definecolor{currentstroke}{rgb}{0.917647,0.917647,0.949020}%
\pgfsetstrokecolor{currentstroke}%
\pgfsetdash{}{0pt}%
\pgfpathmoveto{\pgfqpoint{0.439176in}{0.313488in}}%
\pgfpathlineto{\pgfqpoint{4.101970in}{0.313488in}}%
\pgfusepath{stroke}%
\end{pgfscope}%
\begin{pgfscope}%
\pgfsetrectcap%
\pgfsetmiterjoin%
\pgfsetlinewidth{1.003750pt}%
\definecolor{currentstroke}{rgb}{0.917647,0.917647,0.949020}%
\pgfsetstrokecolor{currentstroke}%
\pgfsetdash{}{0pt}%
\pgfpathmoveto{\pgfqpoint{0.439176in}{1.950000in}}%
\pgfpathlineto{\pgfqpoint{4.101970in}{1.950000in}}%
\pgfusepath{stroke}%
\end{pgfscope}%
\begin{pgfscope}%
\pgfsetbuttcap%
\pgfsetmiterjoin%
\definecolor{currentfill}{rgb}{1.000000,0.756863,0.529412}%
\pgfsetfillcolor{currentfill}%
\pgfsetlinewidth{1.505625pt}%
\definecolor{currentstroke}{rgb}{0.921569,0.505882,0.105882}%
\pgfsetstrokecolor{currentstroke}%
\pgfsetdash{}{0pt}%
\pgfpathmoveto{\pgfqpoint{2.932915in}{1.719477in}}%
\pgfpathlineto{\pgfqpoint{3.210693in}{1.719477in}}%
\pgfpathlineto{\pgfqpoint{3.210693in}{1.816699in}}%
\pgfpathlineto{\pgfqpoint{2.932915in}{1.816699in}}%
\pgfpathlineto{\pgfqpoint{2.932915in}{1.719477in}}%
\pgfpathclose%
\pgfusepath{stroke,fill}%
\end{pgfscope}%
\begin{pgfscope}%
\definecolor{textcolor}{rgb}{0.137255,0.215686,0.231373}%
\pgfsetstrokecolor{textcolor}%
\pgfsetfillcolor{textcolor}%
\pgftext[x=3.321804in,y=1.719477in,left,base]{\color{textcolor}\sffamily\fontsize{10.000000}{12.000000}\selectfont \(\displaystyle \psi_\theta\,\left(x\right)\)}%
\end{pgfscope}%
\begin{pgfscope}%
\pgfsetbuttcap%
\pgfsetmiterjoin%
\definecolor{currentfill}{rgb}{0.623529,0.780392,1.000000}%
\pgfsetfillcolor{currentfill}%
\pgfsetlinewidth{1.505625pt}%
\definecolor{currentstroke}{rgb}{0.298039,0.447059,0.690196}%
\pgfsetstrokecolor{currentstroke}%
\pgfsetdash{}{0pt}%
\pgfpathmoveto{\pgfqpoint{2.932915in}{1.509041in}}%
\pgfpathlineto{\pgfqpoint{3.210693in}{1.509041in}}%
\pgfpathlineto{\pgfqpoint{3.210693in}{1.606263in}}%
\pgfpathlineto{\pgfqpoint{2.932915in}{1.606263in}}%
\pgfpathlineto{\pgfqpoint{2.932915in}{1.509041in}}%
\pgfpathclose%
\pgfusepath{stroke,fill}%
\end{pgfscope}%
\begin{pgfscope}%
\definecolor{textcolor}{rgb}{0.137255,0.215686,0.231373}%
\pgfsetstrokecolor{textcolor}%
\pgfsetfillcolor{textcolor}%
\pgftext[x=3.321804in,y=1.509041in,left,base]{\color{textcolor}\sffamily\fontsize{10.000000}{12.000000}\selectfont \(\displaystyle \phi\,\left(\texttt{apple}\right)\)}%
\end{pgfscope}%
\end{pgfpicture}%
\makeatother%
\endgroup%

  }

  \note[item]{Una funzione di Loss largamente impiegata nel training di modelli
  è la Cross Entropy. Definita per $p$ e $q$ distribuzioni di probabilità, è
  meno di prodotto scalare tra $q$ e il logaritmo di $p$. Nel nostro caso $p$
  è l'output del modello e $q$ l'encoding della classe.}
  \note[item]{Se $phi$ è One-hot encoding solo una componente di $p$ da
  contributo, quella associata alla classe corretta}
  \note[item]{NEXT}
  \note[item]{Minimizzare $\mathcal{L}$ corrisponde ad alzare la probabilità
  relativa alla classe corretta e causa della normalizzazione di $\psi$ le
  probabilità associate alle altre classi saranno necessariamente ridotte.}
  \note[item]{NEXT, NEXT}
\end{frame}
