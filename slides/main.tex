\documentclass[10pt]{beamer}

% Packages ------------------------------------------------------------------- %

% Set the language of the document (e.g. title, section, abstract, …)
\usepackage[english]{babel}

% Theme
\usetheme{metropolis}

% Finer control on caption formatting
\usepackage{caption}

% Handle appendix numbering
\usepackage{appendixnumberbeamer}

% Math related packages
\usepackage{amsfonts}
\usepackage{amsmath}
\usepackage{amssymb}
\usepackage{bm}
\usepackage{mathtools}

% Checkmark and X mark
\usepackage{pifont}

% Inspect layout dimensions
\usepackage{layouts}

% Include graphics
\usepackage{graphicx}

% Include diagrams in tikz
\usepackage{tikzit}

% import tikzstyle
\input{metropolis.tikzstyles}


% Custom commands ------------------------------------------------------------- %

% use := for "is defined to be equal to"
% http://projekte.dante.de/DanteFAQ/Symbole#20
\mathchardef\ordinarycolon\mathcode`\:
\mathcode`\:=\string"8000
\begingroup \catcode`\:=\active
  \gdef:{\mathrel{\mathop\ordinarycolon}}
\endgroup

% Checkmark and X mark
\newcommand{\cmark}{\ding{51}}%
\newcommand{\xmark}{\ding{55}}%


% Build Options -------------------------------------------------------------- %

\setbeameroption{hide notes} % Only slides
% \setbeameroption{show only notes} % Only notes
% \setbeameroption{show notes on second screen=right}


% Title ---------------------------------------------------------------------- %

\title{Semantic Transfer Learning (½ Thesis)}

%TODO: add prof or Dr
\author{
  \textit{Author:} \ \textbf{Simone~\textsc{Bertolotto}} \\
  \textit{Supervisors:} \
    \textbf{André~\textsc{Panisson}}\inst{1 2} \&
    \textbf{Alan~\textsc{Perotti}}\inst{2}
    \vspace{1em}
}

\date{\today}

\institute{
  \inst{1}\textsc{Unito --- Dipartimento di Fisica dei Sistemi Complessi} \\
  \inst{2}\textsc{CENTAI Institute}
}


% Slides --------------------------------------------------------------------- %

\begin{document}

  \begin{frame}
    \titlepage
    \note{Buongiorno, sono Simone Bertolotto e oggi presento la prima parte
    della tesi di laurea svolta presso CENTAI institute, un istituto di ricerca
    torinese. I relatori sono il professor André Panisson e il
    dottor Alan Perotti.}
  \end{frame}

  \section{Introduction}
  \begin{frame}{Introduction | Framework}
  \begin{center}
    \huge{Models for images classification\\}
    \huge{ignore\\}
    \huge{relationships between classes}
  \end{center}
  \note[item]{I modelli per la classificatione di immagini generalmente ignorano
  le relazioni tra le classi.}
\end{frame}

\begin{frame}{Introduction | Framework}
  \tikzfig{figures/framework}
  \vspace{1.5em}
  \pause
  \begin{center}
    \large{We fix $\psi$ and play with $\phi$ and $\mathcal{L}$}
  \end{center}

  \note[item]{\footnotesize{Un modello per la classificatione di immagini è un funzione
  $\psi_\theta$ che prende come input un'immagine e restituisce un vettore di
  output.}}
  \note[item]{La classe associata all'immagine è convertita da testo a numeri
  dalla funzione $\phi$ producendo il vettore di encoding. Spesso tale funzione
  è triviale, una semplice lookup table, tanto da non venir esplicitamente
  rappresentata nei diagrammi tipo quello mostrato.}
  \note[item]{La funzione di Loss $\mathcal{L}$ quantifica con un numero la
  dissimilarità tra output del modello e corrispondente encoding, minimizzarla
  significa avviciare l'output del modello alla codifica della classe corretta
  e quindi migliorare la classificazione del modello. L'argoritmo di
  backpropagation minimizza in modo iterativo $\mathcal{L}$ agendo sui i
  parametri del modello $\theta$.}
  \note[item]{Generalmente si sperimenta modificando l'architettura del modello
  ovvero cambiando la forma funzionale di $\psi$.}
  \note[item]{NEXT}
  \note[item]{Noi invece abbiamo fissato $\psi$ e variato $\phi$ e
  $\mathcal{L}$.}
\end{frame}

\begin{frame}{Introduction | Framework}
  \begin{center}
    \huge{Models for images classification\\}
    \huge{\alert{exploit}\\}
    \huge{relationships between classes}
  \end{center}
  \note[item]{con l'obiettivo di construire modelli che sfruttino le relazioni
  tra le classi.}
\end{frame}


\begin{frame}{Introduction | One-hot Enconding}
  \begin{align*}
    \phi &:
    \mathcal{C} \rightarrow {\left\{0, 1\right\}}^{|\mathcal{C}|} :
    c_i \mapsto \phi \, (c_i)
    \quad &\text{where}& \quad
    {\phi \, (c_i)}_j := \delta_{i,j} \\
    \psi_\theta &:
    \mathcal{X} \rightarrow {\left[0, 1\right]}^{|\mathcal{C}|} :
    x \mapsto \psi_\theta \, (x)
    \quad &\text{where}& \quad 
    \sum_{j=1}^{|\mathcal{C}|} {\psi_\theta \, (x)}_j = 1
  \end{align*}

  \metroset{block=fill}
  \begin{block}{Example}
    \begin{align*}
      \mathcal{C} &= [
        &\texttt{lemon} &,  &\texttt{pear} &,  &\texttt{apple} &,
        &\texttt{dog} &, &\texttt{cat} &, &\texttt{car} &\
      ] \\
      \phi\,\left(\texttt{apple}\right) &= [
        & 0 &, & 0 &, & 1 &,
        & 0 &, & 0 &, & 0 &\
      ] \\
      \psi_\theta \,\left(x\right) &= [
        & 0.2 &, & 0.3 &, & 0.1 &,
        & 0.15 &, & 0.05 &, & 0.2 &\
      ]
    \end{align*}

    \note{\footnotesize{La più semplice e diffusa funzione di encoding è One-hot
    encoding, una funzione $\phi$ che mappa dalla lista delle classi a un
    vettore binario le cui componenti sono nulle ad eccezione di quella il cui
    indice coincide con l'indice della classe mappata.}}
    \note[item]{Introducendo un esempio che ci accompagnerà nel proseguio, se
    le classi del dataset sono "limone", "pera", "mela", "cane", "gatto" e
    "auto", $\phi(\texttt{apple})$, l'encoding della classe "mela", sarà un
    vettore di zeri con un "uno" nella terza componente.}
    \note[item]{Il modello mappa invece dallo spazio delle immagini a una
    "probability mass function" sull'iniseme delle classi.}
    \note[item]{L'indice della componente maggiore del output è ciò che
    consideriamo la classe predetta dal modello.}
    \note[item]{Inizialmente non c'è correlazione tra classe predetta e classe
    reale, in quanto in parametri sono inizializzati in modo casuale. Solo in
    seguito ad un processo di training è possibile produrre una corretta
    classificazione.}
  \end{block}


\end{frame}

\begin{frame}{Introduction | Cross Entropy}

  \begin{equation*}
    \mathcal{L} \, \left(p, q\right) := - q \cdot \log p
    \quad \Longrightarrow \quad
    \mathcal{L} \, \left(\psi_\theta(x), \phi(c_i)\right) =
    - \log \left({\psi_\theta(x)}_i\right)
  \end{equation*}

  \pause
  \only<2>{
    \vspace{-0.0cm}
    \input{figures/one-hot_cross-entropy_1.pgf}
  }
  \only<3>{
    \vspace{-0.0cm}
    %% Creator: Matplotlib, PGF backend
%%
%% To include the figure in your LaTeX document, write
%%   \input{<filename>.pgf}
%%
%% Make sure the required packages are loaded in your preamble
%%   \usepackage{pgf}
%%
%% Also ensure that all the required font packages are loaded; for instance,
%% the lmodern package is sometimes necessary when using math font.
%%   \usepackage{lmodern}
%%
%% Figures using additional raster images can only be included by \input if
%% they are in the same directory as the main LaTeX file. For loading figures
%% from other directories you can use the `import` package
%%   \usepackage{import}
%%
%% and then include the figures with
%%   \import{<path to file>}{<filename>.pgf}
%%
%% Matplotlib used the following preamble
%%   
%%   \usepackage{fontspec}
%%   \setmainfont{DejaVuSerif.ttf}[Path=\detokenize{/Users/simo/.local/share/virtualenvs/master-thesis-code/lib/python3.10/site-packages/matplotlib/mpl-data/fonts/ttf/}]
%%   \setsansfont{DejaVuSans.ttf}[Path=\detokenize{/Users/simo/.local/share/virtualenvs/master-thesis-code/lib/python3.10/site-packages/matplotlib/mpl-data/fonts/ttf/}]
%%   \setmonofont{DejaVuSansMono.ttf}[Path=\detokenize{/Users/simo/.local/share/virtualenvs/master-thesis-code/lib/python3.10/site-packages/matplotlib/mpl-data/fonts/ttf/}]
%%   \makeatletter\@ifpackageloaded{underscore}{}{\usepackage[strings]{underscore}}\makeatother
%%
\begingroup%
\makeatletter%
\begin{pgfpicture}%
\pgfpathrectangle{\pgfpointorigin}{\pgfqpoint{4.251970in}{2.100000in}}%
\pgfusepath{use as bounding box, clip}%
\begin{pgfscope}%
\pgfsetbuttcap%
\pgfsetmiterjoin%
\definecolor{currentfill}{rgb}{0.980392,0.980392,0.980392}%
\pgfsetfillcolor{currentfill}%
\pgfsetlinewidth{0.000000pt}%
\definecolor{currentstroke}{rgb}{1.000000,1.000000,1.000000}%
\pgfsetstrokecolor{currentstroke}%
\pgfsetdash{}{0pt}%
\pgfpathmoveto{\pgfqpoint{0.000000in}{0.000000in}}%
\pgfpathlineto{\pgfqpoint{4.251970in}{0.000000in}}%
\pgfpathlineto{\pgfqpoint{4.251970in}{2.100000in}}%
\pgfpathlineto{\pgfqpoint{0.000000in}{2.100000in}}%
\pgfpathlineto{\pgfqpoint{0.000000in}{0.000000in}}%
\pgfpathclose%
\pgfusepath{fill}%
\end{pgfscope}%
\begin{pgfscope}%
\pgfsetbuttcap%
\pgfsetmiterjoin%
\definecolor{currentfill}{rgb}{0.917647,0.917647,0.949020}%
\pgfsetfillcolor{currentfill}%
\pgfsetlinewidth{0.000000pt}%
\definecolor{currentstroke}{rgb}{0.000000,0.000000,0.000000}%
\pgfsetstrokecolor{currentstroke}%
\pgfsetstrokeopacity{0.000000}%
\pgfsetdash{}{0pt}%
\pgfpathmoveto{\pgfqpoint{0.439176in}{0.313488in}}%
\pgfpathlineto{\pgfqpoint{4.101970in}{0.313488in}}%
\pgfpathlineto{\pgfqpoint{4.101970in}{1.950000in}}%
\pgfpathlineto{\pgfqpoint{0.439176in}{1.950000in}}%
\pgfpathlineto{\pgfqpoint{0.439176in}{0.313488in}}%
\pgfpathclose%
\pgfusepath{fill}%
\end{pgfscope}%
\begin{pgfscope}%
\definecolor{textcolor}{rgb}{0.137255,0.215686,0.231373}%
\pgfsetstrokecolor{textcolor}%
\pgfsetfillcolor{textcolor}%
\pgftext[x=0.835309in,y=0.174599in,,base]{\color{textcolor}\sffamily\fontsize{8.000000}{9.600000}\selectfont \texttt{lemon}}%
\end{pgfscope}%
\begin{pgfscope}%
\definecolor{textcolor}{rgb}{0.137255,0.215686,0.231373}%
\pgfsetstrokecolor{textcolor}%
\pgfsetfillcolor{textcolor}%
\pgftext[x=1.409415in,y=0.174599in,,base]{\color{textcolor}\sffamily\fontsize{8.000000}{9.600000}\selectfont \texttt{pear}}%
\end{pgfscope}%
\begin{pgfscope}%
\definecolor{textcolor}{rgb}{0.137255,0.215686,0.231373}%
\pgfsetstrokecolor{textcolor}%
\pgfsetfillcolor{textcolor}%
\pgftext[x=1.983520in,y=0.174599in,,base]{\color{textcolor}\sffamily\fontsize{8.000000}{9.600000}\selectfont \texttt{apple}}%
\end{pgfscope}%
\begin{pgfscope}%
\definecolor{textcolor}{rgb}{0.137255,0.215686,0.231373}%
\pgfsetstrokecolor{textcolor}%
\pgfsetfillcolor{textcolor}%
\pgftext[x=2.557626in,y=0.174599in,,base]{\color{textcolor}\sffamily\fontsize{8.000000}{9.600000}\selectfont \texttt{dog}}%
\end{pgfscope}%
\begin{pgfscope}%
\definecolor{textcolor}{rgb}{0.137255,0.215686,0.231373}%
\pgfsetstrokecolor{textcolor}%
\pgfsetfillcolor{textcolor}%
\pgftext[x=3.131732in,y=0.174599in,,base]{\color{textcolor}\sffamily\fontsize{8.000000}{9.600000}\selectfont \texttt{cat}}%
\end{pgfscope}%
\begin{pgfscope}%
\definecolor{textcolor}{rgb}{0.137255,0.215686,0.231373}%
\pgfsetstrokecolor{textcolor}%
\pgfsetfillcolor{textcolor}%
\pgftext[x=3.705837in,y=0.174599in,,base]{\color{textcolor}\sffamily\fontsize{8.000000}{9.600000}\selectfont \texttt{car}}%
\end{pgfscope}%
\begin{pgfscope}%
\definecolor{textcolor}{rgb}{0.137255,0.215686,0.231373}%
\pgfsetstrokecolor{textcolor}%
\pgfsetfillcolor{textcolor}%
\pgftext[x=0.149436in, y=0.271279in, left, base]{\color{textcolor}\sffamily\fontsize{8.000000}{9.600000}\selectfont \(\displaystyle {0.0}\)}%
\end{pgfscope}%
\begin{pgfscope}%
\definecolor{textcolor}{rgb}{0.137255,0.215686,0.231373}%
\pgfsetstrokecolor{textcolor}%
\pgfsetfillcolor{textcolor}%
\pgftext[x=0.149436in, y=0.568826in, left, base]{\color{textcolor}\sffamily\fontsize{8.000000}{9.600000}\selectfont \(\displaystyle {0.2}\)}%
\end{pgfscope}%
\begin{pgfscope}%
\definecolor{textcolor}{rgb}{0.137255,0.215686,0.231373}%
\pgfsetstrokecolor{textcolor}%
\pgfsetfillcolor{textcolor}%
\pgftext[x=0.149436in, y=0.866374in, left, base]{\color{textcolor}\sffamily\fontsize{8.000000}{9.600000}\selectfont \(\displaystyle {0.4}\)}%
\end{pgfscope}%
\begin{pgfscope}%
\definecolor{textcolor}{rgb}{0.137255,0.215686,0.231373}%
\pgfsetstrokecolor{textcolor}%
\pgfsetfillcolor{textcolor}%
\pgftext[x=0.149436in, y=1.163922in, left, base]{\color{textcolor}\sffamily\fontsize{8.000000}{9.600000}\selectfont \(\displaystyle {0.6}\)}%
\end{pgfscope}%
\begin{pgfscope}%
\definecolor{textcolor}{rgb}{0.137255,0.215686,0.231373}%
\pgfsetstrokecolor{textcolor}%
\pgfsetfillcolor{textcolor}%
\pgftext[x=0.149436in, y=1.461469in, left, base]{\color{textcolor}\sffamily\fontsize{8.000000}{9.600000}\selectfont \(\displaystyle {0.8}\)}%
\end{pgfscope}%
\begin{pgfscope}%
\definecolor{textcolor}{rgb}{0.137255,0.215686,0.231373}%
\pgfsetstrokecolor{textcolor}%
\pgfsetfillcolor{textcolor}%
\pgftext[x=0.149436in, y=1.759017in, left, base]{\color{textcolor}\sffamily\fontsize{8.000000}{9.600000}\selectfont \(\displaystyle {1.0}\)}%
\end{pgfscope}%
\begin{pgfscope}%
\pgfpathrectangle{\pgfqpoint{0.439176in}{0.313488in}}{\pgfqpoint{3.662794in}{1.636512in}}%
\pgfusepath{clip}%
\pgfsetbuttcap%
\pgfsetmiterjoin%
\definecolor{currentfill}{rgb}{0.623529,0.780392,1.000000}%
\pgfsetfillcolor{currentfill}%
\pgfsetlinewidth{1.505625pt}%
\definecolor{currentstroke}{rgb}{0.298039,0.447059,0.690196}%
\pgfsetstrokecolor{currentstroke}%
\pgfsetdash{}{0pt}%
\pgfpathmoveto{\pgfqpoint{1.753878in}{0.313488in}}%
\pgfpathlineto{\pgfqpoint{2.213163in}{0.313488in}}%
\pgfpathlineto{\pgfqpoint{2.213163in}{1.801226in}}%
\pgfpathlineto{\pgfqpoint{1.753878in}{1.801226in}}%
\pgfpathlineto{\pgfqpoint{1.753878in}{0.313488in}}%
\pgfpathclose%
\pgfusepath{stroke,fill}%
\end{pgfscope}%
\begin{pgfscope}%
\pgfpathrectangle{\pgfqpoint{0.439176in}{0.313488in}}{\pgfqpoint{3.662794in}{1.636512in}}%
\pgfusepath{clip}%
\pgfsetbuttcap%
\pgfsetmiterjoin%
\definecolor{currentfill}{rgb}{1.000000,0.756863,0.529412}%
\pgfsetfillcolor{currentfill}%
\pgfsetlinewidth{1.505625pt}%
\definecolor{currentstroke}{rgb}{0.921569,0.505882,0.105882}%
\pgfsetstrokecolor{currentstroke}%
\pgfsetdash{}{0pt}%
\pgfpathmoveto{\pgfqpoint{0.605667in}{0.313488in}}%
\pgfpathlineto{\pgfqpoint{1.064951in}{0.313488in}}%
\pgfpathlineto{\pgfqpoint{1.064951in}{0.462262in}}%
\pgfpathlineto{\pgfqpoint{0.605667in}{0.462262in}}%
\pgfpathlineto{\pgfqpoint{0.605667in}{0.313488in}}%
\pgfpathclose%
\pgfusepath{stroke,fill}%
\end{pgfscope}%
\begin{pgfscope}%
\pgfpathrectangle{\pgfqpoint{0.439176in}{0.313488in}}{\pgfqpoint{3.662794in}{1.636512in}}%
\pgfusepath{clip}%
\pgfsetbuttcap%
\pgfsetmiterjoin%
\definecolor{currentfill}{rgb}{1.000000,0.756863,0.529412}%
\pgfsetfillcolor{currentfill}%
\pgfsetlinewidth{1.505625pt}%
\definecolor{currentstroke}{rgb}{0.921569,0.505882,0.105882}%
\pgfsetstrokecolor{currentstroke}%
\pgfsetdash{}{0pt}%
\pgfpathmoveto{\pgfqpoint{1.179772in}{0.313488in}}%
\pgfpathlineto{\pgfqpoint{1.639057in}{0.313488in}}%
\pgfpathlineto{\pgfqpoint{1.639057in}{0.536649in}}%
\pgfpathlineto{\pgfqpoint{1.179772in}{0.536649in}}%
\pgfpathlineto{\pgfqpoint{1.179772in}{0.313488in}}%
\pgfpathclose%
\pgfusepath{stroke,fill}%
\end{pgfscope}%
\begin{pgfscope}%
\pgfpathrectangle{\pgfqpoint{0.439176in}{0.313488in}}{\pgfqpoint{3.662794in}{1.636512in}}%
\pgfusepath{clip}%
\pgfsetbuttcap%
\pgfsetmiterjoin%
\definecolor{currentfill}{rgb}{1.000000,0.756863,0.529412}%
\pgfsetfillcolor{currentfill}%
\pgfsetlinewidth{1.505625pt}%
\definecolor{currentstroke}{rgb}{0.921569,0.505882,0.105882}%
\pgfsetstrokecolor{currentstroke}%
\pgfsetdash{}{0pt}%
\pgfpathmoveto{\pgfqpoint{1.753878in}{0.313488in}}%
\pgfpathlineto{\pgfqpoint{2.213163in}{0.313488in}}%
\pgfpathlineto{\pgfqpoint{2.213163in}{0.908583in}}%
\pgfpathlineto{\pgfqpoint{1.753878in}{0.908583in}}%
\pgfpathlineto{\pgfqpoint{1.753878in}{0.313488in}}%
\pgfpathclose%
\pgfusepath{stroke,fill}%
\end{pgfscope}%
\begin{pgfscope}%
\pgfpathrectangle{\pgfqpoint{0.439176in}{0.313488in}}{\pgfqpoint{3.662794in}{1.636512in}}%
\pgfusepath{clip}%
\pgfsetbuttcap%
\pgfsetmiterjoin%
\definecolor{currentfill}{rgb}{1.000000,0.756863,0.529412}%
\pgfsetfillcolor{currentfill}%
\pgfsetlinewidth{1.505625pt}%
\definecolor{currentstroke}{rgb}{0.921569,0.505882,0.105882}%
\pgfsetstrokecolor{currentstroke}%
\pgfsetdash{}{0pt}%
\pgfpathmoveto{\pgfqpoint{2.327984in}{0.313488in}}%
\pgfpathlineto{\pgfqpoint{2.787268in}{0.313488in}}%
\pgfpathlineto{\pgfqpoint{2.787268in}{0.536649in}}%
\pgfpathlineto{\pgfqpoint{2.327984in}{0.536649in}}%
\pgfpathlineto{\pgfqpoint{2.327984in}{0.313488in}}%
\pgfpathclose%
\pgfusepath{stroke,fill}%
\end{pgfscope}%
\begin{pgfscope}%
\pgfpathrectangle{\pgfqpoint{0.439176in}{0.313488in}}{\pgfqpoint{3.662794in}{1.636512in}}%
\pgfusepath{clip}%
\pgfsetbuttcap%
\pgfsetmiterjoin%
\definecolor{currentfill}{rgb}{1.000000,0.756863,0.529412}%
\pgfsetfillcolor{currentfill}%
\pgfsetlinewidth{1.505625pt}%
\definecolor{currentstroke}{rgb}{0.921569,0.505882,0.105882}%
\pgfsetstrokecolor{currentstroke}%
\pgfsetdash{}{0pt}%
\pgfpathmoveto{\pgfqpoint{2.902089in}{0.313488in}}%
\pgfpathlineto{\pgfqpoint{3.361374in}{0.313488in}}%
\pgfpathlineto{\pgfqpoint{3.361374in}{0.387875in}}%
\pgfpathlineto{\pgfqpoint{2.902089in}{0.387875in}}%
\pgfpathlineto{\pgfqpoint{2.902089in}{0.313488in}}%
\pgfpathclose%
\pgfusepath{stroke,fill}%
\end{pgfscope}%
\begin{pgfscope}%
\pgfpathrectangle{\pgfqpoint{0.439176in}{0.313488in}}{\pgfqpoint{3.662794in}{1.636512in}}%
\pgfusepath{clip}%
\pgfsetbuttcap%
\pgfsetmiterjoin%
\definecolor{currentfill}{rgb}{1.000000,0.756863,0.529412}%
\pgfsetfillcolor{currentfill}%
\pgfsetlinewidth{1.505625pt}%
\definecolor{currentstroke}{rgb}{0.921569,0.505882,0.105882}%
\pgfsetstrokecolor{currentstroke}%
\pgfsetdash{}{0pt}%
\pgfpathmoveto{\pgfqpoint{3.476195in}{0.313488in}}%
\pgfpathlineto{\pgfqpoint{3.935479in}{0.313488in}}%
\pgfpathlineto{\pgfqpoint{3.935479in}{0.536649in}}%
\pgfpathlineto{\pgfqpoint{3.476195in}{0.536649in}}%
\pgfpathlineto{\pgfqpoint{3.476195in}{0.313488in}}%
\pgfpathclose%
\pgfusepath{stroke,fill}%
\end{pgfscope}%
\begin{pgfscope}%
\pgfsetrectcap%
\pgfsetmiterjoin%
\pgfsetlinewidth{1.003750pt}%
\definecolor{currentstroke}{rgb}{0.917647,0.917647,0.949020}%
\pgfsetstrokecolor{currentstroke}%
\pgfsetdash{}{0pt}%
\pgfpathmoveto{\pgfqpoint{0.439176in}{0.313488in}}%
\pgfpathlineto{\pgfqpoint{0.439176in}{1.950000in}}%
\pgfusepath{stroke}%
\end{pgfscope}%
\begin{pgfscope}%
\pgfsetrectcap%
\pgfsetmiterjoin%
\pgfsetlinewidth{1.003750pt}%
\definecolor{currentstroke}{rgb}{0.917647,0.917647,0.949020}%
\pgfsetstrokecolor{currentstroke}%
\pgfsetdash{}{0pt}%
\pgfpathmoveto{\pgfqpoint{4.101970in}{0.313488in}}%
\pgfpathlineto{\pgfqpoint{4.101970in}{1.950000in}}%
\pgfusepath{stroke}%
\end{pgfscope}%
\begin{pgfscope}%
\pgfsetrectcap%
\pgfsetmiterjoin%
\pgfsetlinewidth{1.003750pt}%
\definecolor{currentstroke}{rgb}{0.917647,0.917647,0.949020}%
\pgfsetstrokecolor{currentstroke}%
\pgfsetdash{}{0pt}%
\pgfpathmoveto{\pgfqpoint{0.439176in}{0.313488in}}%
\pgfpathlineto{\pgfqpoint{4.101970in}{0.313488in}}%
\pgfusepath{stroke}%
\end{pgfscope}%
\begin{pgfscope}%
\pgfsetrectcap%
\pgfsetmiterjoin%
\pgfsetlinewidth{1.003750pt}%
\definecolor{currentstroke}{rgb}{0.917647,0.917647,0.949020}%
\pgfsetstrokecolor{currentstroke}%
\pgfsetdash{}{0pt}%
\pgfpathmoveto{\pgfqpoint{0.439176in}{1.950000in}}%
\pgfpathlineto{\pgfqpoint{4.101970in}{1.950000in}}%
\pgfusepath{stroke}%
\end{pgfscope}%
\begin{pgfscope}%
\pgfsetbuttcap%
\pgfsetmiterjoin%
\definecolor{currentfill}{rgb}{1.000000,0.756863,0.529412}%
\pgfsetfillcolor{currentfill}%
\pgfsetlinewidth{1.505625pt}%
\definecolor{currentstroke}{rgb}{0.921569,0.505882,0.105882}%
\pgfsetstrokecolor{currentstroke}%
\pgfsetdash{}{0pt}%
\pgfpathmoveto{\pgfqpoint{2.932915in}{1.719477in}}%
\pgfpathlineto{\pgfqpoint{3.210693in}{1.719477in}}%
\pgfpathlineto{\pgfqpoint{3.210693in}{1.816699in}}%
\pgfpathlineto{\pgfqpoint{2.932915in}{1.816699in}}%
\pgfpathlineto{\pgfqpoint{2.932915in}{1.719477in}}%
\pgfpathclose%
\pgfusepath{stroke,fill}%
\end{pgfscope}%
\begin{pgfscope}%
\definecolor{textcolor}{rgb}{0.137255,0.215686,0.231373}%
\pgfsetstrokecolor{textcolor}%
\pgfsetfillcolor{textcolor}%
\pgftext[x=3.321804in,y=1.719477in,left,base]{\color{textcolor}\sffamily\fontsize{10.000000}{12.000000}\selectfont \(\displaystyle \psi_\theta\,\left(x\right)\)}%
\end{pgfscope}%
\begin{pgfscope}%
\pgfsetbuttcap%
\pgfsetmiterjoin%
\definecolor{currentfill}{rgb}{0.623529,0.780392,1.000000}%
\pgfsetfillcolor{currentfill}%
\pgfsetlinewidth{1.505625pt}%
\definecolor{currentstroke}{rgb}{0.298039,0.447059,0.690196}%
\pgfsetstrokecolor{currentstroke}%
\pgfsetdash{}{0pt}%
\pgfpathmoveto{\pgfqpoint{2.932915in}{1.509041in}}%
\pgfpathlineto{\pgfqpoint{3.210693in}{1.509041in}}%
\pgfpathlineto{\pgfqpoint{3.210693in}{1.606263in}}%
\pgfpathlineto{\pgfqpoint{2.932915in}{1.606263in}}%
\pgfpathlineto{\pgfqpoint{2.932915in}{1.509041in}}%
\pgfpathclose%
\pgfusepath{stroke,fill}%
\end{pgfscope}%
\begin{pgfscope}%
\definecolor{textcolor}{rgb}{0.137255,0.215686,0.231373}%
\pgfsetstrokecolor{textcolor}%
\pgfsetfillcolor{textcolor}%
\pgftext[x=3.321804in,y=1.509041in,left,base]{\color{textcolor}\sffamily\fontsize{10.000000}{12.000000}\selectfont \(\displaystyle \phi\,\left(\texttt{apple}\right)\)}%
\end{pgfscope}%
\end{pgfpicture}%
\makeatother%
\endgroup%

  }
  \only<4>{
    \vspace{-0.0cm}
    \input{figures/one-hot_cross-entropy_3.pgf}
  }

  \note[item]{Una funzione di Loss largamente impiegata nel training di modelli
  è la Cross Entropy. Definita per $p$ e $q$ distribuzioni di probabilità, è
  meno il prodotto scalare tra $q$ e il logaritmo di $p$. Nel nostro caso $p$
  è l'output del modello e $q$ l'encoding della classe.}
  \note[item]{Se $\phi$ è One-hot encoding solo una componente di $p$ da
  contributo, quella associata alla classe corretta}
  \note[item]{NEXT}
  \note[item]{Minimizzare $\mathcal{L}$ corrisponde ad alzare la probabilità
  relativa alla classe corretta e causa della normalizzazione di $\psi$ le
  probabilità associate alle altre classi saranno necessariamente ridotte.}
  \note[item]{NEXT, NEXT}
\end{frame}


  \section{Hierarchical Encodings}
  \begin{frame}{Hierarchical Encondings}
  Hierarchical encodings are derived from \alert{hierarchical trees}.

  \metroset{block=fill}
  \begin{block}{Example}
    \begin{align*}
      \mathcal{C} &= [
        &\texttt{lemon} &,  &\texttt{pear} &,  &\texttt{apple} &,
        &\texttt{dog} &, &\texttt{cat} &, &\texttt{car} &\
      ]
    \end{align*}
    \tikzfig{figures/example_hierarchical_tree}
  \end{block}

  \note[item]{Una gererchia tra le classi è un'informazione che possiamo
  sfruttare per costruire una funzione di encoding che tenga conto delle
  relazioni tra queste.}
  \note[item]{Un modo di rappresentare una gerachia è con una struttura ad
  albero.}
  \note[item]{Limone, pera e mela sono frutti, cane e gatto animali mentre l'auto è
  un veicolo. Frutti e animali sono presenti in natura, i veicoli sono
  oggetti artificiali.}
  \note[item]{Questo è un esempio di gerarchia dalla quale possiamo
  derivare un embedding gerarchico.}
\end{frame}

\begin{frame}{Hierarchical Encondings}
  \begin{columns}
    \column{0.7\textwidth}
    \only<1>{\tikzfig{figures/example_lca}}
    \only<2>{\tikzfig{figures/example_similarity}}
    \only<3>{\tikzfig{figures/example_probability}}
    \only<4>{\tikzfig{figures/example_probability_apple}}

    \column{0.3\textwidth}
    \begin{enumerate}[<+- | alert@+>]
      \item distance
      \item similarity
      \item probability
    \end{enumerate}
  \end{columns}

  \note[item]{Iniziamo costruendo una matrice di distanza tra le classi per
  esempio usando l'altezza dell'ultimo antenato comune. Ogni classe ha distanza
  0 da se stessa. I frutti hanno distanza 1 tra loro, 2 con gli animali e 3,
  distanza massima con i veicoli.}
  \note[item]{Dalla distanza construiamo una matrice di similarità. 1 -
  distanza normalizzata. Le classi avranno similarità 1 con se stesse e 0 con
  quelle con cui non hanno niente in comune.}
  \note[item]{Infine, otteniamo una probability mass function sulle classi
  applicando la softmax lungo le righe. E ora ogni righa corrisponde
  all'encoding di una classe.}
  \note[item]{NEXT}
  \note[item]{Consideriamo l'encoding per la classe "mela".}
\end{frame}

\begin{frame}{Hierarchical Encondings}
  \only<1>{
    \centering
    similarity $\longrightarrow$ probability
    \vspace{-0.0cm}
    %% Creator: Matplotlib, PGF backend
%%
%% To include the figure in your LaTeX document, write
%%   \input{<filename>.pgf}
%%
%% Make sure the required packages are loaded in your preamble
%%   \usepackage{pgf}
%%
%% Also ensure that all the required font packages are loaded; for instance,
%% the lmodern package is sometimes necessary when using math font.
%%   \usepackage{lmodern}
%%
%% Figures using additional raster images can only be included by \input if
%% they are in the same directory as the main LaTeX file. For loading figures
%% from other directories you can use the `import` package
%%   \usepackage{import}
%%
%% and then include the figures with
%%   \import{<path to file>}{<filename>.pgf}
%%
%% Matplotlib used the following preamble
%%   
%%   \usepackage{fontspec}
%%   \setmainfont{DejaVuSerif.ttf}[Path=\detokenize{/Users/simo/.local/share/virtualenvs/master-thesis-code/lib/python3.10/site-packages/matplotlib/mpl-data/fonts/ttf/}]
%%   \setsansfont{DejaVuSans.ttf}[Path=\detokenize{/Users/simo/.local/share/virtualenvs/master-thesis-code/lib/python3.10/site-packages/matplotlib/mpl-data/fonts/ttf/}]
%%   \setmonofont{DejaVuSansMono.ttf}[Path=\detokenize{/Users/simo/.local/share/virtualenvs/master-thesis-code/lib/python3.10/site-packages/matplotlib/mpl-data/fonts/ttf/}]
%%   \makeatletter\@ifpackageloaded{underscore}{}{\usepackage[strings]{underscore}}\makeatother
%%
\begingroup%
\makeatletter%
\begin{pgfpicture}%
\pgfpathrectangle{\pgfpointorigin}{\pgfqpoint{4.251970in}{2.400000in}}%
\pgfusepath{use as bounding box, clip}%
\begin{pgfscope}%
\pgfsetbuttcap%
\pgfsetmiterjoin%
\definecolor{currentfill}{rgb}{0.980392,0.980392,0.980392}%
\pgfsetfillcolor{currentfill}%
\pgfsetlinewidth{0.000000pt}%
\definecolor{currentstroke}{rgb}{1.000000,1.000000,1.000000}%
\pgfsetstrokecolor{currentstroke}%
\pgfsetdash{}{0pt}%
\pgfpathmoveto{\pgfqpoint{0.000000in}{0.000000in}}%
\pgfpathlineto{\pgfqpoint{4.251970in}{0.000000in}}%
\pgfpathlineto{\pgfqpoint{4.251970in}{2.400000in}}%
\pgfpathlineto{\pgfqpoint{0.000000in}{2.400000in}}%
\pgfpathlineto{\pgfqpoint{0.000000in}{0.000000in}}%
\pgfpathclose%
\pgfusepath{fill}%
\end{pgfscope}%
\begin{pgfscope}%
\pgfsetbuttcap%
\pgfsetmiterjoin%
\definecolor{currentfill}{rgb}{0.917647,0.917647,0.949020}%
\pgfsetfillcolor{currentfill}%
\pgfsetlinewidth{0.000000pt}%
\definecolor{currentstroke}{rgb}{0.000000,0.000000,0.000000}%
\pgfsetstrokecolor{currentstroke}%
\pgfsetstrokeopacity{0.000000}%
\pgfsetdash{}{0pt}%
\pgfpathmoveto{\pgfqpoint{0.439176in}{0.313488in}}%
\pgfpathlineto{\pgfqpoint{4.101970in}{0.313488in}}%
\pgfpathlineto{\pgfqpoint{4.101970in}{2.250000in}}%
\pgfpathlineto{\pgfqpoint{0.439176in}{2.250000in}}%
\pgfpathlineto{\pgfqpoint{0.439176in}{0.313488in}}%
\pgfpathclose%
\pgfusepath{fill}%
\end{pgfscope}%
\begin{pgfscope}%
\definecolor{textcolor}{rgb}{0.137255,0.215686,0.231373}%
\pgfsetstrokecolor{textcolor}%
\pgfsetfillcolor{textcolor}%
\pgftext[x=0.835309in,y=0.174599in,,base]{\color{textcolor}\sffamily\fontsize{8.000000}{9.600000}\selectfont \texttt{lemon}}%
\end{pgfscope}%
\begin{pgfscope}%
\definecolor{textcolor}{rgb}{0.137255,0.215686,0.231373}%
\pgfsetstrokecolor{textcolor}%
\pgfsetfillcolor{textcolor}%
\pgftext[x=1.409415in,y=0.174599in,,base]{\color{textcolor}\sffamily\fontsize{8.000000}{9.600000}\selectfont \texttt{pear}}%
\end{pgfscope}%
\begin{pgfscope}%
\definecolor{textcolor}{rgb}{0.137255,0.215686,0.231373}%
\pgfsetstrokecolor{textcolor}%
\pgfsetfillcolor{textcolor}%
\pgftext[x=1.983520in,y=0.174599in,,base]{\color{textcolor}\sffamily\fontsize{8.000000}{9.600000}\selectfont \texttt{apple}}%
\end{pgfscope}%
\begin{pgfscope}%
\definecolor{textcolor}{rgb}{0.137255,0.215686,0.231373}%
\pgfsetstrokecolor{textcolor}%
\pgfsetfillcolor{textcolor}%
\pgftext[x=2.557626in,y=0.174599in,,base]{\color{textcolor}\sffamily\fontsize{8.000000}{9.600000}\selectfont \texttt{dog}}%
\end{pgfscope}%
\begin{pgfscope}%
\definecolor{textcolor}{rgb}{0.137255,0.215686,0.231373}%
\pgfsetstrokecolor{textcolor}%
\pgfsetfillcolor{textcolor}%
\pgftext[x=3.131732in,y=0.174599in,,base]{\color{textcolor}\sffamily\fontsize{8.000000}{9.600000}\selectfont \texttt{cat}}%
\end{pgfscope}%
\begin{pgfscope}%
\definecolor{textcolor}{rgb}{0.137255,0.215686,0.231373}%
\pgfsetstrokecolor{textcolor}%
\pgfsetfillcolor{textcolor}%
\pgftext[x=3.705837in,y=0.174599in,,base]{\color{textcolor}\sffamily\fontsize{8.000000}{9.600000}\selectfont \texttt{car}}%
\end{pgfscope}%
\begin{pgfscope}%
\definecolor{textcolor}{rgb}{0.137255,0.215686,0.231373}%
\pgfsetstrokecolor{textcolor}%
\pgfsetfillcolor{textcolor}%
\pgftext[x=0.149436in, y=0.271279in, left, base]{\color{textcolor}\sffamily\fontsize{8.000000}{9.600000}\selectfont \(\displaystyle {0.0}\)}%
\end{pgfscope}%
\begin{pgfscope}%
\definecolor{textcolor}{rgb}{0.137255,0.215686,0.231373}%
\pgfsetstrokecolor{textcolor}%
\pgfsetfillcolor{textcolor}%
\pgftext[x=0.149436in, y=0.623372in, left, base]{\color{textcolor}\sffamily\fontsize{8.000000}{9.600000}\selectfont \(\displaystyle {0.2}\)}%
\end{pgfscope}%
\begin{pgfscope}%
\definecolor{textcolor}{rgb}{0.137255,0.215686,0.231373}%
\pgfsetstrokecolor{textcolor}%
\pgfsetfillcolor{textcolor}%
\pgftext[x=0.149436in, y=0.975465in, left, base]{\color{textcolor}\sffamily\fontsize{8.000000}{9.600000}\selectfont \(\displaystyle {0.4}\)}%
\end{pgfscope}%
\begin{pgfscope}%
\definecolor{textcolor}{rgb}{0.137255,0.215686,0.231373}%
\pgfsetstrokecolor{textcolor}%
\pgfsetfillcolor{textcolor}%
\pgftext[x=0.149436in, y=1.327558in, left, base]{\color{textcolor}\sffamily\fontsize{8.000000}{9.600000}\selectfont \(\displaystyle {0.6}\)}%
\end{pgfscope}%
\begin{pgfscope}%
\definecolor{textcolor}{rgb}{0.137255,0.215686,0.231373}%
\pgfsetstrokecolor{textcolor}%
\pgfsetfillcolor{textcolor}%
\pgftext[x=0.149436in, y=1.679651in, left, base]{\color{textcolor}\sffamily\fontsize{8.000000}{9.600000}\selectfont \(\displaystyle {0.8}\)}%
\end{pgfscope}%
\begin{pgfscope}%
\definecolor{textcolor}{rgb}{0.137255,0.215686,0.231373}%
\pgfsetstrokecolor{textcolor}%
\pgfsetfillcolor{textcolor}%
\pgftext[x=0.149436in, y=2.031744in, left, base]{\color{textcolor}\sffamily\fontsize{8.000000}{9.600000}\selectfont \(\displaystyle {1.0}\)}%
\end{pgfscope}%
\begin{pgfscope}%
\pgfpathrectangle{\pgfqpoint{0.439176in}{0.313488in}}{\pgfqpoint{3.662794in}{1.936512in}}%
\pgfusepath{clip}%
\pgfsetbuttcap%
\pgfsetmiterjoin%
\definecolor{currentfill}{rgb}{0.623529,0.780392,1.000000}%
\pgfsetfillcolor{currentfill}%
\pgfsetlinewidth{1.505625pt}%
\definecolor{currentstroke}{rgb}{0.298039,0.447059,0.690196}%
\pgfsetstrokecolor{currentstroke}%
\pgfsetdash{}{0pt}%
\pgfpathmoveto{\pgfqpoint{0.605667in}{0.313488in}}%
\pgfpathlineto{\pgfqpoint{1.064951in}{0.313488in}}%
\pgfpathlineto{\pgfqpoint{1.064951in}{0.643034in}}%
\pgfpathlineto{\pgfqpoint{0.605667in}{0.643034in}}%
\pgfpathlineto{\pgfqpoint{0.605667in}{0.313488in}}%
\pgfpathclose%
\pgfusepath{stroke,fill}%
\end{pgfscope}%
\begin{pgfscope}%
\pgfpathrectangle{\pgfqpoint{0.439176in}{0.313488in}}{\pgfqpoint{3.662794in}{1.936512in}}%
\pgfusepath{clip}%
\pgfsetbuttcap%
\pgfsetmiterjoin%
\definecolor{currentfill}{rgb}{0.623529,0.780392,1.000000}%
\pgfsetfillcolor{currentfill}%
\pgfsetlinewidth{1.505625pt}%
\definecolor{currentstroke}{rgb}{0.298039,0.447059,0.690196}%
\pgfsetstrokecolor{currentstroke}%
\pgfsetdash{}{0pt}%
\pgfpathmoveto{\pgfqpoint{1.179772in}{0.313488in}}%
\pgfpathlineto{\pgfqpoint{1.639057in}{0.313488in}}%
\pgfpathlineto{\pgfqpoint{1.639057in}{0.643034in}}%
\pgfpathlineto{\pgfqpoint{1.179772in}{0.643034in}}%
\pgfpathlineto{\pgfqpoint{1.179772in}{0.313488in}}%
\pgfpathclose%
\pgfusepath{stroke,fill}%
\end{pgfscope}%
\begin{pgfscope}%
\pgfpathrectangle{\pgfqpoint{0.439176in}{0.313488in}}{\pgfqpoint{3.662794in}{1.936512in}}%
\pgfusepath{clip}%
\pgfsetbuttcap%
\pgfsetmiterjoin%
\definecolor{currentfill}{rgb}{0.623529,0.780392,1.000000}%
\pgfsetfillcolor{currentfill}%
\pgfsetlinewidth{1.505625pt}%
\definecolor{currentstroke}{rgb}{0.298039,0.447059,0.690196}%
\pgfsetstrokecolor{currentstroke}%
\pgfsetdash{}{0pt}%
\pgfpathmoveto{\pgfqpoint{1.753878in}{0.313488in}}%
\pgfpathlineto{\pgfqpoint{2.213163in}{0.313488in}}%
\pgfpathlineto{\pgfqpoint{2.213163in}{0.773407in}}%
\pgfpathlineto{\pgfqpoint{1.753878in}{0.773407in}}%
\pgfpathlineto{\pgfqpoint{1.753878in}{0.313488in}}%
\pgfpathclose%
\pgfusepath{stroke,fill}%
\end{pgfscope}%
\begin{pgfscope}%
\pgfpathrectangle{\pgfqpoint{0.439176in}{0.313488in}}{\pgfqpoint{3.662794in}{1.936512in}}%
\pgfusepath{clip}%
\pgfsetbuttcap%
\pgfsetmiterjoin%
\definecolor{currentfill}{rgb}{0.623529,0.780392,1.000000}%
\pgfsetfillcolor{currentfill}%
\pgfsetlinewidth{1.505625pt}%
\definecolor{currentstroke}{rgb}{0.298039,0.447059,0.690196}%
\pgfsetstrokecolor{currentstroke}%
\pgfsetdash{}{0pt}%
\pgfpathmoveto{\pgfqpoint{2.327984in}{0.313488in}}%
\pgfpathlineto{\pgfqpoint{2.787268in}{0.313488in}}%
\pgfpathlineto{\pgfqpoint{2.787268in}{0.549618in}}%
\pgfpathlineto{\pgfqpoint{2.327984in}{0.549618in}}%
\pgfpathlineto{\pgfqpoint{2.327984in}{0.313488in}}%
\pgfpathclose%
\pgfusepath{stroke,fill}%
\end{pgfscope}%
\begin{pgfscope}%
\pgfpathrectangle{\pgfqpoint{0.439176in}{0.313488in}}{\pgfqpoint{3.662794in}{1.936512in}}%
\pgfusepath{clip}%
\pgfsetbuttcap%
\pgfsetmiterjoin%
\definecolor{currentfill}{rgb}{0.623529,0.780392,1.000000}%
\pgfsetfillcolor{currentfill}%
\pgfsetlinewidth{1.505625pt}%
\definecolor{currentstroke}{rgb}{0.298039,0.447059,0.690196}%
\pgfsetstrokecolor{currentstroke}%
\pgfsetdash{}{0pt}%
\pgfpathmoveto{\pgfqpoint{2.902089in}{0.313488in}}%
\pgfpathlineto{\pgfqpoint{3.361374in}{0.313488in}}%
\pgfpathlineto{\pgfqpoint{3.361374in}{0.549618in}}%
\pgfpathlineto{\pgfqpoint{2.902089in}{0.549618in}}%
\pgfpathlineto{\pgfqpoint{2.902089in}{0.313488in}}%
\pgfpathclose%
\pgfusepath{stroke,fill}%
\end{pgfscope}%
\begin{pgfscope}%
\pgfpathrectangle{\pgfqpoint{0.439176in}{0.313488in}}{\pgfqpoint{3.662794in}{1.936512in}}%
\pgfusepath{clip}%
\pgfsetbuttcap%
\pgfsetmiterjoin%
\definecolor{currentfill}{rgb}{0.623529,0.780392,1.000000}%
\pgfsetfillcolor{currentfill}%
\pgfsetlinewidth{1.505625pt}%
\definecolor{currentstroke}{rgb}{0.298039,0.447059,0.690196}%
\pgfsetstrokecolor{currentstroke}%
\pgfsetdash{}{0pt}%
\pgfpathmoveto{\pgfqpoint{3.476195in}{0.313488in}}%
\pgfpathlineto{\pgfqpoint{3.935479in}{0.313488in}}%
\pgfpathlineto{\pgfqpoint{3.935479in}{0.482683in}}%
\pgfpathlineto{\pgfqpoint{3.476195in}{0.482683in}}%
\pgfpathlineto{\pgfqpoint{3.476195in}{0.313488in}}%
\pgfpathclose%
\pgfusepath{stroke,fill}%
\end{pgfscope}%
\begin{pgfscope}%
\pgfsetrectcap%
\pgfsetmiterjoin%
\pgfsetlinewidth{1.003750pt}%
\definecolor{currentstroke}{rgb}{0.917647,0.917647,0.949020}%
\pgfsetstrokecolor{currentstroke}%
\pgfsetdash{}{0pt}%
\pgfpathmoveto{\pgfqpoint{0.439176in}{0.313488in}}%
\pgfpathlineto{\pgfqpoint{0.439176in}{2.250000in}}%
\pgfusepath{stroke}%
\end{pgfscope}%
\begin{pgfscope}%
\pgfsetrectcap%
\pgfsetmiterjoin%
\pgfsetlinewidth{1.003750pt}%
\definecolor{currentstroke}{rgb}{0.917647,0.917647,0.949020}%
\pgfsetstrokecolor{currentstroke}%
\pgfsetdash{}{0pt}%
\pgfpathmoveto{\pgfqpoint{4.101970in}{0.313488in}}%
\pgfpathlineto{\pgfqpoint{4.101970in}{2.250000in}}%
\pgfusepath{stroke}%
\end{pgfscope}%
\begin{pgfscope}%
\pgfsetrectcap%
\pgfsetmiterjoin%
\pgfsetlinewidth{1.003750pt}%
\definecolor{currentstroke}{rgb}{0.917647,0.917647,0.949020}%
\pgfsetstrokecolor{currentstroke}%
\pgfsetdash{}{0pt}%
\pgfpathmoveto{\pgfqpoint{0.439176in}{0.313488in}}%
\pgfpathlineto{\pgfqpoint{4.101970in}{0.313488in}}%
\pgfusepath{stroke}%
\end{pgfscope}%
\begin{pgfscope}%
\pgfsetrectcap%
\pgfsetmiterjoin%
\pgfsetlinewidth{1.003750pt}%
\definecolor{currentstroke}{rgb}{0.917647,0.917647,0.949020}%
\pgfsetstrokecolor{currentstroke}%
\pgfsetdash{}{0pt}%
\pgfpathmoveto{\pgfqpoint{0.439176in}{2.250000in}}%
\pgfpathlineto{\pgfqpoint{4.101970in}{2.250000in}}%
\pgfusepath{stroke}%
\end{pgfscope}%
\begin{pgfscope}%
\pgfsetbuttcap%
\pgfsetmiterjoin%
\definecolor{currentfill}{rgb}{0.623529,0.780392,1.000000}%
\pgfsetfillcolor{currentfill}%
\pgfsetlinewidth{1.505625pt}%
\definecolor{currentstroke}{rgb}{0.298039,0.447059,0.690196}%
\pgfsetstrokecolor{currentstroke}%
\pgfsetdash{}{0pt}%
\pgfpathmoveto{\pgfqpoint{2.932915in}{2.018731in}}%
\pgfpathlineto{\pgfqpoint{3.210693in}{2.018731in}}%
\pgfpathlineto{\pgfqpoint{3.210693in}{2.115953in}}%
\pgfpathlineto{\pgfqpoint{2.932915in}{2.115953in}}%
\pgfpathlineto{\pgfqpoint{2.932915in}{2.018731in}}%
\pgfpathclose%
\pgfusepath{stroke,fill}%
\end{pgfscope}%
\begin{pgfscope}%
\definecolor{textcolor}{rgb}{0.137255,0.215686,0.231373}%
\pgfsetstrokecolor{textcolor}%
\pgfsetfillcolor{textcolor}%
\pgftext[x=3.321804in,y=2.018731in,left,base]{\color{textcolor}\sffamily\fontsize{10.000000}{12.000000}\selectfont \(\displaystyle \phi\,\left(\texttt{apple}\right)\)}%
\end{pgfscope}%
\end{pgfpicture}%
\makeatother%
\endgroup%

  }
  \only<2>{
    \centering
    similarity + hyperparam $\longrightarrow$ probability
    \vspace{-0.0cm}
    \input{figures/hier_cross-entropy_2.pgf}
  }
  \only<3>{
    \centering
    $\mathcal{L} = - \phi\left(\texttt{apple}\right) \cdot \log \psi_\theta\left(x\right)$
    \vspace{-0.0cm}
    \input{figures/hier_cross-entropy_3.pgf}
  }
  \only<4>{
    \centering
    $\mathcal{L} = - \phi\left(\texttt{apple}\right) \cdot \log \psi_\theta\left(x\right)$
    \vspace{-0.0cm}
    %% Creator: Matplotlib, PGF backend
%%
%% To include the figure in your LaTeX document, write
%%   \input{<filename>.pgf}
%%
%% Make sure the required packages are loaded in your preamble
%%   \usepackage{pgf}
%%
%% Also ensure that all the required font packages are loaded; for instance,
%% the lmodern package is sometimes necessary when using math font.
%%   \usepackage{lmodern}
%%
%% Figures using additional raster images can only be included by \input if
%% they are in the same directory as the main LaTeX file. For loading figures
%% from other directories you can use the `import` package
%%   \usepackage{import}
%%
%% and then include the figures with
%%   \import{<path to file>}{<filename>.pgf}
%%
%% Matplotlib used the following preamble
%%   
%%   \usepackage{fontspec}
%%   \setmainfont{DejaVuSerif.ttf}[Path=\detokenize{/Users/simo/.local/share/virtualenvs/master-thesis-code/lib/python3.10/site-packages/matplotlib/mpl-data/fonts/ttf/}]
%%   \setsansfont{DejaVuSans.ttf}[Path=\detokenize{/Users/simo/.local/share/virtualenvs/master-thesis-code/lib/python3.10/site-packages/matplotlib/mpl-data/fonts/ttf/}]
%%   \setmonofont{DejaVuSansMono.ttf}[Path=\detokenize{/Users/simo/.local/share/virtualenvs/master-thesis-code/lib/python3.10/site-packages/matplotlib/mpl-data/fonts/ttf/}]
%%   \makeatletter\@ifpackageloaded{underscore}{}{\usepackage[strings]{underscore}}\makeatother
%%
\begingroup%
\makeatletter%
\begin{pgfpicture}%
\pgfpathrectangle{\pgfpointorigin}{\pgfqpoint{4.251970in}{2.400000in}}%
\pgfusepath{use as bounding box, clip}%
\begin{pgfscope}%
\pgfsetbuttcap%
\pgfsetmiterjoin%
\definecolor{currentfill}{rgb}{0.980392,0.980392,0.980392}%
\pgfsetfillcolor{currentfill}%
\pgfsetlinewidth{0.000000pt}%
\definecolor{currentstroke}{rgb}{1.000000,1.000000,1.000000}%
\pgfsetstrokecolor{currentstroke}%
\pgfsetdash{}{0pt}%
\pgfpathmoveto{\pgfqpoint{0.000000in}{0.000000in}}%
\pgfpathlineto{\pgfqpoint{4.251970in}{0.000000in}}%
\pgfpathlineto{\pgfqpoint{4.251970in}{2.400000in}}%
\pgfpathlineto{\pgfqpoint{0.000000in}{2.400000in}}%
\pgfpathlineto{\pgfqpoint{0.000000in}{0.000000in}}%
\pgfpathclose%
\pgfusepath{fill}%
\end{pgfscope}%
\begin{pgfscope}%
\pgfsetbuttcap%
\pgfsetmiterjoin%
\definecolor{currentfill}{rgb}{0.917647,0.917647,0.949020}%
\pgfsetfillcolor{currentfill}%
\pgfsetlinewidth{0.000000pt}%
\definecolor{currentstroke}{rgb}{0.000000,0.000000,0.000000}%
\pgfsetstrokecolor{currentstroke}%
\pgfsetstrokeopacity{0.000000}%
\pgfsetdash{}{0pt}%
\pgfpathmoveto{\pgfqpoint{0.439176in}{0.313488in}}%
\pgfpathlineto{\pgfqpoint{4.101970in}{0.313488in}}%
\pgfpathlineto{\pgfqpoint{4.101970in}{2.250000in}}%
\pgfpathlineto{\pgfqpoint{0.439176in}{2.250000in}}%
\pgfpathlineto{\pgfqpoint{0.439176in}{0.313488in}}%
\pgfpathclose%
\pgfusepath{fill}%
\end{pgfscope}%
\begin{pgfscope}%
\definecolor{textcolor}{rgb}{0.137255,0.215686,0.231373}%
\pgfsetstrokecolor{textcolor}%
\pgfsetfillcolor{textcolor}%
\pgftext[x=0.781390in,y=0.174599in,,base]{\color{textcolor}\sffamily\fontsize{8.000000}{9.600000}\selectfont \texttt{lemon}}%
\end{pgfscope}%
\begin{pgfscope}%
\definecolor{textcolor}{rgb}{0.137255,0.215686,0.231373}%
\pgfsetstrokecolor{textcolor}%
\pgfsetfillcolor{textcolor}%
\pgftext[x=1.377063in,y=0.174599in,,base]{\color{textcolor}\sffamily\fontsize{8.000000}{9.600000}\selectfont \texttt{pear}}%
\end{pgfscope}%
\begin{pgfscope}%
\definecolor{textcolor}{rgb}{0.137255,0.215686,0.231373}%
\pgfsetstrokecolor{textcolor}%
\pgfsetfillcolor{textcolor}%
\pgftext[x=1.972737in,y=0.174599in,,base]{\color{textcolor}\sffamily\fontsize{8.000000}{9.600000}\selectfont \texttt{apple}}%
\end{pgfscope}%
\begin{pgfscope}%
\definecolor{textcolor}{rgb}{0.137255,0.215686,0.231373}%
\pgfsetstrokecolor{textcolor}%
\pgfsetfillcolor{textcolor}%
\pgftext[x=2.568410in,y=0.174599in,,base]{\color{textcolor}\sffamily\fontsize{8.000000}{9.600000}\selectfont \texttt{dog}}%
\end{pgfscope}%
\begin{pgfscope}%
\definecolor{textcolor}{rgb}{0.137255,0.215686,0.231373}%
\pgfsetstrokecolor{textcolor}%
\pgfsetfillcolor{textcolor}%
\pgftext[x=3.164083in,y=0.174599in,,base]{\color{textcolor}\sffamily\fontsize{8.000000}{9.600000}\selectfont \texttt{cat}}%
\end{pgfscope}%
\begin{pgfscope}%
\definecolor{textcolor}{rgb}{0.137255,0.215686,0.231373}%
\pgfsetstrokecolor{textcolor}%
\pgfsetfillcolor{textcolor}%
\pgftext[x=3.759756in,y=0.174599in,,base]{\color{textcolor}\sffamily\fontsize{8.000000}{9.600000}\selectfont \texttt{car}}%
\end{pgfscope}%
\begin{pgfscope}%
\definecolor{textcolor}{rgb}{0.137255,0.215686,0.231373}%
\pgfsetstrokecolor{textcolor}%
\pgfsetfillcolor{textcolor}%
\pgftext[x=0.149436in, y=0.271279in, left, base]{\color{textcolor}\sffamily\fontsize{8.000000}{9.600000}\selectfont \(\displaystyle {0.0}\)}%
\end{pgfscope}%
\begin{pgfscope}%
\definecolor{textcolor}{rgb}{0.137255,0.215686,0.231373}%
\pgfsetstrokecolor{textcolor}%
\pgfsetfillcolor{textcolor}%
\pgftext[x=0.149436in, y=0.623372in, left, base]{\color{textcolor}\sffamily\fontsize{8.000000}{9.600000}\selectfont \(\displaystyle {0.2}\)}%
\end{pgfscope}%
\begin{pgfscope}%
\definecolor{textcolor}{rgb}{0.137255,0.215686,0.231373}%
\pgfsetstrokecolor{textcolor}%
\pgfsetfillcolor{textcolor}%
\pgftext[x=0.149436in, y=0.975465in, left, base]{\color{textcolor}\sffamily\fontsize{8.000000}{9.600000}\selectfont \(\displaystyle {0.4}\)}%
\end{pgfscope}%
\begin{pgfscope}%
\definecolor{textcolor}{rgb}{0.137255,0.215686,0.231373}%
\pgfsetstrokecolor{textcolor}%
\pgfsetfillcolor{textcolor}%
\pgftext[x=0.149436in, y=1.327558in, left, base]{\color{textcolor}\sffamily\fontsize{8.000000}{9.600000}\selectfont \(\displaystyle {0.6}\)}%
\end{pgfscope}%
\begin{pgfscope}%
\definecolor{textcolor}{rgb}{0.137255,0.215686,0.231373}%
\pgfsetstrokecolor{textcolor}%
\pgfsetfillcolor{textcolor}%
\pgftext[x=0.149436in, y=1.679651in, left, base]{\color{textcolor}\sffamily\fontsize{8.000000}{9.600000}\selectfont \(\displaystyle {0.8}\)}%
\end{pgfscope}%
\begin{pgfscope}%
\definecolor{textcolor}{rgb}{0.137255,0.215686,0.231373}%
\pgfsetstrokecolor{textcolor}%
\pgfsetfillcolor{textcolor}%
\pgftext[x=0.149436in, y=2.031744in, left, base]{\color{textcolor}\sffamily\fontsize{8.000000}{9.600000}\selectfont \(\displaystyle {1.0}\)}%
\end{pgfscope}%
\begin{pgfscope}%
\pgfpathrectangle{\pgfqpoint{0.439176in}{0.313488in}}{\pgfqpoint{3.662794in}{1.936512in}}%
\pgfusepath{clip}%
\pgfsetbuttcap%
\pgfsetmiterjoin%
\definecolor{currentfill}{rgb}{0.623529,0.780392,1.000000}%
\pgfsetfillcolor{currentfill}%
\pgfsetlinewidth{1.505625pt}%
\definecolor{currentstroke}{rgb}{0.298039,0.447059,0.690196}%
\pgfsetstrokecolor{currentstroke}%
\pgfsetdash{}{0pt}%
\pgfpathmoveto{\pgfqpoint{0.808196in}{0.313488in}}%
\pgfpathlineto{\pgfqpoint{0.957114in}{0.313488in}}%
\pgfpathlineto{\pgfqpoint{0.957114in}{0.588976in}}%
\pgfpathlineto{\pgfqpoint{0.808196in}{0.588976in}}%
\pgfpathlineto{\pgfqpoint{0.808196in}{0.313488in}}%
\pgfpathclose%
\pgfusepath{stroke,fill}%
\end{pgfscope}%
\begin{pgfscope}%
\pgfpathrectangle{\pgfqpoint{0.439176in}{0.313488in}}{\pgfqpoint{3.662794in}{1.936512in}}%
\pgfusepath{clip}%
\pgfsetbuttcap%
\pgfsetmiterjoin%
\definecolor{currentfill}{rgb}{0.623529,0.780392,1.000000}%
\pgfsetfillcolor{currentfill}%
\pgfsetlinewidth{1.505625pt}%
\definecolor{currentstroke}{rgb}{0.298039,0.447059,0.690196}%
\pgfsetstrokecolor{currentstroke}%
\pgfsetdash{}{0pt}%
\pgfpathmoveto{\pgfqpoint{1.403869in}{0.313488in}}%
\pgfpathlineto{\pgfqpoint{1.552787in}{0.313488in}}%
\pgfpathlineto{\pgfqpoint{1.552787in}{0.588976in}}%
\pgfpathlineto{\pgfqpoint{1.403869in}{0.588976in}}%
\pgfpathlineto{\pgfqpoint{1.403869in}{0.313488in}}%
\pgfpathclose%
\pgfusepath{stroke,fill}%
\end{pgfscope}%
\begin{pgfscope}%
\pgfpathrectangle{\pgfqpoint{0.439176in}{0.313488in}}{\pgfqpoint{3.662794in}{1.936512in}}%
\pgfusepath{clip}%
\pgfsetbuttcap%
\pgfsetmiterjoin%
\definecolor{currentfill}{rgb}{0.623529,0.780392,1.000000}%
\pgfsetfillcolor{currentfill}%
\pgfsetlinewidth{1.505625pt}%
\definecolor{currentstroke}{rgb}{0.298039,0.447059,0.690196}%
\pgfsetstrokecolor{currentstroke}%
\pgfsetdash{}{0pt}%
\pgfpathmoveto{\pgfqpoint{1.999542in}{0.313488in}}%
\pgfpathlineto{\pgfqpoint{2.148460in}{0.313488in}}%
\pgfpathlineto{\pgfqpoint{2.148460in}{1.358599in}}%
\pgfpathlineto{\pgfqpoint{1.999542in}{1.358599in}}%
\pgfpathlineto{\pgfqpoint{1.999542in}{0.313488in}}%
\pgfpathclose%
\pgfusepath{stroke,fill}%
\end{pgfscope}%
\begin{pgfscope}%
\pgfpathrectangle{\pgfqpoint{0.439176in}{0.313488in}}{\pgfqpoint{3.662794in}{1.936512in}}%
\pgfusepath{clip}%
\pgfsetbuttcap%
\pgfsetmiterjoin%
\definecolor{currentfill}{rgb}{0.623529,0.780392,1.000000}%
\pgfsetfillcolor{currentfill}%
\pgfsetlinewidth{1.505625pt}%
\definecolor{currentstroke}{rgb}{0.298039,0.447059,0.690196}%
\pgfsetstrokecolor{currentstroke}%
\pgfsetdash{}{0pt}%
\pgfpathmoveto{\pgfqpoint{2.595215in}{0.313488in}}%
\pgfpathlineto{\pgfqpoint{2.744133in}{0.313488in}}%
\pgfpathlineto{\pgfqpoint{2.744133in}{0.386106in}}%
\pgfpathlineto{\pgfqpoint{2.595215in}{0.386106in}}%
\pgfpathlineto{\pgfqpoint{2.595215in}{0.313488in}}%
\pgfpathclose%
\pgfusepath{stroke,fill}%
\end{pgfscope}%
\begin{pgfscope}%
\pgfpathrectangle{\pgfqpoint{0.439176in}{0.313488in}}{\pgfqpoint{3.662794in}{1.936512in}}%
\pgfusepath{clip}%
\pgfsetbuttcap%
\pgfsetmiterjoin%
\definecolor{currentfill}{rgb}{0.623529,0.780392,1.000000}%
\pgfsetfillcolor{currentfill}%
\pgfsetlinewidth{1.505625pt}%
\definecolor{currentstroke}{rgb}{0.298039,0.447059,0.690196}%
\pgfsetstrokecolor{currentstroke}%
\pgfsetdash{}{0pt}%
\pgfpathmoveto{\pgfqpoint{3.190888in}{0.313488in}}%
\pgfpathlineto{\pgfqpoint{3.339806in}{0.313488in}}%
\pgfpathlineto{\pgfqpoint{3.339806in}{0.386106in}}%
\pgfpathlineto{\pgfqpoint{3.190888in}{0.386106in}}%
\pgfpathlineto{\pgfqpoint{3.190888in}{0.313488in}}%
\pgfpathclose%
\pgfusepath{stroke,fill}%
\end{pgfscope}%
\begin{pgfscope}%
\pgfpathrectangle{\pgfqpoint{0.439176in}{0.313488in}}{\pgfqpoint{3.662794in}{1.936512in}}%
\pgfusepath{clip}%
\pgfsetbuttcap%
\pgfsetmiterjoin%
\definecolor{currentfill}{rgb}{0.623529,0.780392,1.000000}%
\pgfsetfillcolor{currentfill}%
\pgfsetlinewidth{1.505625pt}%
\definecolor{currentstroke}{rgb}{0.298039,0.447059,0.690196}%
\pgfsetstrokecolor{currentstroke}%
\pgfsetdash{}{0pt}%
\pgfpathmoveto{\pgfqpoint{3.786561in}{0.313488in}}%
\pgfpathlineto{\pgfqpoint{3.935479in}{0.313488in}}%
\pgfpathlineto{\pgfqpoint{3.935479in}{0.332630in}}%
\pgfpathlineto{\pgfqpoint{3.786561in}{0.332630in}}%
\pgfpathlineto{\pgfqpoint{3.786561in}{0.313488in}}%
\pgfpathclose%
\pgfusepath{stroke,fill}%
\end{pgfscope}%
\begin{pgfscope}%
\pgfpathrectangle{\pgfqpoint{0.439176in}{0.313488in}}{\pgfqpoint{3.662794in}{1.936512in}}%
\pgfusepath{clip}%
\pgfsetbuttcap%
\pgfsetmiterjoin%
\definecolor{currentfill}{rgb}{1.000000,0.756863,0.529412}%
\pgfsetfillcolor{currentfill}%
\pgfsetlinewidth{1.505625pt}%
\definecolor{currentstroke}{rgb}{0.921569,0.505882,0.105882}%
\pgfsetstrokecolor{currentstroke}%
\pgfsetdash{}{0pt}%
\pgfpathmoveto{\pgfqpoint{0.605667in}{0.313488in}}%
\pgfpathlineto{\pgfqpoint{0.754585in}{0.313488in}}%
\pgfpathlineto{\pgfqpoint{0.754585in}{0.524744in}}%
\pgfpathlineto{\pgfqpoint{0.605667in}{0.524744in}}%
\pgfpathlineto{\pgfqpoint{0.605667in}{0.313488in}}%
\pgfpathclose%
\pgfusepath{stroke,fill}%
\end{pgfscope}%
\begin{pgfscope}%
\pgfpathrectangle{\pgfqpoint{0.439176in}{0.313488in}}{\pgfqpoint{3.662794in}{1.936512in}}%
\pgfusepath{clip}%
\pgfsetbuttcap%
\pgfsetmiterjoin%
\definecolor{currentfill}{rgb}{1.000000,0.756863,0.529412}%
\pgfsetfillcolor{currentfill}%
\pgfsetlinewidth{1.505625pt}%
\definecolor{currentstroke}{rgb}{0.921569,0.505882,0.105882}%
\pgfsetstrokecolor{currentstroke}%
\pgfsetdash{}{0pt}%
\pgfpathmoveto{\pgfqpoint{1.201340in}{0.313488in}}%
\pgfpathlineto{\pgfqpoint{1.350258in}{0.313488in}}%
\pgfpathlineto{\pgfqpoint{1.350258in}{0.665581in}}%
\pgfpathlineto{\pgfqpoint{1.201340in}{0.665581in}}%
\pgfpathlineto{\pgfqpoint{1.201340in}{0.313488in}}%
\pgfpathclose%
\pgfusepath{stroke,fill}%
\end{pgfscope}%
\begin{pgfscope}%
\pgfpathrectangle{\pgfqpoint{0.439176in}{0.313488in}}{\pgfqpoint{3.662794in}{1.936512in}}%
\pgfusepath{clip}%
\pgfsetbuttcap%
\pgfsetmiterjoin%
\definecolor{currentfill}{rgb}{1.000000,0.756863,0.529412}%
\pgfsetfillcolor{currentfill}%
\pgfsetlinewidth{1.505625pt}%
\definecolor{currentstroke}{rgb}{0.921569,0.505882,0.105882}%
\pgfsetstrokecolor{currentstroke}%
\pgfsetdash{}{0pt}%
\pgfpathmoveto{\pgfqpoint{1.797013in}{0.313488in}}%
\pgfpathlineto{\pgfqpoint{1.945931in}{0.313488in}}%
\pgfpathlineto{\pgfqpoint{1.945931in}{1.281744in}}%
\pgfpathlineto{\pgfqpoint{1.797013in}{1.281744in}}%
\pgfpathlineto{\pgfqpoint{1.797013in}{0.313488in}}%
\pgfpathclose%
\pgfusepath{stroke,fill}%
\end{pgfscope}%
\begin{pgfscope}%
\pgfpathrectangle{\pgfqpoint{0.439176in}{0.313488in}}{\pgfqpoint{3.662794in}{1.936512in}}%
\pgfusepath{clip}%
\pgfsetbuttcap%
\pgfsetmiterjoin%
\definecolor{currentfill}{rgb}{1.000000,0.756863,0.529412}%
\pgfsetfillcolor{currentfill}%
\pgfsetlinewidth{1.505625pt}%
\definecolor{currentstroke}{rgb}{0.921569,0.505882,0.105882}%
\pgfsetstrokecolor{currentstroke}%
\pgfsetdash{}{0pt}%
\pgfpathmoveto{\pgfqpoint{2.392686in}{0.313488in}}%
\pgfpathlineto{\pgfqpoint{2.541604in}{0.313488in}}%
\pgfpathlineto{\pgfqpoint{2.541604in}{0.489534in}}%
\pgfpathlineto{\pgfqpoint{2.392686in}{0.489534in}}%
\pgfpathlineto{\pgfqpoint{2.392686in}{0.313488in}}%
\pgfpathclose%
\pgfusepath{stroke,fill}%
\end{pgfscope}%
\begin{pgfscope}%
\pgfpathrectangle{\pgfqpoint{0.439176in}{0.313488in}}{\pgfqpoint{3.662794in}{1.936512in}}%
\pgfusepath{clip}%
\pgfsetbuttcap%
\pgfsetmiterjoin%
\definecolor{currentfill}{rgb}{1.000000,0.756863,0.529412}%
\pgfsetfillcolor{currentfill}%
\pgfsetlinewidth{1.505625pt}%
\definecolor{currentstroke}{rgb}{0.921569,0.505882,0.105882}%
\pgfsetstrokecolor{currentstroke}%
\pgfsetdash{}{0pt}%
\pgfpathmoveto{\pgfqpoint{2.988359in}{0.313488in}}%
\pgfpathlineto{\pgfqpoint{3.137277in}{0.313488in}}%
\pgfpathlineto{\pgfqpoint{3.137277in}{0.348697in}}%
\pgfpathlineto{\pgfqpoint{2.988359in}{0.348697in}}%
\pgfpathlineto{\pgfqpoint{2.988359in}{0.313488in}}%
\pgfpathclose%
\pgfusepath{stroke,fill}%
\end{pgfscope}%
\begin{pgfscope}%
\pgfpathrectangle{\pgfqpoint{0.439176in}{0.313488in}}{\pgfqpoint{3.662794in}{1.936512in}}%
\pgfusepath{clip}%
\pgfsetbuttcap%
\pgfsetmiterjoin%
\definecolor{currentfill}{rgb}{1.000000,0.756863,0.529412}%
\pgfsetfillcolor{currentfill}%
\pgfsetlinewidth{1.505625pt}%
\definecolor{currentstroke}{rgb}{0.921569,0.505882,0.105882}%
\pgfsetstrokecolor{currentstroke}%
\pgfsetdash{}{0pt}%
\pgfpathmoveto{\pgfqpoint{3.584032in}{0.313488in}}%
\pgfpathlineto{\pgfqpoint{3.732951in}{0.313488in}}%
\pgfpathlineto{\pgfqpoint{3.732951in}{0.327572in}}%
\pgfpathlineto{\pgfqpoint{3.584032in}{0.327572in}}%
\pgfpathlineto{\pgfqpoint{3.584032in}{0.313488in}}%
\pgfpathclose%
\pgfusepath{stroke,fill}%
\end{pgfscope}%
\begin{pgfscope}%
\pgfsetrectcap%
\pgfsetmiterjoin%
\pgfsetlinewidth{1.003750pt}%
\definecolor{currentstroke}{rgb}{0.917647,0.917647,0.949020}%
\pgfsetstrokecolor{currentstroke}%
\pgfsetdash{}{0pt}%
\pgfpathmoveto{\pgfqpoint{0.439176in}{0.313488in}}%
\pgfpathlineto{\pgfqpoint{0.439176in}{2.250000in}}%
\pgfusepath{stroke}%
\end{pgfscope}%
\begin{pgfscope}%
\pgfsetrectcap%
\pgfsetmiterjoin%
\pgfsetlinewidth{1.003750pt}%
\definecolor{currentstroke}{rgb}{0.917647,0.917647,0.949020}%
\pgfsetstrokecolor{currentstroke}%
\pgfsetdash{}{0pt}%
\pgfpathmoveto{\pgfqpoint{4.101970in}{0.313488in}}%
\pgfpathlineto{\pgfqpoint{4.101970in}{2.250000in}}%
\pgfusepath{stroke}%
\end{pgfscope}%
\begin{pgfscope}%
\pgfsetrectcap%
\pgfsetmiterjoin%
\pgfsetlinewidth{1.003750pt}%
\definecolor{currentstroke}{rgb}{0.917647,0.917647,0.949020}%
\pgfsetstrokecolor{currentstroke}%
\pgfsetdash{}{0pt}%
\pgfpathmoveto{\pgfqpoint{0.439176in}{0.313488in}}%
\pgfpathlineto{\pgfqpoint{4.101970in}{0.313488in}}%
\pgfusepath{stroke}%
\end{pgfscope}%
\begin{pgfscope}%
\pgfsetrectcap%
\pgfsetmiterjoin%
\pgfsetlinewidth{1.003750pt}%
\definecolor{currentstroke}{rgb}{0.917647,0.917647,0.949020}%
\pgfsetstrokecolor{currentstroke}%
\pgfsetdash{}{0pt}%
\pgfpathmoveto{\pgfqpoint{0.439176in}{2.250000in}}%
\pgfpathlineto{\pgfqpoint{4.101970in}{2.250000in}}%
\pgfusepath{stroke}%
\end{pgfscope}%
\begin{pgfscope}%
\pgfsetbuttcap%
\pgfsetmiterjoin%
\definecolor{currentfill}{rgb}{1.000000,0.756863,0.529412}%
\pgfsetfillcolor{currentfill}%
\pgfsetlinewidth{1.505625pt}%
\definecolor{currentstroke}{rgb}{0.921569,0.505882,0.105882}%
\pgfsetstrokecolor{currentstroke}%
\pgfsetdash{}{0pt}%
\pgfpathmoveto{\pgfqpoint{2.932915in}{2.019477in}}%
\pgfpathlineto{\pgfqpoint{3.210693in}{2.019477in}}%
\pgfpathlineto{\pgfqpoint{3.210693in}{2.116699in}}%
\pgfpathlineto{\pgfqpoint{2.932915in}{2.116699in}}%
\pgfpathlineto{\pgfqpoint{2.932915in}{2.019477in}}%
\pgfpathclose%
\pgfusepath{stroke,fill}%
\end{pgfscope}%
\begin{pgfscope}%
\definecolor{textcolor}{rgb}{0.137255,0.215686,0.231373}%
\pgfsetstrokecolor{textcolor}%
\pgfsetfillcolor{textcolor}%
\pgftext[x=3.321804in,y=2.019477in,left,base]{\color{textcolor}\sffamily\fontsize{10.000000}{12.000000}\selectfont \(\displaystyle \psi_\theta\,\left(x\right)\)}%
\end{pgfscope}%
\begin{pgfscope}%
\pgfsetbuttcap%
\pgfsetmiterjoin%
\definecolor{currentfill}{rgb}{0.623529,0.780392,1.000000}%
\pgfsetfillcolor{currentfill}%
\pgfsetlinewidth{1.505625pt}%
\definecolor{currentstroke}{rgb}{0.298039,0.447059,0.690196}%
\pgfsetstrokecolor{currentstroke}%
\pgfsetdash{}{0pt}%
\pgfpathmoveto{\pgfqpoint{2.932915in}{1.809041in}}%
\pgfpathlineto{\pgfqpoint{3.210693in}{1.809041in}}%
\pgfpathlineto{\pgfqpoint{3.210693in}{1.906263in}}%
\pgfpathlineto{\pgfqpoint{2.932915in}{1.906263in}}%
\pgfpathlineto{\pgfqpoint{2.932915in}{1.809041in}}%
\pgfpathclose%
\pgfusepath{stroke,fill}%
\end{pgfscope}%
\begin{pgfscope}%
\definecolor{textcolor}{rgb}{0.137255,0.215686,0.231373}%
\pgfsetstrokecolor{textcolor}%
\pgfsetfillcolor{textcolor}%
\pgftext[x=3.321804in,y=1.809041in,left,base]{\color{textcolor}\sffamily\fontsize{10.000000}{12.000000}\selectfont \(\displaystyle \phi\,\left(\texttt{apple}\right)\)}%
\end{pgfscope}%
\end{pgfpicture}%
\makeatother%
\endgroup%

  }
 
  \note[item]{Il risulato della softmax è una distruibuzione di probabilità in
    cui non è particolarmente evidente la classe di partenza e ciò ostacola il
    training.}
  \note[item]{NEXT}
  \note[item]{Per ovviare a questo problema possiamo introdurre un
    iperparamentro che riscalando le similarià procude un encoding che risulta
    più efficiace nel training.}
  \note[item]{NEXT}
  \note[item]{Come prima partiamo da una inizializzazione casuale del modello e
    usiamo la cross entropy come funzione di loss.}
  \note[item]{NEXT}
  \note[item]{Stavolta, tutte le componenti di $\psi$ contribuiscono e, agendo
    i paramentri $\theta$, sono aumentate o ridotte per allinersi all'encoding.}
\end{frame}

\begin{frame}{Hierarchical Encondings}
  \begin{figure}
    \centering
    \includegraphics[width=.8\linewidth]{figures/paper/title.pdf}
  \end{figure}
  \begin{figure}
    \centering
    \includegraphics[width=.8\linewidth]{figures/paper/errors.pdf}
  \end{figure}
  \begin{figure}
    \centering
    \includegraphics[width=.8\linewidth]{figures/paper/features.pdf}
  \end{figure}

  \note{\scriptsize{Questa tecnica per sfruttare le reazioni tra classi usando un
    esplicita gerarchia, è stata già descritta in alcune pubblicazioni.
    Una di queste è un articolo che sarà presentato a xAI 2023.}}
  \note[item]{Abbiamo applicato una simile tecnica per produrre un embedding
    gerarchico e confrontato le performance dei modelli con quelle ottenute
    che sfruttano e non la gerachica tra le classi.}
  \note[item]{Per performance intendo la quatità di errori, asse orizzontale, e
    la qualità degli errori, asse verticale, a diversi livelli della gerarchica.
    Questi plots saranno ripresi in seguito.}
  \note[item]{Abbiamo inoltre studiato la disposizione spaziale dei features
    vectors, ovvero come sono organizzati nello spazio i vettori dal penultimo
    livello del modello associati alle varie immaginni ottenuti}
  \note[item]{Proiettarli nel piano dà un'indicazione su quali immagini il
    modello pensa siano simili. Ciò che ci attendiamo è la comparsa di cluster
    relativi alle varie classi.}
  \note[item]{Abbiamo usato delle metriche di clustering per quantificare il
    raggruppamento dei features vectors e capire quali modelli producevano una
    rappresentazione interna più strutturata.}
  \note[item]{Un limite di tale approccio è la necessità di avere o poter
    costruire una gerarchica tra le classi. Uno degli approcci presentati supera
    tale limite ricavando gli encoding dai word embeddings delle classi. Ma ciò
    introduce un altro problema, non tutte le parole hanno un'embedding e una
    parola può avere diversi un significati.}
  \note[item]{Da qui l'idea per i descriptions encodings.}
\end{frame}


  \section{Description Encodings}
  \begin{frame}{Description Encodings}
  Recipe:
  \begin{enumerate}
    \item Use Language Model (e.g. chatGPT) to \alert{generate descriptions}\\
      for each class.
    \item Use Embedding Model (e.g. BERT-like) to \alert{generate embeddings}\\
      for each description.
    \item Use Dimensionality Reduction (e.g. t-SNE) to \alert{generate encodings}\\
      for each embedding.
  \end{enumerate}

    \note[item]{La ricetta per generarli è la seguente.}
  \note[item]{"Uno": usare modelli generativi di testo per produrre
    delle descrizioni delle classi.}
  \note[item]{"Due": usate modelli di embeddings che convertano le descrizioni
    testuali in un vettore di numeri reali detto embeddings. Descrizioni simili
    avranno embeddings simili.}
  \note[item]{"Tre": usare un algoritmo di riduzione della dimensionalità
    mappare gli embeddings in uno spazio di dimensionalità inferiore. Inoltre
    agendo sugli iperparametri di tali algoritmi e possibile modificare la
    distiribuzione spaziale, preservando le relazioni di similarità.}
  \note[item]{Il risulato sono i Description Encodings.}
\end{frame}

\begin{frame}{Description Encodings}
  \begin{itemize}
    \item \texttt{lemon}: \emph{\small``Lemons are \alert{oval-shaped} fruits
      known for their \alert{bright yellow} color and acidic juice.''}
    \item \texttt{pear}: \emph{\small``Pears are fruits a with \alert{rounded
      bottom} and a narrower, \alert{elongated top}.''}
    \item \texttt{apple}: \emph{\small``Apples are \alert{round} fruits that
      come in a variety of colors, including \alert{red}, \alert{green}, and
      \alert{yellow}.''}
  \end{itemize}
  \note[item]{Queste sono esempi di parti di descrizioni di alcune delle
    classi ottenute da un modello generativo.}
  \note[item]{I limoni sono frutti ovali di un giallo brillante, le pere
    hanno la base rotonda e la parte superiore allungata, le mele sono
    tondeggianti rosse, verdi e gialle. Tutte carattiristiche che aiutano a
    distingure visivamente le diverse classi.}
\end{frame}

\begin{frame}{Description Encodings}
  \begin{columns}
    \column{0.65\textwidth}
    \tikzfig{figures/example_embeddings}

    \column{0.35\textwidth}
    \textbf{\textsc{Desideratum}}\\
    \vspace{0.2cm}
    \emph{Distinctive features in the descriptions translate to
    a hierarchy-like structure in the encodings space}
  \end{columns}

  \note[item]{Dalle descrizioni passiamo agli embedding e da questi agli
    encoding. Questa è una rappresentazione 2D semplificata degli encoding.}
  \note[item]{La speranza è che le caratteristiche distintive presenti nelle
    descrizioni si traducano in una struttura gerarchica nello spazio degli
    encoding. Ciò è una conferma che i descriptions encoding riescono a
    ricavare una gerarchia tra le classi dal significato delle descrizioni.}
  \note[item]{Tuttavia gerarchia e similarità tra classi sono concetti
    soggettivi e non necessariamente collegati.}
  \note[item]{Come misura di similarità tra due encodings utilizziamo l'angolo
    tra i due vettori. Mela e pera sono più simili di mela e cane.}
\end{frame}

\begin{frame}{Description Encodings}
  \begin{columns}
    \column{0.65\textwidth}
    \only<1>{\tikzfig{figures/example_embeddings_apple1}}
    \only<2>{\tikzfig{figures/example_embeddings_apple2}}

    \column{0.35\textwidth}
    Align $\psi_\theta$ to $\phi$ with
    \alert{Cosine Distance}.
    \begin{align*}
      \phi &:
      \mathcal{C} \rightarrow \mathbb{R}^{d} \\
      \psi_\theta &:
      \mathcal{X} \rightarrow \mathbb{R}^{d}
    \end{align*}
    \begin{equation*}
      \mathcal{L} := 1 -
      \frac{
        \psi_\theta \cdot \phi
      }{
        \left|\psi_\theta\right|_2
        \left|\phi\right|_2
      }
    \end{equation*}
  \end{columns}

  \note[item]{Consideriamo l'encoding per mela (in blu) e l'output del
    modello non addestrato (arancione).}
  \note[item]{Ora $\phi$ mappa dall'inisieme delle classi a $\mathbb{R}^d$ dove
    la dimensione $d$ e stabilita dall'algoritmo di riduzione di
    dimensionalità. $\mathbb{R}^d$ è anche lo spazio degli output del modello
    che non sono più distribuzioni di probabilità ma semplici vettori con
    componeti reali.}
  \note[item]{Il training consiste nel modificare i parametri $\theta$ in modo
    tale da allineare l'output all'encoding, quindi minimizzare l'angolo compreso.}
  \note[item]{NEXT}
  \note[item]{In questo caso la funzione di loss è la Cosine Distance.}
  \note[item]{La classe predetta dal modello sarà quindi quella associata al
    vettore di encoding più vicino all'output del modello.}
\end{frame}

\begin{frame}{Encondings Comparison}
  \begin{block}{\textsc{One-hot Encoding + Cross Entropy Loss}}
    \begin{itemize}
      \item[\cmark] Battle tested
      \item[\xmark] Ignore classes similarities
    \end{itemize}
  \end{block}
  \begin{block}{\textsc{Hierarchical Encoding + Cross Entropy Loss}}
    \begin{itemize}
      \item[\cmark] Exploit classes similarities
      \item[\xmark] Require a hierarchy
    \end{itemize}
  \end{block}
  \begin{block}{\textsc{Description Encoding + Cosine Distance Loss}}
    \begin{itemize}
      \item[\cmark] Exploit classes similarities
      \item[\cmark] Do not require a hierarchy
    \end{itemize}
  \end{block}

  \note[item]{Riassumendo. One-hot encoding è largamente utilizzato e dà buoni
  risulati. È semplice da implementare e non necessita di informazione extra.
  Tuttavia non tiene conto delle relazioni tra le cassi con la conseguenza che
  scambiare una mela con una pera è equivalente a scambiare una mela con un
  automobile.}
  \note[item]{Hierarchical encoding è una codifica che sfrutta le relazioni tra
  classi basandosi su un'esplicita gerarchia. È quindi applicaile a quei dataset
  per i quali esite o è possibile costruire una gerarchia. Sperimentalmente si
  verifica che è sensibile agli iperparametri conivolti e alla scelta delle
  funzioni che costruiscono le distanze e le probabilità.}
  \note[item]{Infine i modelli che fanno uso di Description Encodings sfruttano
  la relazioni tra classi senza richedere una gerarchica. Costruiscono tali
  relazioni basandosi le conoscenze semantiche apprese dai modelli generativi e
  di embeddings.}
\end{frame}


  \section{Experiments}
  \begin{frame}{Experiments | Datasets}
  \begin{columns}
    \begin{column}{0.35\textwidth}
      \alert{{\large CIFAR-100}}
      \begin{itemize}
        \item 100 classes
        \item 6 hierarchy levels
      \end{itemize}
    \end{column}
    \begin{column}{0.2\textwidth}
      \begin{figure}
        \centering
        \includegraphics[width=.7\linewidth]{figures/CIFAR100/example_1.jpg}
        \captionsetup{labelformat=empty, justification=centering, font=scriptsize}
        \caption{\emph{Apple}}
      \end{figure}
    \end{column}
    \begin{column}{0.2\textwidth}
      \begin{figure}
        \centering
        \includegraphics[width=.7\linewidth]{figures/CIFAR100/example_2.jpg}
        \captionsetup{labelformat=empty, justification=centering, font=scriptsize}
        \caption{\emph{Chair}}
      \end{figure}
    \end{column}
    \begin{column}{0.2\textwidth}
      \begin{figure}
        \centering
        \includegraphics[width=.7\linewidth]{figures/CIFAR100/example_3.jpg}
        \captionsetup{labelformat=empty, justification=centering, font=scriptsize}
        \caption{\emph{Lobster}}
      \end{figure}
    \end{column}
  \end{columns}

  \rule{\linewidth}{0.4pt}

  \begin{columns}
    \begin{column}{0.35\textwidth}
      \alert{{\large iNaturalist19}}
      \begin{itemize}
        \item 1010 classes
        \item 8 hierarchy levels
      \end{itemize}
    \end{column}
    \begin{column}{0.2\textwidth}
      \begin{figure}
        \centering
        \includegraphics[width=.7\linewidth]{figures/iNaturalist19/example_1.jpg}
        \captionsetup{labelformat=empty, justification=centering, font=scriptsize}
        \caption{\emph{Amanita Muscaria}}
      \end{figure}
    \end{column}
    \begin{column}{0.2\textwidth}
      \begin{figure}
        \centering
        \includegraphics[width=.7\linewidth]{figures/iNaturalist19/example_2.jpg}
        \captionsetup{labelformat=empty, justification=centering, font=scriptsize}
        \caption{\emph{Junonia Orithya}}
      \end{figure}
    \end{column}
    \begin{column}{0.2\textwidth}
      \begin{figure}
        \centering
        \includegraphics[width=.7\linewidth]{figures/iNaturalist19/example_3.jpg}
        \captionsetup{labelformat=empty, justification=centering, font=scriptsize}
        \caption{\emph{Tringa Ochropus}}
      \end{figure}
    \end{column}
  \end{columns}

  \rule{\linewidth}{0.4pt}

  \begin{columns}
    \begin{column}{0.35\textwidth}
      \alert{{\large tieredImageNet}}
      \begin{itemize}
        \item 608 classes
        \item 13 hierarchy levels
      \end{itemize}
    \end{column}
    \begin{column}{0.2\textwidth}
      \begin{figure}
        \centering
        \includegraphics[width=.7\linewidth]{figures/tieredImageNet/example_1.jpg}
        \captionsetup{labelformat=empty, justification=centering, font=scriptsize}
        \caption{\emph{Hammerhead}}
      \end{figure}
    \end{column}
    \begin{column}{0.2\textwidth}
      \begin{figure}
        \centering
        \includegraphics[width=.7\linewidth]{figures/tieredImageNet/example_2.jpg}
        \captionsetup{labelformat=empty, justification=centering, font=scriptsize}
        \caption{\emph{Basketball}}
      \end{figure}
    \end{column}
    \begin{column}{0.2\textwidth}
      \begin{figure}
        \centering
        \includegraphics[width=.7\linewidth]{figures/tieredImageNet/example_3.jpg}
        \captionsetup{labelformat=empty, justification=centering, font=scriptsize}
        \caption{\emph{Cliff}}
      \end{figure}
    \end{column}
  \end{columns}

  \note[item]{I datasets utilizzati negli esperimenti sono CIFAR-100,
    iNaturalist19, tieredImageNet.}
  \note[item]{CIFAR-100 è un dataset di immagini a bassa risoluzione, 100
    classi per le quali è definita una gerarchia a 6 livelli.}
  \note[item]{iNaturaluralist sono foto di soggetti nautarali e le etichette
    associate sono il nome scientifico. 1010 classi e gerarchia a 8 livelli}
  \note[item]{tieredImageNet è un sotto insieme di ImageNet, 608 classi e 13
    livelli di gerarchia.}
\end{frame}


\begin{frame}{Experiments | Metrics}
  Let $\bm{M}$ be the \alert{confusion matrix} \\
  and $\bm{D}$ the \alert{distances matrix} previously introduced.
  \vspace{0.5cm}
  \begin{spreadlines}{15pt}
  \begin{align*}
    \text{Error Rate} &:=
      \frac{\sum \bm{M} - \text{tr} \, \bm{M}}{\sum \bm{M}} \\
    \text{Hierarchical Distance} &:=
      \frac{\sum \bm{M} \odot \bm{D}}{\sum \bm{M}} \\
    \text{Hierarchical Distance Mistake} &:=
      \frac{\sum \bm{M} \odot \bm{D}}{\sum \bm{M} - \text{tr} \, \bm{M}}
  \end{align*}
  \end{spreadlines}

  \note[Item]{Per misurare le performance del modello abbiamo fatto riscorso a
  3 metriche: Error Rate, Hiearchical Distance e Hiearchical Distance Mistake.}
  \note[item]{Queste possono essere definite a partire dalla matrice di
  confusione M e dalla matrice delle distaze D (tipo quella precedentemente
  introdotta).}
  \note[item]{L'Error Rate Il numero di errori commessi diviso il numero di
  esempi nel dataset di Test.}
  \note[item]{Hierarchical Distace la si ottiene moltiplicando M con D elemento
  per elemento, sommando le entrate della matrice risultante e dividendo per il
  totale degli esempi.}
  \note[item]{In Hierarchical distance Mistake nel denominatore è il numero di errori.}
  \note[item]{Error Rate è un indicatore di quanti errori il modello commette,
  mentre Hieararchical Distance Mistake quantifica la gravità di tali errori
  utilizzando come pesi le distanze nella gerarchia.}
\end{frame}


\begin{frame}{Experiments | Results}
  %% Creator: Matplotlib, PGF backend
%%
%% To include the figure in your LaTeX document, write
%%   \input{<filename>.pgf}
%%
%% Make sure the required packages are loaded in your preamble
%%   \usepackage{pgf}
%%
%% Also ensure that all the required font packages are loaded; for instance,
%% the lmodern package is sometimes necessary when using math font.
%%   \usepackage{lmodern}
%%
%% Figures using additional raster images can only be included by \input if
%% they are in the same directory as the main LaTeX file. For loading figures
%% from other directories you can use the `import` package
%%   \usepackage{import}
%%
%% and then include the figures with
%%   \import{<path to file>}{<filename>.pgf}
%%
%% Matplotlib used the following preamble
%%   
%%   \usepackage{fontspec}
%%   \setmainfont{DejaVuSerif.ttf}[Path=\detokenize{/Users/simo/.local/share/virtualenvs/master-thesis-code/lib/python3.10/site-packages/matplotlib/mpl-data/fonts/ttf/}]
%%   \setsansfont{DejaVuSans.ttf}[Path=\detokenize{/Users/simo/.local/share/virtualenvs/master-thesis-code/lib/python3.10/site-packages/matplotlib/mpl-data/fonts/ttf/}]
%%   \setmonofont{DejaVuSansMono.ttf}[Path=\detokenize{/Users/simo/.local/share/virtualenvs/master-thesis-code/lib/python3.10/site-packages/matplotlib/mpl-data/fonts/ttf/}]
%%   \makeatletter\@ifpackageloaded{underscore}{}{\usepackage[strings]{underscore}}\makeatother
%%
\begingroup%
\makeatletter%
\begin{pgfpicture}%
\pgfpathrectangle{\pgfpointorigin}{\pgfqpoint{4.251970in}{2.627862in}}%
\pgfusepath{use as bounding box, clip}%
\begin{pgfscope}%
\pgfsetbuttcap%
\pgfsetmiterjoin%
\definecolor{currentfill}{rgb}{0.980392,0.980392,0.980392}%
\pgfsetfillcolor{currentfill}%
\pgfsetlinewidth{0.000000pt}%
\definecolor{currentstroke}{rgb}{1.000000,1.000000,1.000000}%
\pgfsetstrokecolor{currentstroke}%
\pgfsetdash{}{0pt}%
\pgfpathmoveto{\pgfqpoint{0.000000in}{0.000000in}}%
\pgfpathlineto{\pgfqpoint{4.251970in}{0.000000in}}%
\pgfpathlineto{\pgfqpoint{4.251970in}{2.627862in}}%
\pgfpathlineto{\pgfqpoint{0.000000in}{2.627862in}}%
\pgfpathlineto{\pgfqpoint{0.000000in}{0.000000in}}%
\pgfpathclose%
\pgfusepath{fill}%
\end{pgfscope}%
\begin{pgfscope}%
\pgfsetbuttcap%
\pgfsetmiterjoin%
\definecolor{currentfill}{rgb}{0.917647,0.917647,0.949020}%
\pgfsetfillcolor{currentfill}%
\pgfsetlinewidth{0.000000pt}%
\definecolor{currentstroke}{rgb}{0.000000,0.000000,0.000000}%
\pgfsetstrokecolor{currentstroke}%
\pgfsetstrokeopacity{0.000000}%
\pgfsetdash{}{0pt}%
\pgfpathmoveto{\pgfqpoint{0.956688in}{0.387285in}}%
\pgfpathlineto{\pgfqpoint{2.790353in}{0.387285in}}%
\pgfpathlineto{\pgfqpoint{2.790353in}{2.390228in}}%
\pgfpathlineto{\pgfqpoint{0.956688in}{2.390228in}}%
\pgfpathlineto{\pgfqpoint{0.956688in}{0.387285in}}%
\pgfpathclose%
\pgfusepath{fill}%
\end{pgfscope}%
\begin{pgfscope}%
\definecolor{textcolor}{rgb}{0.137255,0.215686,0.231373}%
\pgfsetstrokecolor{textcolor}%
\pgfsetfillcolor{textcolor}%
\pgftext[x=1.160428in,y=0.248396in,,top]{\color{textcolor}\sffamily\fontsize{8.000000}{9.600000}\selectfont 27.0}%
\end{pgfscope}%
\begin{pgfscope}%
\definecolor{textcolor}{rgb}{0.137255,0.215686,0.231373}%
\pgfsetstrokecolor{textcolor}%
\pgfsetfillcolor{textcolor}%
\pgftext[x=1.466039in,y=0.248396in,,top]{\color{textcolor}\sffamily\fontsize{8.000000}{9.600000}\selectfont 27.3}%
\end{pgfscope}%
\begin{pgfscope}%
\definecolor{textcolor}{rgb}{0.137255,0.215686,0.231373}%
\pgfsetstrokecolor{textcolor}%
\pgfsetfillcolor{textcolor}%
\pgftext[x=1.771650in,y=0.248396in,,top]{\color{textcolor}\sffamily\fontsize{8.000000}{9.600000}\selectfont 27.6}%
\end{pgfscope}%
\begin{pgfscope}%
\definecolor{textcolor}{rgb}{0.137255,0.215686,0.231373}%
\pgfsetstrokecolor{textcolor}%
\pgfsetfillcolor{textcolor}%
\pgftext[x=2.077261in,y=0.248396in,,top]{\color{textcolor}\sffamily\fontsize{8.000000}{9.600000}\selectfont 27.9}%
\end{pgfscope}%
\begin{pgfscope}%
\definecolor{textcolor}{rgb}{0.137255,0.215686,0.231373}%
\pgfsetstrokecolor{textcolor}%
\pgfsetfillcolor{textcolor}%
\pgftext[x=2.382872in,y=0.248396in,,top]{\color{textcolor}\sffamily\fontsize{8.000000}{9.600000}\selectfont 28.2}%
\end{pgfscope}%
\begin{pgfscope}%
\definecolor{textcolor}{rgb}{0.137255,0.215686,0.231373}%
\pgfsetstrokecolor{textcolor}%
\pgfsetfillcolor{textcolor}%
\pgftext[x=2.790353in,y=0.248396in,,top]{\color{textcolor}\sffamily\fontsize{8.000000}{9.600000}\selectfont Error \%}%
\end{pgfscope}%
\begin{pgfscope}%
\definecolor{textcolor}{rgb}{0.137255,0.215686,0.231373}%
\pgfsetstrokecolor{textcolor}%
\pgfsetfillcolor{textcolor}%
\pgftext[x=0.570403in, y=0.499149in, left, base]{\color{textcolor}\sffamily\fontsize{8.000000}{9.600000}\selectfont 2.26}%
\end{pgfscope}%
\begin{pgfscope}%
\definecolor{textcolor}{rgb}{0.137255,0.215686,0.231373}%
\pgfsetstrokecolor{textcolor}%
\pgfsetfillcolor{textcolor}%
\pgftext[x=0.570403in, y=0.807294in, left, base]{\color{textcolor}\sffamily\fontsize{8.000000}{9.600000}\selectfont 2.30}%
\end{pgfscope}%
\begin{pgfscope}%
\definecolor{textcolor}{rgb}{0.137255,0.215686,0.231373}%
\pgfsetstrokecolor{textcolor}%
\pgfsetfillcolor{textcolor}%
\pgftext[x=0.570403in, y=1.115439in, left, base]{\color{textcolor}\sffamily\fontsize{8.000000}{9.600000}\selectfont 2.34}%
\end{pgfscope}%
\begin{pgfscope}%
\definecolor{textcolor}{rgb}{0.137255,0.215686,0.231373}%
\pgfsetstrokecolor{textcolor}%
\pgfsetfillcolor{textcolor}%
\pgftext[x=0.570403in, y=1.423583in, left, base]{\color{textcolor}\sffamily\fontsize{8.000000}{9.600000}\selectfont 2.38}%
\end{pgfscope}%
\begin{pgfscope}%
\definecolor{textcolor}{rgb}{0.137255,0.215686,0.231373}%
\pgfsetstrokecolor{textcolor}%
\pgfsetfillcolor{textcolor}%
\pgftext[x=0.570403in, y=1.731728in, left, base]{\color{textcolor}\sffamily\fontsize{8.000000}{9.600000}\selectfont 2.42}%
\end{pgfscope}%
\begin{pgfscope}%
\definecolor{textcolor}{rgb}{0.137255,0.215686,0.231373}%
\pgfsetstrokecolor{textcolor}%
\pgfsetfillcolor{textcolor}%
\pgftext[x=0.570403in, y=2.039873in, left, base]{\color{textcolor}\sffamily\fontsize{8.000000}{9.600000}\selectfont 2.46}%
\end{pgfscope}%
\begin{pgfscope}%
\definecolor{textcolor}{rgb}{0.137255,0.215686,0.231373}%
\pgfsetstrokecolor{textcolor}%
\pgfsetfillcolor{textcolor}%
\pgftext[x=0.155581in, y=2.401783in, left, base]{\color{textcolor}\sffamily\fontsize{8.000000}{9.600000}\selectfont Hierarchical}%
\end{pgfscope}%
\begin{pgfscope}%
\definecolor{textcolor}{rgb}{0.137255,0.215686,0.231373}%
\pgfsetstrokecolor{textcolor}%
\pgfsetfillcolor{textcolor}%
\pgftext[x=0.095142in, y=2.277370in, left, base]{\color{textcolor}\sffamily\fontsize{8.000000}{9.600000}\selectfont dist. mistake}%
\end{pgfscope}%
\begin{pgfscope}%
\pgfpathrectangle{\pgfqpoint{0.956688in}{0.387285in}}{\pgfqpoint{1.833666in}{2.002942in}}%
\pgfusepath{clip}%
\pgfsetbuttcap%
\pgfsetroundjoin%
\pgfsetlinewidth{0.752812pt}%
\definecolor{currentstroke}{rgb}{0.298039,0.447059,0.690196}%
\pgfsetstrokecolor{currentstroke}%
\pgfsetdash{}{0pt}%
\pgfpathmoveto{\pgfqpoint{1.807951in}{1.911346in}}%
\pgfpathlineto{\pgfqpoint{2.660332in}{1.911346in}}%
\pgfusepath{stroke}%
\end{pgfscope}%
\begin{pgfscope}%
\pgfpathrectangle{\pgfqpoint{0.956688in}{0.387285in}}{\pgfqpoint{1.833666in}{2.002942in}}%
\pgfusepath{clip}%
\pgfsetbuttcap%
\pgfsetroundjoin%
\pgfsetlinewidth{0.752812pt}%
\definecolor{currentstroke}{rgb}{0.298039,0.447059,0.690196}%
\pgfsetstrokecolor{currentstroke}%
\pgfsetdash{}{0pt}%
\pgfpathmoveto{\pgfqpoint{2.234141in}{1.789100in}}%
\pgfpathlineto{\pgfqpoint{2.234141in}{2.033592in}}%
\pgfusepath{stroke}%
\end{pgfscope}%
\begin{pgfscope}%
\pgfpathrectangle{\pgfqpoint{0.956688in}{0.387285in}}{\pgfqpoint{1.833666in}{2.002942in}}%
\pgfusepath{clip}%
\pgfsetbuttcap%
\pgfsetroundjoin%
\pgfsetlinewidth{0.752812pt}%
\definecolor{currentstroke}{rgb}{0.921569,0.505882,0.105882}%
\pgfsetstrokecolor{currentstroke}%
\pgfsetdash{}{0pt}%
\pgfpathmoveto{\pgfqpoint{1.609789in}{0.685162in}}%
\pgfpathlineto{\pgfqpoint{2.263571in}{0.685162in}}%
\pgfusepath{stroke}%
\end{pgfscope}%
\begin{pgfscope}%
\pgfpathrectangle{\pgfqpoint{0.956688in}{0.387285in}}{\pgfqpoint{1.833666in}{2.002942in}}%
\pgfusepath{clip}%
\pgfsetbuttcap%
\pgfsetroundjoin%
\pgfsetlinewidth{0.752812pt}%
\definecolor{currentstroke}{rgb}{0.921569,0.505882,0.105882}%
\pgfsetstrokecolor{currentstroke}%
\pgfsetdash{}{0pt}%
\pgfpathmoveto{\pgfqpoint{1.936680in}{0.605056in}}%
\pgfpathlineto{\pgfqpoint{1.936680in}{0.765268in}}%
\pgfusepath{stroke}%
\end{pgfscope}%
\begin{pgfscope}%
\pgfpathrectangle{\pgfqpoint{0.956688in}{0.387285in}}{\pgfqpoint{1.833666in}{2.002942in}}%
\pgfusepath{clip}%
\pgfsetbuttcap%
\pgfsetroundjoin%
\pgfsetlinewidth{0.752812pt}%
\definecolor{currentstroke}{rgb}{0.078431,0.690196,0.239216}%
\pgfsetstrokecolor{currentstroke}%
\pgfsetdash{}{0pt}%
\pgfpathmoveto{\pgfqpoint{1.283327in}{1.024880in}}%
\pgfpathlineto{\pgfqpoint{2.084756in}{1.024880in}}%
\pgfusepath{stroke}%
\end{pgfscope}%
\begin{pgfscope}%
\pgfpathrectangle{\pgfqpoint{0.956688in}{0.387285in}}{\pgfqpoint{1.833666in}{2.002942in}}%
\pgfusepath{clip}%
\pgfsetbuttcap%
\pgfsetroundjoin%
\pgfsetlinewidth{0.752812pt}%
\definecolor{currentstroke}{rgb}{0.078431,0.690196,0.239216}%
\pgfsetstrokecolor{currentstroke}%
\pgfsetdash{}{0pt}%
\pgfpathmoveto{\pgfqpoint{1.684042in}{0.841201in}}%
\pgfpathlineto{\pgfqpoint{1.684042in}{1.208559in}}%
\pgfusepath{stroke}%
\end{pgfscope}%
\begin{pgfscope}%
\pgfpathrectangle{\pgfqpoint{0.956688in}{0.387285in}}{\pgfqpoint{1.833666in}{2.002942in}}%
\pgfusepath{clip}%
\pgfsetbuttcap%
\pgfsetroundjoin%
\definecolor{currentfill}{rgb}{0.298039,0.447059,0.690196}%
\pgfsetfillcolor{currentfill}%
\pgfsetlinewidth{0.000000pt}%
\definecolor{currentstroke}{rgb}{0.298039,0.447059,0.690196}%
\pgfsetstrokecolor{currentstroke}%
\pgfsetdash{}{0pt}%
\pgfsys@defobject{currentmarker}{\pgfqpoint{-0.048611in}{-0.048611in}}{\pgfqpoint{0.048611in}{0.048611in}}{%
\pgfpathmoveto{\pgfqpoint{0.000000in}{-0.048611in}}%
\pgfpathcurveto{\pgfqpoint{0.012892in}{-0.048611in}}{\pgfqpoint{0.025257in}{-0.043489in}}{\pgfqpoint{0.034373in}{-0.034373in}}%
\pgfpathcurveto{\pgfqpoint{0.043489in}{-0.025257in}}{\pgfqpoint{0.048611in}{-0.012892in}}{\pgfqpoint{0.048611in}{0.000000in}}%
\pgfpathcurveto{\pgfqpoint{0.048611in}{0.012892in}}{\pgfqpoint{0.043489in}{0.025257in}}{\pgfqpoint{0.034373in}{0.034373in}}%
\pgfpathcurveto{\pgfqpoint{0.025257in}{0.043489in}}{\pgfqpoint{0.012892in}{0.048611in}}{\pgfqpoint{0.000000in}{0.048611in}}%
\pgfpathcurveto{\pgfqpoint{-0.012892in}{0.048611in}}{\pgfqpoint{-0.025257in}{0.043489in}}{\pgfqpoint{-0.034373in}{0.034373in}}%
\pgfpathcurveto{\pgfqpoint{-0.043489in}{0.025257in}}{\pgfqpoint{-0.048611in}{0.012892in}}{\pgfqpoint{-0.048611in}{0.000000in}}%
\pgfpathcurveto{\pgfqpoint{-0.048611in}{-0.012892in}}{\pgfqpoint{-0.043489in}{-0.025257in}}{\pgfqpoint{-0.034373in}{-0.034373in}}%
\pgfpathcurveto{\pgfqpoint{-0.025257in}{-0.043489in}}{\pgfqpoint{-0.012892in}{-0.048611in}}{\pgfqpoint{0.000000in}{-0.048611in}}%
\pgfpathlineto{\pgfqpoint{0.000000in}{-0.048611in}}%
\pgfpathclose%
\pgfusepath{fill}%
}%
\begin{pgfscope}%
\pgfsys@transformshift{2.234141in}{1.911346in}%
\pgfsys@useobject{currentmarker}{}%
\end{pgfscope}%
\end{pgfscope}%
\begin{pgfscope}%
\pgfpathrectangle{\pgfqpoint{0.956688in}{0.387285in}}{\pgfqpoint{1.833666in}{2.002942in}}%
\pgfusepath{clip}%
\pgfsetbuttcap%
\pgfsetroundjoin%
\definecolor{currentfill}{rgb}{0.921569,0.505882,0.105882}%
\pgfsetfillcolor{currentfill}%
\pgfsetlinewidth{0.000000pt}%
\definecolor{currentstroke}{rgb}{0.921569,0.505882,0.105882}%
\pgfsetstrokecolor{currentstroke}%
\pgfsetdash{}{0pt}%
\pgfsys@defobject{currentmarker}{\pgfqpoint{-0.048611in}{-0.048611in}}{\pgfqpoint{0.048611in}{0.048611in}}{%
\pgfpathmoveto{\pgfqpoint{0.000000in}{-0.048611in}}%
\pgfpathcurveto{\pgfqpoint{0.012892in}{-0.048611in}}{\pgfqpoint{0.025257in}{-0.043489in}}{\pgfqpoint{0.034373in}{-0.034373in}}%
\pgfpathcurveto{\pgfqpoint{0.043489in}{-0.025257in}}{\pgfqpoint{0.048611in}{-0.012892in}}{\pgfqpoint{0.048611in}{0.000000in}}%
\pgfpathcurveto{\pgfqpoint{0.048611in}{0.012892in}}{\pgfqpoint{0.043489in}{0.025257in}}{\pgfqpoint{0.034373in}{0.034373in}}%
\pgfpathcurveto{\pgfqpoint{0.025257in}{0.043489in}}{\pgfqpoint{0.012892in}{0.048611in}}{\pgfqpoint{0.000000in}{0.048611in}}%
\pgfpathcurveto{\pgfqpoint{-0.012892in}{0.048611in}}{\pgfqpoint{-0.025257in}{0.043489in}}{\pgfqpoint{-0.034373in}{0.034373in}}%
\pgfpathcurveto{\pgfqpoint{-0.043489in}{0.025257in}}{\pgfqpoint{-0.048611in}{0.012892in}}{\pgfqpoint{-0.048611in}{0.000000in}}%
\pgfpathcurveto{\pgfqpoint{-0.048611in}{-0.012892in}}{\pgfqpoint{-0.043489in}{-0.025257in}}{\pgfqpoint{-0.034373in}{-0.034373in}}%
\pgfpathcurveto{\pgfqpoint{-0.025257in}{-0.043489in}}{\pgfqpoint{-0.012892in}{-0.048611in}}{\pgfqpoint{0.000000in}{-0.048611in}}%
\pgfpathlineto{\pgfqpoint{0.000000in}{-0.048611in}}%
\pgfpathclose%
\pgfusepath{fill}%
}%
\begin{pgfscope}%
\pgfsys@transformshift{1.936680in}{0.685162in}%
\pgfsys@useobject{currentmarker}{}%
\end{pgfscope}%
\end{pgfscope}%
\begin{pgfscope}%
\pgfpathrectangle{\pgfqpoint{0.956688in}{0.387285in}}{\pgfqpoint{1.833666in}{2.002942in}}%
\pgfusepath{clip}%
\pgfsetbuttcap%
\pgfsetroundjoin%
\definecolor{currentfill}{rgb}{0.078431,0.690196,0.239216}%
\pgfsetfillcolor{currentfill}%
\pgfsetlinewidth{0.000000pt}%
\definecolor{currentstroke}{rgb}{0.078431,0.690196,0.239216}%
\pgfsetstrokecolor{currentstroke}%
\pgfsetdash{}{0pt}%
\pgfsys@defobject{currentmarker}{\pgfqpoint{-0.048611in}{-0.048611in}}{\pgfqpoint{0.048611in}{0.048611in}}{%
\pgfpathmoveto{\pgfqpoint{0.000000in}{-0.048611in}}%
\pgfpathcurveto{\pgfqpoint{0.012892in}{-0.048611in}}{\pgfqpoint{0.025257in}{-0.043489in}}{\pgfqpoint{0.034373in}{-0.034373in}}%
\pgfpathcurveto{\pgfqpoint{0.043489in}{-0.025257in}}{\pgfqpoint{0.048611in}{-0.012892in}}{\pgfqpoint{0.048611in}{0.000000in}}%
\pgfpathcurveto{\pgfqpoint{0.048611in}{0.012892in}}{\pgfqpoint{0.043489in}{0.025257in}}{\pgfqpoint{0.034373in}{0.034373in}}%
\pgfpathcurveto{\pgfqpoint{0.025257in}{0.043489in}}{\pgfqpoint{0.012892in}{0.048611in}}{\pgfqpoint{0.000000in}{0.048611in}}%
\pgfpathcurveto{\pgfqpoint{-0.012892in}{0.048611in}}{\pgfqpoint{-0.025257in}{0.043489in}}{\pgfqpoint{-0.034373in}{0.034373in}}%
\pgfpathcurveto{\pgfqpoint{-0.043489in}{0.025257in}}{\pgfqpoint{-0.048611in}{0.012892in}}{\pgfqpoint{-0.048611in}{0.000000in}}%
\pgfpathcurveto{\pgfqpoint{-0.048611in}{-0.012892in}}{\pgfqpoint{-0.043489in}{-0.025257in}}{\pgfqpoint{-0.034373in}{-0.034373in}}%
\pgfpathcurveto{\pgfqpoint{-0.025257in}{-0.043489in}}{\pgfqpoint{-0.012892in}{-0.048611in}}{\pgfqpoint{0.000000in}{-0.048611in}}%
\pgfpathlineto{\pgfqpoint{0.000000in}{-0.048611in}}%
\pgfpathclose%
\pgfusepath{fill}%
}%
\begin{pgfscope}%
\pgfsys@transformshift{1.684042in}{1.024880in}%
\pgfsys@useobject{currentmarker}{}%
\end{pgfscope}%
\end{pgfscope}%
\begin{pgfscope}%
\pgfsetrectcap%
\pgfsetmiterjoin%
\pgfsetlinewidth{1.003750pt}%
\definecolor{currentstroke}{rgb}{0.917647,0.917647,0.949020}%
\pgfsetstrokecolor{currentstroke}%
\pgfsetdash{}{0pt}%
\pgfpathmoveto{\pgfqpoint{0.956688in}{0.387285in}}%
\pgfpathlineto{\pgfqpoint{0.956688in}{2.390228in}}%
\pgfusepath{stroke}%
\end{pgfscope}%
\begin{pgfscope}%
\pgfsetrectcap%
\pgfsetmiterjoin%
\pgfsetlinewidth{1.003750pt}%
\definecolor{currentstroke}{rgb}{0.917647,0.917647,0.949020}%
\pgfsetstrokecolor{currentstroke}%
\pgfsetdash{}{0pt}%
\pgfpathmoveto{\pgfqpoint{2.790353in}{0.387285in}}%
\pgfpathlineto{\pgfqpoint{2.790353in}{2.390228in}}%
\pgfusepath{stroke}%
\end{pgfscope}%
\begin{pgfscope}%
\pgfsetrectcap%
\pgfsetmiterjoin%
\pgfsetlinewidth{1.003750pt}%
\definecolor{currentstroke}{rgb}{0.917647,0.917647,0.949020}%
\pgfsetstrokecolor{currentstroke}%
\pgfsetdash{}{0pt}%
\pgfpathmoveto{\pgfqpoint{0.956688in}{0.387285in}}%
\pgfpathlineto{\pgfqpoint{2.790353in}{0.387285in}}%
\pgfusepath{stroke}%
\end{pgfscope}%
\begin{pgfscope}%
\pgfsetrectcap%
\pgfsetmiterjoin%
\pgfsetlinewidth{1.003750pt}%
\definecolor{currentstroke}{rgb}{0.917647,0.917647,0.949020}%
\pgfsetstrokecolor{currentstroke}%
\pgfsetdash{}{0pt}%
\pgfpathmoveto{\pgfqpoint{0.956688in}{2.390228in}}%
\pgfpathlineto{\pgfqpoint{2.790353in}{2.390228in}}%
\pgfusepath{stroke}%
\end{pgfscope}%
\begin{pgfscope}%
\pgfsetbuttcap%
\pgfsetroundjoin%
\definecolor{currentfill}{rgb}{0.298039,0.447059,0.690196}%
\pgfsetfillcolor{currentfill}%
\pgfsetlinewidth{0.000000pt}%
\definecolor{currentstroke}{rgb}{0.298039,0.447059,0.690196}%
\pgfsetstrokecolor{currentstroke}%
\pgfsetdash{}{0pt}%
\pgfsys@defobject{currentmarker}{\pgfqpoint{-0.048611in}{-0.048611in}}{\pgfqpoint{0.048611in}{0.048611in}}{%
\pgfpathmoveto{\pgfqpoint{0.000000in}{-0.048611in}}%
\pgfpathcurveto{\pgfqpoint{0.012892in}{-0.048611in}}{\pgfqpoint{0.025257in}{-0.043489in}}{\pgfqpoint{0.034373in}{-0.034373in}}%
\pgfpathcurveto{\pgfqpoint{0.043489in}{-0.025257in}}{\pgfqpoint{0.048611in}{-0.012892in}}{\pgfqpoint{0.048611in}{0.000000in}}%
\pgfpathcurveto{\pgfqpoint{0.048611in}{0.012892in}}{\pgfqpoint{0.043489in}{0.025257in}}{\pgfqpoint{0.034373in}{0.034373in}}%
\pgfpathcurveto{\pgfqpoint{0.025257in}{0.043489in}}{\pgfqpoint{0.012892in}{0.048611in}}{\pgfqpoint{0.000000in}{0.048611in}}%
\pgfpathcurveto{\pgfqpoint{-0.012892in}{0.048611in}}{\pgfqpoint{-0.025257in}{0.043489in}}{\pgfqpoint{-0.034373in}{0.034373in}}%
\pgfpathcurveto{\pgfqpoint{-0.043489in}{0.025257in}}{\pgfqpoint{-0.048611in}{0.012892in}}{\pgfqpoint{-0.048611in}{0.000000in}}%
\pgfpathcurveto{\pgfqpoint{-0.048611in}{-0.012892in}}{\pgfqpoint{-0.043489in}{-0.025257in}}{\pgfqpoint{-0.034373in}{-0.034373in}}%
\pgfpathcurveto{\pgfqpoint{-0.025257in}{-0.043489in}}{\pgfqpoint{-0.012892in}{-0.048611in}}{\pgfqpoint{0.000000in}{-0.048611in}}%
\pgfpathlineto{\pgfqpoint{0.000000in}{-0.048611in}}%
\pgfpathclose%
\pgfusepath{fill}%
}%
\begin{pgfscope}%
\pgfsys@transformshift{3.001464in}{1.784047in}%
\pgfsys@useobject{currentmarker}{}%
\end{pgfscope}%
\end{pgfscope}%
\begin{pgfscope}%
\definecolor{textcolor}{rgb}{0.137255,0.215686,0.231373}%
\pgfsetstrokecolor{textcolor}%
\pgfsetfillcolor{textcolor}%
\pgftext[x=3.201464in, y=1.818921in, left, base]{\color{textcolor}\sffamily\fontsize{8.000000}{9.600000}\selectfont One-hot enc.}%
\end{pgfscope}%
\begin{pgfscope}%
\definecolor{textcolor}{rgb}{0.137255,0.215686,0.231373}%
\pgfsetstrokecolor{textcolor}%
\pgfsetfillcolor{textcolor}%
\pgftext[x=3.201464in, y=1.694507in, left, base]{\color{textcolor}\sffamily\fontsize{8.000000}{9.600000}\selectfont Cross Entropy}%
\end{pgfscope}%
\begin{pgfscope}%
\pgfsetbuttcap%
\pgfsetroundjoin%
\definecolor{currentfill}{rgb}{0.921569,0.505882,0.105882}%
\pgfsetfillcolor{currentfill}%
\pgfsetlinewidth{0.000000pt}%
\definecolor{currentstroke}{rgb}{0.921569,0.505882,0.105882}%
\pgfsetstrokecolor{currentstroke}%
\pgfsetdash{}{0pt}%
\pgfsys@defobject{currentmarker}{\pgfqpoint{-0.048611in}{-0.048611in}}{\pgfqpoint{0.048611in}{0.048611in}}{%
\pgfpathmoveto{\pgfqpoint{0.000000in}{-0.048611in}}%
\pgfpathcurveto{\pgfqpoint{0.012892in}{-0.048611in}}{\pgfqpoint{0.025257in}{-0.043489in}}{\pgfqpoint{0.034373in}{-0.034373in}}%
\pgfpathcurveto{\pgfqpoint{0.043489in}{-0.025257in}}{\pgfqpoint{0.048611in}{-0.012892in}}{\pgfqpoint{0.048611in}{0.000000in}}%
\pgfpathcurveto{\pgfqpoint{0.048611in}{0.012892in}}{\pgfqpoint{0.043489in}{0.025257in}}{\pgfqpoint{0.034373in}{0.034373in}}%
\pgfpathcurveto{\pgfqpoint{0.025257in}{0.043489in}}{\pgfqpoint{0.012892in}{0.048611in}}{\pgfqpoint{0.000000in}{0.048611in}}%
\pgfpathcurveto{\pgfqpoint{-0.012892in}{0.048611in}}{\pgfqpoint{-0.025257in}{0.043489in}}{\pgfqpoint{-0.034373in}{0.034373in}}%
\pgfpathcurveto{\pgfqpoint{-0.043489in}{0.025257in}}{\pgfqpoint{-0.048611in}{0.012892in}}{\pgfqpoint{-0.048611in}{0.000000in}}%
\pgfpathcurveto{\pgfqpoint{-0.048611in}{-0.012892in}}{\pgfqpoint{-0.043489in}{-0.025257in}}{\pgfqpoint{-0.034373in}{-0.034373in}}%
\pgfpathcurveto{\pgfqpoint{-0.025257in}{-0.043489in}}{\pgfqpoint{-0.012892in}{-0.048611in}}{\pgfqpoint{0.000000in}{-0.048611in}}%
\pgfpathlineto{\pgfqpoint{0.000000in}{-0.048611in}}%
\pgfpathclose%
\pgfusepath{fill}%
}%
\begin{pgfscope}%
\pgfsys@transformshift{3.001464in}{1.385436in}%
\pgfsys@useobject{currentmarker}{}%
\end{pgfscope}%
\end{pgfscope}%
\begin{pgfscope}%
\definecolor{textcolor}{rgb}{0.137255,0.215686,0.231373}%
\pgfsetstrokecolor{textcolor}%
\pgfsetfillcolor{textcolor}%
\pgftext[x=3.201464in, y=1.420310in, left, base]{\color{textcolor}\sffamily\fontsize{8.000000}{9.600000}\selectfont Hierarchical enc.}%
\end{pgfscope}%
\begin{pgfscope}%
\definecolor{textcolor}{rgb}{0.137255,0.215686,0.231373}%
\pgfsetstrokecolor{textcolor}%
\pgfsetfillcolor{textcolor}%
\pgftext[x=3.201464in, y=1.295896in, left, base]{\color{textcolor}\sffamily\fontsize{8.000000}{9.600000}\selectfont Cross Entropy}%
\end{pgfscope}%
\begin{pgfscope}%
\pgfsetbuttcap%
\pgfsetroundjoin%
\definecolor{currentfill}{rgb}{0.078431,0.690196,0.239216}%
\pgfsetfillcolor{currentfill}%
\pgfsetlinewidth{0.000000pt}%
\definecolor{currentstroke}{rgb}{0.078431,0.690196,0.239216}%
\pgfsetstrokecolor{currentstroke}%
\pgfsetdash{}{0pt}%
\pgfsys@defobject{currentmarker}{\pgfqpoint{-0.048611in}{-0.048611in}}{\pgfqpoint{0.048611in}{0.048611in}}{%
\pgfpathmoveto{\pgfqpoint{0.000000in}{-0.048611in}}%
\pgfpathcurveto{\pgfqpoint{0.012892in}{-0.048611in}}{\pgfqpoint{0.025257in}{-0.043489in}}{\pgfqpoint{0.034373in}{-0.034373in}}%
\pgfpathcurveto{\pgfqpoint{0.043489in}{-0.025257in}}{\pgfqpoint{0.048611in}{-0.012892in}}{\pgfqpoint{0.048611in}{0.000000in}}%
\pgfpathcurveto{\pgfqpoint{0.048611in}{0.012892in}}{\pgfqpoint{0.043489in}{0.025257in}}{\pgfqpoint{0.034373in}{0.034373in}}%
\pgfpathcurveto{\pgfqpoint{0.025257in}{0.043489in}}{\pgfqpoint{0.012892in}{0.048611in}}{\pgfqpoint{0.000000in}{0.048611in}}%
\pgfpathcurveto{\pgfqpoint{-0.012892in}{0.048611in}}{\pgfqpoint{-0.025257in}{0.043489in}}{\pgfqpoint{-0.034373in}{0.034373in}}%
\pgfpathcurveto{\pgfqpoint{-0.043489in}{0.025257in}}{\pgfqpoint{-0.048611in}{0.012892in}}{\pgfqpoint{-0.048611in}{0.000000in}}%
\pgfpathcurveto{\pgfqpoint{-0.048611in}{-0.012892in}}{\pgfqpoint{-0.043489in}{-0.025257in}}{\pgfqpoint{-0.034373in}{-0.034373in}}%
\pgfpathcurveto{\pgfqpoint{-0.025257in}{-0.043489in}}{\pgfqpoint{-0.012892in}{-0.048611in}}{\pgfqpoint{0.000000in}{-0.048611in}}%
\pgfpathlineto{\pgfqpoint{0.000000in}{-0.048611in}}%
\pgfpathclose%
\pgfusepath{fill}%
}%
\begin{pgfscope}%
\pgfsys@transformshift{3.001464in}{0.986825in}%
\pgfsys@useobject{currentmarker}{}%
\end{pgfscope}%
\end{pgfscope}%
\begin{pgfscope}%
\definecolor{textcolor}{rgb}{0.137255,0.215686,0.231373}%
\pgfsetstrokecolor{textcolor}%
\pgfsetfillcolor{textcolor}%
\pgftext[x=3.201464in, y=1.021699in, left, base]{\color{textcolor}\sffamily\fontsize{8.000000}{9.600000}\selectfont Description enc.}%
\end{pgfscope}%
\begin{pgfscope}%
\definecolor{textcolor}{rgb}{0.137255,0.215686,0.231373}%
\pgfsetstrokecolor{textcolor}%
\pgfsetfillcolor{textcolor}%
\pgftext[x=3.201464in, y=0.897285in, left, base]{\color{textcolor}\sffamily\fontsize{8.000000}{9.600000}\selectfont Cosine Distance}%
\end{pgfscope}%
\end{pgfpicture}%
\makeatother%
\endgroup%

  \note[item]{TODO}
\end{frame}


  \section{Future Developments}
  \begin{frame}{Future Developments}
  Work in progress:
  \begin{itemize}
    \item Hierarchical Encoding Experiments
    \item Descriptions Encoding Experiments
  \end{itemize}
  Next:
  \begin{itemize}
    \item Test robustness against Adversarial Attacks
    \item Improve description encodings
    \item Write thesis document
  \end{itemize}

  \note[item]{Al momento stiamo facendo di training dei modelli che fanno uso
  di Hierarchical Encoding e Descriptions Encoding}
  \note[item]{Una volta addestrati i modelli, ne si vuole testare la robustezza
  contro attacchi avversari.}
  \note[item]{Vi sono ancora degli aspetti da approfondire per i description encoding.
  La qualità delle descrizioni, quindi variando il prompt del modello generativo di testo.
  O ancora gli iperparametri delle funzioni di riduzione della dimensionalità
  come numero di componenti e sparsità degli encoding.}
  \note[item]{Infine deve essere scritta la vera e propria tesi.}
\end{frame}


  % \begin{frame}[standout]
  %   Thank you!
  % \end{frame}

\end{document}
