\chapter{Related Work}
\label{ch:related-work}

% TODO: Image classification
% TODO: XAI
% TODO: Read papers

\section{Image Classification with Neural Network}
\label{sec:image-classification-with-neural-network}

One of the first instances where Neural Network were employed for image
classification was in 1998 by Lecun et al.~\cite{GradientBasedLecun1998}. Their
model, LeNet, was able to indentify handwritten digit with higher accuracy than
previous methods which relyed on manual fetures extraction such edges detection
and hardcoded pattern recognition. A raw image of a digit is passed as input to LeNet and
the model spits out a number from 0 to 9. Some foundational ideas were at the
core of the architecture (convolutional operations, gradient based
optimization, pooling layers) stood the test of time and proved to be highly
sucessful.

Fastforwanding to 2012. AlexNet~\cite{ImagenetClassiKrizhe2017}, a Convolution
Neural Network (CNN) similar in spirit to LeNet, annihilated competitions at
ImageNet Large Scale Visual Recognition Challenge~\cite{ImagenetALarDeng2009}
with a whopping 10.8 \% gap on the runner up in top-5 error metric. This result
proved the potential of CNNs to process bigger color images (224 x 224 x 3) and
this sparked interest in the field which abbandoned the idea of manual
featueres extraction in favour of the end-to-end approach.



% LeNet (MNIST) first CNN applyed successfully to image classification
% AlexNet (ImageNet) first CNN to win the ImageNet challenge
% GoogLeNet and VGGNet popularized the idea of going deeper using fix block (e.g. inceptions)
% ResNet introduce the idea of skip connections
% EfficientNet formalized scaling direcitons width vs depth for Squeeze-Extitaion block (SENet)
% RegNet improve skip connections by introduciong regulator module hybrid approach.
