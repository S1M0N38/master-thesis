\chapter{Related Work}
\label{ch:related-work}

\section{Image Classification with Neural Network}
\label{sec:image-classification-with-neural-network}

One of the first instances where Neural Network were employed for image
classification was in 1998 by Lecun et al.~\cite{GradientBasedLecun1998}. Their
model, LeNet-5, was able to identify handwritten digit with higher accuracy
than previous methods which relied on manual features extraction such edges
detection and hard-coded pattern recognition. A raw image of a digit is passed
as input to LeNet and the model spits out a number from 0 to 9. Some
foundational ideas were at the core of the architecture (convolutional
operations, gradient based optimization, pooling layers) stood the test of time
and proved to be highly successful.

Fast forwarding to 2012. AlexNet~\cite{ImagenetClassiKrizhe2017}, a Convolution
Neural Network (CNN) similar in spirit to LeNet-5, annihilated competitions at
ImageNet Large Scale Visual Recognition Challenge~\cite{ImagenetALarDeng2009}
(ILSVRC) with a whopping 10.8 \% gap on the runner-up in top-5 error metric.
This result proved the potential of CNNs to process bigger color images (224 ×
224 × 3) and this sparked interest in the field which abandoned the idea of
manual features extraction in favour of the end-to-end approach.

Researchers investigated various moving part of CNNs architecture: depth of the
model, optimal kernel size, number of filters, pooling operations.
\cite{VeryDeepConvoSimony2014,GoingDeeperWiSzeged2014} focus on increasing the
depth of the models and \cite{GoingDeeperWiSzeged2014} was able to do so by the
introduction of a highly efficient ``Inception module'' which was able to increase
the depth while keeping the FLOPs under control. Efficiency at inference time is
crucial for embedding models into resource constrained devices such smartphones or
IOTs~\cite{SqueezenetAleIandol2016, MnasnetPlatfoTanM2018,
MobilenetsEffHoward2017}.

A problem that arose with the increase in depth was the so-called ``vanishing
gradient''. The gradient of the loss function with respect to the model's
parameters in the early layers tends to zero and this affect the ability of the
optimizer to update the weights towards a local minimum. Residual connections,
introduced in ResNet~\cite{DeepResidualLHeKa2015, IdentityMappinHeKa2016}, are
shortcuts in the network architecture that circumvent the problem allowing the
gradient to flow backwards effectively, so deeper models can be trained.
Connections between layers, i.e. the topology of the network, was further
explored by~\cite{DualPathNetwoChen2017, DenselyConnectHuang2016} and prior
using evolutionary methods by~\cite{DesigningNeuraMiller1989,
EvolvingNeuralStanle2002}.

Neural Architecture Search was employed to search for the width and the depth
of a model given a basic building block~\cite{ProgressiveNeuLiuC2017,
GeneticCnnXieL2017, LargeScaleEvoReal2017, RegularizedEvoReal2018}. Using
different building blocks and optimizing for different budget types (e.g.
FLOPs, number of parameters, inference time) various architecture were
produced. EfficientNets~\cite{EfficientnetRTanM2019, Efficientnetv2TanM2021}
are a well-known example of such approach that make use of
Squeeze-and-Excitation module~\cite{SqueezeAndExcHuJi2017} as basic component.

Stacks of convolution blocks followed by a fully connected layers start to
reach the limit regarding classification accuracy. CNNs were equipped with an
attention mechanism be able to effectively process global features in an image
defining hybrid architecture~\cite{SpatialTransfoJaderb2015,
LookAndThinkCaoC2015, ShowAttendAnXuKe2015, ScaCnnSpatiaChen2016}. Recently
interest shifted towards Vision Transformers
(ViT)~\cite{AnImageIsWorDosovi2020}, architecture that use solely attention
mechanism for features extraction which are able to achieve state-of-the-art
results in some well-established image classification
problems~\cite{TransformersInKhan2021, ASurveyOnVisHanK2023}. ViT required to
be trained of huge datasets to achieve competitive performance and, at the
moment of the writing, cannot run on consumer and low power devices due to high
computational cost.

\section{Semantic Information Representations}
\label{sec:semantic-information-sources}
Image classification consists in assigning a label to an image given a fixed
set of labels. Without any additional information about labels they are just a
set of symbols with no structure nor relationship between them. On the other
hand having richer information on labels turns out to be beneficial for model
interpretability~\cite{ImprovingInterDong2017}, image
summarization~\cite{SemanticImagePasini2022} and image classification as
well~\cite{MakingBetterMBertin2019}. Assuming that label a label of an image is
an English word indicating what is depicted in the image, we can obtain extra
information about a label by various sources which produce different representations.

\paragraph{Hierarchies (trees)}
\label{par:hierarchies-tree}
WordNet~\cite{WordnetMi1995} is a lexical database for the English language
where words are linked together by semantic relations (synonyms, hyponyms and
meronyms) resulting in a graph structure. The aforementioned ImageNet database
was build using WordNet as a source of labels, so information from WordNet could
be exploited when training a model on ImageNet. Other examples of semantically
enriched datasets include iNaturalist~\cite{TheInaturalistHorn2017} (taxonomy)
and CIFAR-100~\cite{LearningMultipKrizhe2009} (handcrafted). Given a graph
structure, it can be pruned in such a way to obtain a rooted tree, that is a
connected acyclic undirected graph with one vertex as root. Moreover, we can
assign a natural orientation to edges towards (or away from) the root. The
resulting data structure can be used to represent hierarchical organised
knowledge such ontologies or taxonomies.

\paragraph{Embeddings (vectors)}
\label{par:embeddings-vectors}
The majority of times labels are just a set of words with no structure between,
so semantic structure has to be constructed from bare words. A flexible
approach is to resort to internal representation of a pre-trained \acrfull{lm}
commonly knows as word embeddings~\cite{BeyondWordEmbIncitt2023}.
The extracted semantic structure is organised as a vector space where words are
represented as real-valued vectors. The input to a \acrshort{lm} can be the raw label or a
detailed description of the label. Additionally, such a description can be written
by human experts, scrape from the web or generated by a \acrshort{lm} itself.\medskip

These two approaches produce different representations of semantic information
(tree vs vectors) so exist different ways to inject it into a model. It is
worth mentioning that those are not the only source of external source of
information useful for vision models (e.g. human-annotated attributes were left
out). But those are the ones that requires the least human effort to be
exploit, a crucial aspect for training large models which requires huge
datasets.

\section{Semantic Information Injection}
\label{sec:semantic-information-injection}

Semantic Information can be injected in different part of a model: directly into
labels, in the loss function or by designing a custom architecture. Of course
the line between custom loss functions and custom architectures is somewhat
blurred because come up with a custom architecture usually requires
non-standard approach to its optimization.

\subsection{Hierarchies}
\label{subsec:hierarchies}
% Custom Labels Encoding
Adding semantic information directly at labels level requires to find an
encoding capable of representing the meaningful relationship described by
hierarchy tree. We use the term \emph{hierarchical encoding} to refer to an
encoding of labels that is able to represent the hierarchical relationship. The
most straightforward way to produce a hierarchical encoding is to define a
notion of similarity in the tree structure and port it to the encoding. That is
two labels similar in the hierarchical tree have similar encoding. The various
proposed approaches differs by the definition of similarity and the derivation
of the encoding. \cite{HierarchyBasedBarz2018} solve systems of equations
requiring that the dot product of two encoding is proportional to the
similarity in the hierarchical tree. \cite{MakingBetterMBertin2019} apply the
softmax function to the rows of pairwise similarity matrix derived from the
hierarchy. \cite{BeyondOneHotPerott2023} followed similar strategy but rescale
intervals instead of using softmax and combine the result with the standard
one-hot encoding. Redmon et al.~\cite{Yolo9000BetteRedmon2016} improved the
YOLO model~\cite{YouOnlyLookORedmon2015} by exploiting WordTree, a tree-like
structure derived from WordNet where labels are coming from different datasets.
They do not apply softmax over fine-grained classes but concatenate labels
encoding from different levels of the hierarchy and compute the softmax over
all sysnsets that are hyponyms of the same concept.

% Custom Loss
Another approach consists in defining a custom loss function that takes into
account the relationship between labels. The idea is that a properly constructed
function can steer the model towards a more hierarchical structured internal
representation. For nearest-neighbour classifiers, Verma et
al.~\cite{LearningHierarVerma2012} learn distances metrics for each node of the
hierarchy tree by using ``context sensitive loss'' accounting for distances
between nodes. Wu et al.~\cite{LearningToMakWuHu2016} add on top of CNN parallel
linked fully connected (FC) layers corresponding to hierarchy levels and used
their outputs as regularized for standard cross entropy loss. Conversely,
Alsallah et al.~\cite{DoConvolutionaAlsall2017} add FC heads at different depth
of an AlexNet model each of which contributing to the total loss. Bertinetto et
al.~\cite{MakingBetterMBertin2019} introduce ``hierarchical cross-entropy'', a
loss function that incorporates class hierarchy information by factorizing the
predicted class probabilities into conditional probabilities along the paths in
the hierarchy tree, and weighting the cross entropy of each conditional
probability based on the depth in the hierarchy. In a series of
works~\cite{ComplementObjeChen2019, ImprovingAdverChen2019,
LearningWithHChen2019}, Chen et al. introduce ``complement entropy loss
function'' and its hierarchical version. That loss revolves around lowering the
model probabilities of the wrong classes instead of increasing the one
associated to the correct class.

% Custom Architecture
Some of the aforementioned works~\cite{LearningToMakWuHu2016,
DoConvolutionaAlsall2017} implemented a custom architecture alongside with a
custom loss function as well. Grag et al.~\cite{LearningHierarGarg2022} propose
a architecture similar to~\cite{LearningToMakWuHu2016} (parallel FC layers per
hierarchical level as classifiers) but put more emphasis on forcing structure
in the feature space by using four terms loss function. The first term is the
standard cross entropy on fine-grained labels while the second is the
Jensen-Shannon divergence between the predictions of a coarse classifier and
soft label distributions derived from the finer-grained classifier predictions.
The others two term promoted a more discriminative (marginal loss) and a better
oriented (geometric consistency loss) feature space. In the context of
fine-grained visual classification, Chang et al.~\cite{YourFlamingoChang2020}
split the features vector in $K$-fold and use them to feed $K$ classifiers
where $K$ is the number of level in the hierarchy. In the forward pass the
features used in finer classifiers are also used in the coarser ones but a
gradient controller block the gradient propagation avoiding that fine-grained
features will be biased towards coarse-grained recognition. In this way there
is a disentanglement of the features space but still fine-level features help
coarse-level classifiers.

\subsection{Embeddings}
\label{subsec:embeddings}
% Make use of semantic information in computer vision is not a new idea. In fact,
% it was an active area of research before the advent of deep learning.
% Researchers employed it to improve models classification
% accuracy~\cite{ScalableRecognNister2006, SemanticHierarMarsza2007,
% ExploitingObjeZweig2007, LearningMultipKrizhe2009} and to cope with a lot of
% classes~\cite{WhatDoesClassDeng2010}. Accuracy was dramatically improved with
% the introduction of CNNs and the interest in semantic information injection
% faded. The recent revamped interest is concerned with different challenges:
% Zero-Shot-Learning, fine-grained classification, model explainability and
% quality of errors.

Extracting semantic information from bare labels text using \acrshort{lm} is a
very flexible approach that do not required additional information about labels
and, sometimes, not even the labels themselves. In image classification,
\acrfull{zsl} is a collection of techniques where a model is trained to
recognize classes that were not present during
training~\cite{AnIntroductionSoysal2020}. This can be achieved by leveraging
the capability of other pre-trained models and instilling such knowledge into
the image classification model. DeViSE~\cite{DeviseADeepFrome2013} use latent
information extracted from big text corpus, Wikipedia, to build capable
\acrshort{zsl} model. The key idea was the use of the recently introduced
learnable words embeddings~\cite{EfficientEstimMikolo2013,
DistributedRepMikolo2013}, later patented by Google as
Word2Vec~\cite{ComputingNumerMikolo2013}. It was shown
in~\cite{EvaluationOfOAkata2014, LabelEmbeddingAkata2015} that embeddings
coming from different source (e.g. hierarchy and text corpora) can carry
non-redundant information and can be combined to learn a better encoding. For
that reason various hybrid approach, in the spirit of DeViSE, has been
proposed. Akata et al.~\cite{EvaluationOfOAkata2014} propose \acrshort{zsl}
model specifically geared towards fine-grained classification. Liu et
al.~\cite{HyperbolicVisuLiuS2020} move parts of the model pipeline in
hyperbolic space arguing for its suitability regarding hierarchical
representation, evidence supported by~\cite{PoincareEmbeddNickel2017,
HyperbolicEntaGanea2018}.

% Talk a bit about cutting edge research and multimodal learning
Nowadays it seems that embeddings, i.e. latent representation of information as
real-valued vectors, are a powerful way to make models different talks and
represent concepts in a somewhat similar fashion. Transformer
architecture~\cite{AttentionIsAlVaswan2017} was introduced in the context of
Natural Language Processing. They proved to be a general-purpose differentiable
computer: expressive (in the forward pass), optimizable (via backpropagation
and gradient descent) and efficient (highly
parallelizable).\footnote{Paraphrasing a post on $\mathbb{X}$ by Andrej
Karpathy, a prominent figure in the field.} Then they were adapted for
processing images~\cite{ImageTransformParmar2018, EndToEndObjeCarion2020,
AnImageIsWorDosovi2020}, videos~\cite{VivitAVideoArnab2021,
TemporalContexShao2020}, audio~\cite{NeuralSpeechSLiNa2018,
SpeechTransforDong2018, ConformerConvGulati2020}. Now the cutting edge research
focus on processing different inputs types coming from different sources trying
to exploit the correlation between them. Those model are called multimodal
models~\cite{ASurveyOnMulYinS2023}.


\section{Adversarial Attacks}
\label{sec:adversarial-attacks}

% TODO:
% - read about various adversaral attacks
% - Examplain them breifly
% - Search for evidences that robustness can be improved with semantic injection
