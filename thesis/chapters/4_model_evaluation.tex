\chapter{Model evaluation}
\label{ch:model-evaluation}

In this chapter we will discuss all the parts involved in model's performance
evaluation, i.e. how to derived the predicted class from the model's
output~(\Cref{sec:decoding}), various metrics~(\Cref{sec:metrics}),
output's projections~(\Cref{sec:output-projections}) and adversarial attacks
(\Cref{sec:adversarial-attacks}).

\section{Decoding}
\label{sec:decoding}
As stated in~\Cref{subsec:one-hot-encoding}, when using one-hot encoding and
optimizing with cross-entropy loss, we force the model's output $\hat{y}$ to
pump up the probability associated with the correct class disregarding the
others. So the focus is only on a single component and, if model is well
trained, its index correspond to the ground truth class. In this case, is
sufficient to take the $\argmax$ over the model's output to obtained the
predicted class. Due to the trivial nature of this operation, people usually
gloss over it, but things are slightly different when using others
encodings.\medskip

Formally, getting the class from output can be thought as the inverse of the
encoding function $\phi$, that is
\begin{equation}
  \gls{decoder} : \mathcal{Y} \to \mathcal{C}
  : \hat{y} \mapsto \gls{predicted_class} := \gls{decoder}(\hat{y})
  \label{eq:decoding}
\end{equation}
hence the name \emph{decoding}. For one-hot encoding $\phi^{-1} := \argmax$,
i.e. a function that returns the index of the maximum value of the vector given
as input. In contrast others encodings make use of multiple components in
model's output during training so blindly taking the $\argmax$ throws away
potentially useful information. Still both loss functions employed try to make
the model's output similar to the ground truth encoding so it is reasonable to
consider the predicted class to be the one corresponding to the encoding most
similar to the model's output. To this end, we choose cosine similarity as
measure for such quantity, i.e.
\begin{equation}
  \phi^{-1}(\hat{y}) := \argmax_{c \in \mathcal{C}}
  \left( \gls{cosine_similarity} \left(\hat{y}, \phi(c) \right) \right)
  \label{eq:decoding-cos-sim}
\end{equation}
If $\phi$ is one-hot encoding~\Cref{eq:decoding-cos-sim} is equivalent to simply
take $\argmax$ over output vector.

An advantage of defining a decoding function based on encoding similarity
is that it works with different encoding schemes. For example selecting the
highest component in model's output trained using description encodings makes no
sense: the vector dimensions do not represent classes but are coordinates of a
point in a high dimensional space. What matters is the relative distance of that
point (class encoding) to others points (others class encodings). On the other
hand if components directly represent a class (e.g. hierarchical encodings) the
provided decoding function works as expected.
Another benefit is to leverage information coming from all output's components:
not only we can try to guess what class is present in the provided image but
also what the model thinks the image is definitely not (e.g. the class with the
encoding that differ the most from $\hat{y}$).\medskip

The decoding concept could be straightforwardly extended to
top-\gls{number_predicted_classes} predicted classes, that is instead of
returning the class associated to the most similar encoding, return the
top-\gls{number_predicted_classes} most similar ones.

% Talk about how it defeat adversarial attacks (maybe, in theory
% but not sure about practice)
% \paragraph{Example}
% Explain with example make the most sense


\section{Metrics}
\label{sec:metrics}
Russakovsky et al.~\cite{ImagenetLargeRussak2014} propose \acrfull{ilsvrc} which
become the de facto benchmark to evaluate image classification models. Even
though they propose the use top-\gls{number_predicted_classes} and hierarchical
measure to asses model's performance, they conclude ``[..] all three measures of
error (top-5, top-1, and hierarchical) produced the same ordering of results.
Thus, since \acrshort{ilsvrc} 2012 we have been exclusively using the top-5
metric which is the simplest and most suitable to the dataset.'' The deep
learning community stick to top-\gls{number_predicted_classes} accuracies which
are easier and computable for every dataset.

\paragraph {Quantity \& Quality}
\label{par:metrics-quantity-quality}
However this work focuses on developing not only models that produce the fewer
error but also less severe ones. While the top-\gls{number_predicted_classes}
accuracies are useful to capture the \emph{quantity} aspect, they do not provide
any in information about their \emph{quality}. For example confusing a
\texttt{dog} for a \texttt{cat} can be considered a milder error compared to
\texttt{dog} - \texttt{car} misclassification; both are mammals and as such
share some distinguishing features like fur, an head, a pairs of eyes, four
paws, etc. Cars present straight lines, uniform coated surface, absence of
typical traits of living things, etc. This notion of visual similarity is
somewhat encoded in the classes' hierarchy, so make sense to use distances on
hierarchical tree to quantify the mistake severity.\footnote{The assumption that
  a hierarchy carries information about visual characteristics is indeed strong
  and must be empirically checked for dataset-hierarchy pairs considered.
Datasets used in this work satisfy the ansatz so hierarchical similarity is a
good proxy for visual similarity.} In the following sections we introduce some
performance metrics that enable this quantitative/qualitative errors assessment.

\paragraph {Levels}
\label{par:metrics-levels}
Having access to datasets equipped with a hierarchy introduce a new dimension to
evaluate models across: theirs performance at different levels of the hierarchy.
With reference to toy dataset in~\Cref{fig:2_toy-dataset}, suppose we are
interested in a coarser grained classification which distinguish between Fruits,
Animals and Vehicles (level 1). In this case, after applying the decoding to
\gls{output} to get \gls{predicted_class}, we need to map the predicted class to
its ancestor in the hierarchy obtaining the \emph{level-1 predicted class}. Of
course we can produce and even coarser classification discriminating Natural vs
Artificial instances by mapping the level-1 predicted class to its ancestor
obtaining the \emph{level-2 predicted class} and so on. We denote the predicted
class at l-th level as $\hat{c}^l$ and $c^l$ its corresponding ground truth; the
index $l$ is omitted at level zero.

\subsection{Error Rate}
\label{subsec:error-rate}
No need to reinvent the wheel for give an estimate of the amount of error: top-1
accuracy (or simply accuracy) provide such estimation. However for easier
comparison with other metrics we use \emph{top-1 error rate} (or simply error
rate) which is defined as $1 - \,$accuracy. The usage of \emph{confusion matrix}
make the definition of error rate, and the subsequent metrics, straightforward
at the cost of extensibility to top-\gls{number_predicted_classes} versions.

\paragraph{Confusion Matrix}
\label{par:confusion-matrix}
It's a square matrix $|\mathcal{C}| \times |\mathcal{C}|$ which rows are
ground-truth classes \gls{class} and columns predicted classes
\gls{predicted_class}. It's entries $m_{ij}$ count the number of times a model
produce the pair $(c_i, \hat{c}_j)$ when evaluated on a dataset. For example a
model applied on the example dataset can produce the confusion matrices at
different hierarchical level as in~\Cref{fig:5_confusion-matrices}
\begin{figure}[htbp]
  \centering
  \begin{subfigure}{0.35\textwidth}
    \ctikzfig{5_confusion-matrix-level-0}
  \end{subfigure}
  \begin{subfigure}{0.30\textwidth}
    \ctikzfig{6_confusion-matrix-level-1}
  \end{subfigure}
  \begin{subfigure}{0.25\textwidth}
    \ctikzfig{7_confusion-matrix-level-2}
  \end{subfigure}
  \caption{Example of confusion matrix: $m_{ij}^l$}
  \label{fig:5_confusion-matrices}
\end{figure}
\medskip

Using confusion matrix the error rate at l-th level is sum of the elements
off-diagonal divided by the sum of all elements, i.e.
\begin{equation}
  \textrm{error rate} :=
  \sum_{i \ne j} m_{ij}^l \bigg/ \sum_{i, \,j} m_{ij}^l.
  \label{eq:error-rate}
\end{equation}


\subsection{Hierarchical Distance Mistake}
\label{subsec:hierarchical-distance-mistake}
\emph{Hierarchical distance mistake} is a metric that take into account the
hierarchy to quantify the severity of a mistake, i.e. make use of the
\acrshort{lca} height $h_{ij}$ as weighting factor
(example~\Cref{fig:lca-height-matrix}). The Hierarchical distance mistake is
define to be
\begin{equation}
  \textrm{hier.\ dist.\ mistake} :=
  \sum_{i, \, j} m_{ij}^l \, h_{ij}^l \bigg/ \sum_{i \ne j} m_{ij}^l.
  \label{eq:hierarchical-distance-mistake}
\end{equation}
So it is the average \acrshort{lca} height between predicted and ground-truth
classes. Note that $h_{ii}$ are all zeros so we are effectively summing only
weighted errors in the numerator while the denominator is the total number of
errors. Due to his definition it's not related to error rate, i.e. it's possible
to have a low error rate (few errors) but a high hierarchical distance mistake
(those errors are severe).

\paragraph{Error rate -- Hier.\ dist.\ mistake}
\label{par:error-rate-hier-dist-mistake}
In order to compare models at glance we can plot the error rate vs hierarchical
distance mistake on a two dimensional scatter plot. The x-axis is the
amount of error (error rate) while the y-axis is errors' severity. Each point of
the scatter plot correspond to a trained model and the overall best model is the
one in the bottom left corner. Moreover we can draw this plot for different
level in the hierarchy highlighting the fact that milder errors in a lower
levels usually results in less errors in higher ones.
\begin{figure}[htbp]
  \ctikzfig{8_error-rate-hier-dist-mistake}
  \caption{Quadrants in error rate vs hier.\ dist.\ mistake plot}
  \label{fig:error-rate-hier-dist-mistake}
\end{figure}

\subsection{Hierarchical Distance}
\label{subsec:hierarchical-distance}
\emph{Hierarchical distance} is somewhat hybrid: it account for both quality and
quantity of error. It's definition it's almost identical to hierarchical
distance mistake but instead of divining by the number of errors we divide by
the total number of sample in the dataset, i.e.
\begin{equation}
  \textrm{hier.\ dist.\ } :=
  \sum_{i, \, j} m_{ij}^l \, h_{ij}^l \bigg/ \sum_{i, \, j} m_{ij}^l.
  \label{eq:hierarchical-distance}
\end{equation}
If the model produce few errors the values off-diagonal in the confusion matrix
will be smaller which are the only ones multiply by non-zeros values, hence the
numerator is smaller. The denominator always stays the same resulting in a lower
value for hierarchical distance.

\section{Output projections}
\label{sec:output-projections}

\section{Adversarial attacks}
\label{sec:adversarial-attacks}

% TODO: How to get predicted class from model's output. Explain the difference
% between argmax and align+argmax For one-hot encoding the two approaches are
% equivalent. For others encodings there is some difference. Explore and Explain
% it This is a clever way to fully use information about what models predict,
% not only this take into account the fact that the highest probability is the
% correct class, but also what the model thinks the image is not. Effectively
% denoising the prediction, a sort of top-\gls{number_predicted_classes}
% approach with a single class as output.
%
% TODO: Motivate why we are evaluating all these different aspect
% TODO: Metric definitions.
% TODO: Features vector, clustering, plots ...
% TODO: Adversarial attack: def, how they works, strength weakness
