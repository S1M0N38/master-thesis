\makeglossaries

% Dimensions
\newglossaryentry{batch_size}{
  sort=dimensions,
  name={\ensuremath{B}},
  description={Batch size}
}
\newglossaryentry{channel_size}{
  sort=dimensions,
  name={\ensuremath{C}},
  description={Channel size}
}
\newglossaryentry{height_size}{
  sort=dimensions,
  name={\ensuremath{H}},
  description={Height size}
}
\newglossaryentry{width_size}{
  sort=dimensions,
  name={\ensuremath{W}},
  description={Width size}
}
\newglossaryentry{output_size}{
  sort=dimensions,
  name={\ensuremath{D}},
  description={Output and encoding size}
}
\newglossaryentry{hierarchy_levels}{
  sort=dimensions,
  name={\ensuremath{L}},
  description={Number of level in a hierarchy tree}
}
\newglossaryentry{dataset_size}{
  sort=dimensions,
  name={\ensuremath{N}},
  description={Number of images in the dataset}
}
\newglossaryentry{number_predicted_classes}{
  sort=dimensions,
  name={\ensuremath{K}},
  description={Number of predicted class}
}
\newglossaryentry{embedding_size}{
  sort=dimensions,
  name={\ensuremath{\tilde{D}}},
  description={Embedding size}
}

% Model symbols
% Data points
\newglossaryentry{input}{
  sort=model,
  name={\ensuremath{x}},
  description={Image given as input to the model}
}
\newglossaryentry{input_i}{
  sort=model,
  name={\ensuremath{x_i}},
  description={Image with i-th class as ground-truth}
}
\newglossaryentry{hidden}{
  sort=model,
  name={\ensuremath{h}},
  description={Feature vector produce by the penuiltime hidden layer}
}
\newglossaryentry{hidden_i}{
  sort=model,
  name={\ensuremath{h_i}},
  description={Feature vector of an image with i-th class as ground-truth}
}
\newglossaryentry{output}{
  sort=model,
  name={\ensuremath{\hat{y}}},
  description={Output of the model}
}
\newglossaryentry{output_i}{
  sort=model,
  name={\ensuremath{\hat{y_i}}},
  description={Output of the model of an image with i-th class as ground-truth}
}
\newglossaryentry{class}{
  sort=model,
  name={\ensuremath{c}},
  description={Class, the text associated to a given image}
}
\newglossaryentry{class_i}{
  sort=model,
  name={\ensuremath{c_i}},
  description={The i-th class in the list of classes}
}
\newglossaryentry{encoding}{
  sort=model,
  name={\ensuremath{y}},
  description={Encoding, a numerical representation of the class}
}
\newglossaryentry{encoding_i}{
  sort=model,
  name={\ensuremath{y_i}},
  description={Encoding for the i-th class in the list of classes}
}
\newglossaryentry{predicted_class}{
  sort=model,
  name={\ensuremath{\hat{c}}},
  description={Predicted class, text associated to a given image}
}
\newglossaryentry{predicted_class_i}{
  sort=model,
  name={\ensuremath{\hat{c_i}}},
  description={Predicted class of an image with i-th class as ground-truth}
}
% functions
\newglossaryentry{parameters}{
  sort=model,
  name={\ensuremath{\theta}},
  description={Parameters of the model}
}
\newglossaryentry{model}{
  sort=model,
  name={\ensuremath{\psi_\theta}},
  description={CNN backbone}
}
\newglossaryentry{encoder}{
  sort=model,
  name={\ensuremath{\phi}},
  description={Encoder, function maps class to encoding}
}
\newglossaryentry{decoder}{
  sort=model,
  name={\ensuremath{\phi^{-1}}},
  description={Encoder, function maps encoding to class}
}
\newglossaryentry{loss}{
  sort=model,
  name={\ensuremath{\Loss}},
  description={Loss function of the model}
}
% adversarial attacks 
\newglossaryentry{perturbation}{
  sort=symbols,
  name={\ensuremath{\delta_\epsilon}},
  description={Untargeted perturbation of image}
}
\newglossaryentry{perturbation_i}{
  sort=symbols,
  name={\ensuremath{\delta_{\epsilon, c_i}}},
  description={Targeted perturbation of image with the $c_i$ as targetd class}
}
\newglossaryentry{perturbed_input}{
  sort=symbols,
  name={\ensuremath{\tilde{x}}},
  description={The perturbed image, i.e.\ the adversarial example}
}

% Function
\newglossaryentry{cosine_similarity}{
  sort=functions,
  name={\ensuremath{\cos}},
  description={Cosine similarity}
}
\newglossaryentry{lca_fn}{
  sort=functions,
  name={\ensuremath{\lca}},
  description={Lowest common ancestor}
}
\newglossaryentry{height_fn}{
  sort=functions,
  name={\ensuremath{\height}},
  description={Height of a node in a tree}
}

% Symbols
\newglossaryentry{compose}{
  sort=symbols,
  name={\ensuremath{\circ}},
  description={Function composition; $f \circ g \, (x)$ is an alternative notation for $f(g(x))$}
}
\newglossaryentry{equiv}{
  sort=symbols,
  name={\ensuremath{\equiv}},
  description={\acrshort{lhs} is an alternative notation for \acrshort{rhs} and vice versa}
}
\newglossaryentry{define_lhs_from_rhs}{
  sort=symbols,
  name={\ensuremath{:=}},
  description={\acrshort{lhs} is defined to be equal to \acrshort{rhs}}
}
\newglossaryentry{define_rhs_from_lhs}{
  sort=symbols,
  name={\ensuremath{=:}},
  description={\acrshort{rhs} is defined to be equal to \acrshort{lhs}}
}
\newglossaryentry{kronecker_delta}{
  sort=symbols,
  name={\ensuremath{\delta_{ij}}},
  description={Kronecker delta}
}
\newglossaryentry{l2_norm}{
  sort=symbols,
  name={\ensuremath{\|\cdot\|}},
  description={\ensuremath{L^2} norm}
}
\newglossaryentry{len}{
  sort=symbols,
  name={\ensuremath{|\cdot|}},
  description={The number of elements in a set}
}
\newglossaryentry{dot_product}{
  sort=symbols,
  name={\ensuremath{\cdot}},
  description={Dot product, i.e. \ensuremath{\sum_{i=1}^n x_i y_i} where \ensuremath{x, y \in \mathbb{R}^n}}
}
\newglossaryentry{signum}{
  sort=symbols,
  name={\ensuremath{\textrm{sgn}}},
  description={The signum function. It returns 1 if the argument is positive, -1 otherwise.}
}
\newglossaryentry{grad}{
  sort=symbols,
  name={\ensuremath{\nabla_x}},
  description={Gradient with respect to \ensuremath{x}}
}

% Sets
\newglossaryentry{images_set}{
  sort=sets,
  name={\ensuremath{\mathcal{X}}},
  description={Images set}
}
\newglossaryentry{hidden_set}{
  sort=sets,
  name={\ensuremath{\mathcal{H}}},
  description={Features vectors set produce by penuiltime hidden layer}
}
\newglossaryentry{hidden_set_i}{
  sort=sets,
  name={\ensuremath{\mathcal{H}_i}},
  description={Features vectors set produce by penuiltime hidden layer of an image with i-th class as ground-truth}
}
\newglossaryentry{outputs_set}{
  sort=sets,
  name={\ensuremath{\hat{\mathcal{Y}}}},
  description={Outputs set}
}
\newglossaryentry{encoding_set}{
  sort=sets,
  name={\ensuremath{\mathcal{Y}}},
  description={Encodings set, i.e. $\{y_1, y_2, ..., y_{|C|}\}$}
}
\newglossaryentry{classes_set}{
  sort=sets,
  name={\ensuremath{\mathcal{C}}},
  description={Classes set, i.e. $\{c_1, c_2, ..., c_{|C|}\}$}
}
\newglossaryentry{descriptions_set}{
  sort=sets,
  name={\ensuremath{\mathcal{W}}},
  description={Descriptions set}
}
\newglossaryentry{vertices_set}{
  sort=sets,
  name={\ensuremath{\mathcal{V}}},
  description={Hierarchical tree vertices set}
}


% Acronyms
\newacronym{rhs}{RHS}{Right Hand Side}
\newacronym{lhs}{LHS}{Left Hand Side}
\newacronym{cnn}{CNN}{Convolutional Neural Network}
\newacronym{zsl}{ZSL}{Zero-Shot-Learning}
\newacronym{lm}{LM}{Language Model}
\newacronym{pdf}{PDF}{Probability Density Function}
\newacronym{pmf}{PMF}{Probability Mass Function}
\newacronym{gpu}{GPU}{Graphics Processing Unit}
\newacronym{lca}{LCA}{lowest common ancestor}
\newacronym{ilsvrc}{ILSVRC}{ImageNet Large Scale Visual Recognition Challenge}
\newacronym{nlp}{NLP}{Natural Language Processing}
\newacronym{pca}{PCA}{Principal Component Analysis}
\newacronym{xe}{XE}{Cross Entropy}
\newacronym{cd}{CD}{Cosine Distance}
\newacronym{fc}{FC}{fully connected}
\newacronym{flops}{FLOPs}{Floating Point Operations}
\newacronym{itos}{IoTs}{Internet of Things}
\newacronym{vit}{ViT}{Vision Transformer}
% Clustering Coefficients
\newacronym{sc}{SC}{Silhouette Coefficient}
\newacronym{chi}{CHI}{Calinski–Harabasz Index}
\newacronym{dbi}{DBI}{Davies–Bouldin Index}
\newacronym{sdbw}{SDBw}{Scatter + Density BetWeen Clusters}
% Adversarial attacks
\newacronym{fgsm}{FGSM}{Fast Gradient Sign Method}
\newacronym{bim}{BMI}{Basic Iterative Method}
\newacronym{pgd}{PGD}{Projected Gradient Descent}
\newacronym{jsma}{JSMA}{Jacobian-based Saliency Map Attack}
\newacronym{cw}{C\&W}{Carlini \& Wagner}
